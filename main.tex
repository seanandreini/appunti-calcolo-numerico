\documentclass{article}
\usepackage[utf8]{inputenc}
\usepackage[italian]{babel}
\usepackage{tikz}
\usepackage{amsmath}
\usepackage{amssymb}
\usepackage{amsthm}
\usepackage{geometry}
\usepackage{listings} % <-- QUESTA È LA RIGA CHE RISOLVE L'ERRORE
\usepackage{xcolor}   % <-- UTILE PER COLORARE IL CODICE
\usepackage{booktabs}
\usepackage{float}
\geometry{a4paper, margin=1in}


% --- Stili e Ambienti Personalizzati ---
\lstset{
  language=Matlab,
  basicstyle=\small\ttfamily,
  keywordstyle=\color{blue},
  commentstyle=\color{green!60!black},
  stringstyle=\color{purple},
  showstringspaces=false,
  breaklines=true,
  frame=single,
  numbers=left,
  literate=%
    {à}{{\`a}}1
    {è}{{\`e}}1
    {ì}{{\`i}}1
    {ò}{{\`o}}1
    {ù}{{\`u}}1
}

\newtheoremstyle{break}
  {\topsep}{\topsep}%
  {\normalfont}{}%
  {\bfseries}{.}%
  {\newline}{}%
  
\theoremstyle{break}
\newtheorem{osservazione}{Osservazione}[section]

\theoremstyle{definition}
\newtheorem{definition}{Definizione}[section]
\newtheorem{esercizio}{Esercizio}[section]
\newtheorem{teorema}{Teorema}[section]
\newtheorem{corollario}{Corollario}[section]
\newtheorem{esempio}{Esempio}[section]
% --- Fine Stili ---

\title{Appunti di Calcolo Numerico}
\author{} % Lasciato vuoto o da compilare
\date{\today}

\begin{document}

\maketitle

% Comando per generare l'indice
\tableofcontents

\newpage % Opzionale: per iniziare l'introduzione in una nuova pagina

% Sezione numerata che verrà aggiunta all'indice
\section{Introduzione}

Molti problemi derivanti da applicazioni pratiche sono descritti tramite un modello matematico. Una volta che le equazioni del modello sono risolte, è possibile fare inferenza sul fenomeno studiato. Tuttavia, la soluzione di tali equazioni non è quasi mai direttamente disponibile, rendendo necessario l'utilizzo di metodi numerici e tecniche di approssimazione.

Questi metodi devono soddisfare alcuni requisiti fondamentali:
\begin{itemize}
    \item \textbf{Accuratezza}: La soluzione approssimata deve essere sufficientemente "vicina" alla soluzione esatta, in base alle specifiche del problema.
    \item \textbf{Facilità di implementazione}: La formulazione del metodo deve consentire una semplice traduzione in un algoritmo da codificare in un opportuno linguaggio di programmazione.
\end{itemize}

\section{Errori ed aritmetica finita}

Il risultato fornito da un metodo numerico è quasi sempre affetto da errore. L'errore commesso è determinato da più cause, spesso intercalate tra loro.

\subsection{Misure dell'Errore}
Supponiamo che $x \in \mathbb{R}$ sia il dato esatto e $\tilde{x}$ la sua approssimazione.

\begin{definition}[Errore Assoluto]
L'errore assoluto è definito come la differenza:
$$ \Delta x = \tilde{x} - x $$
da cui segue $\tilde{x} = x + \Delta x$.
\end{definition}

Questa misura da sola non è completamente esaustiva. Ad esempio, un errore $|\Delta x| = 10^{-6}$ potrebbe essere considerato "grande" o "piccolo" solo rapportandolo al valore esatto $x$.

\begin{definition}[Errore Relativo]
Per ovviare a questo problema, se $x \neq 0$, si introduce l'errore relativo:
$$ \epsilon_x = \frac{\Delta x}{x} = \frac{\tilde{x} - x}{x} $$
\end{definition}

Dalla definizione segue che:
$$ \tilde{x} = x(1 + \epsilon_x) \quad \text{ovvero} \quad \frac{\tilde{x}}{x} = 1 + \epsilon_x $$
Questo mostra che l'errore relativo $\epsilon_x$ deve essere confrontato con 1. Un errore relativo $|\epsilon_x|=10^{-6}$ fornisce un'informazione più "assoluta" sulla qualità dell'approssimazione.

\subsection{Tipologie Principali di Errore}
È possibile individuare, almeno a livello concettuale, tre tipologie principali di errore:
\begin{enumerate}
    \item Errori di troncamento;
    \item Errori di iterazione;
    \item Errori di round-off (o arrotondamento).
\end{enumerate}

\subsubsection{Errori di Discretizzazione (o Troncamento)}
Questi errori nascono quando un problema matematico formulato nel continuo viene sostituito da un problema discreto che lo approssima.

Ad esempio, per calcolare la derivata di una funzione $f(x)$ in un punto $x_0$, si parte dalla definizione:
$$ f'(x_0) = \lim_{h \to 0} \frac{f(x_0+h) - f(x_0)}{h} $$
Per il calcolo numerico, si utilizza un valore di $\bar{h}>0$ (fissato) "piccolo" ma finito, approssimando la derivata con il rapporto incrementale:
$$ f'(x_0) \approx \frac{f(x_0+\bar{h}) - f(x_0)}{\bar{h}} $$
L'errore di troncamento commesso è la differenza tra le due quantità: $f'(x) - \frac{f(x+\bar{h})-f(x)}{\bar{h}}$.
Considerando lo sviluppo in serie di Taylor di $f(x+\bar{h})$ centrato in $x$:
$$ f(x+\bar{h}) = f(x) + \bar{h} f'(x) + \frac{\bar{h}^2}{2} f''(x) + \dots $$
si può isolare il rapporto incrementale e stimare l'errore:
$$ \frac{f(x+\bar{h}) - f(x)}{\bar{h}} = f'(x) + O(\bar{h}) $$
L'errore di troncamento è quindi del primo ordine rispetto ad $h$, ovvero $O(h)$.



\subsubsection{Errori di Convergenza (o Iterazione)}
Molti metodi numerici sono di tipo iterativo: non forniscono la soluzione esatta $x^*$, ma generano una successione di approssimazioni $\{x_n\}$. A partire da un'approssimazione iniziale $x_0$, le approssimazioni successive sono definite da una funzione di iterazione $\Phi(x)$:
$$ x_{n+1} = \Phi(x_n), \quad n=0, 1, 2, \dots $$

\begin{definition}[Metodo Convergente]
Un metodo iterativo si dice convergente se la successione delle approssimazioni tende alla soluzione esatta:
$$ \lim_{n \to \infty} x_n = x^* $$
\end{definition}

Poiché è possibile eseguire solo un numero finito di iterazioni, il processo viene arrestato a un indice $N$, e si utilizza $x_N$ come approssimazione di $x^*$. L'errore $x^* - x_N$ è detto errore di iterazione. L'indice $N$ è tipicamente determinato dinamicamente tramite un criterio di arresto.

\begin{osservazione}
    \begin{enumerate}
        \item L'errore di iterazione è legato all'utilizzo del metodo di base ($\Phi(x)$).
        \item Praticamente sempre, l'indice $N$ a cui si interrompe l'iterazione è determinato dinamicamente mediante un opportuno criterio di arresto.
    \end{enumerate}
\end{osservazione}

\subsubsection{Errori di Round-off}
Questi errori sono dovuti all'utilizzo dell'aritmetica finita di un calcolatore. In particolare, gli errori di rappresentazione nascono dal fatto che non tutti i numeri possono essere rappresentati esattamente nella memoria di un computer.
Analizzeremo la rappresentazione di numeri interi e reali.

\paragraph{Numeri Interi}
Fissata una base di rappresentazione $b$ e un numero di cifre $N$, un numero intero viene memorizzato tramite una stringa del tipo:
$$ \alpha_0 \alpha_1 \dots \alpha_N $$
dove $\alpha_0 \in \{+,-\}$ è il segno e $\alpha_i \in \{0, 1, \dots, b-1\}$ sono le cifre. A questa stringa corrisponde il valore:
\[ n = 
\begin{cases} 
\sum_{i=1}^{N} \alpha_i b^{N-i} & \text{se } \alpha_0 = + \\
\sum_{i=1}^{N} \alpha_i b^{N-i} - b^N & \text{se } \alpha_0 = - \quad \text{(es. complemento a 2)}
\end{cases}
\]

Ad esempio, con $b=2$ e $N=15$ (16 bit totali, cioè 2 byte), è possibile rappresentare tutti gli interi nell'intervallo $[-32768, 32767]$. Se un intero rimane in questo intervallo, non ci sono errori di rappresentazione.

\subsubsection{Numeri reali}
Un numero "reale" è rappresentato in memoria da una stringa del tipo:
$$ \alpha_0 \alpha_1 \dots \alpha_m \beta_1 \dots \beta_s $$
Fissata una base di rappresentazione $b \in \mathbb{N}$ (pari), le cifre sono così definite:
\begin{itemize}
    \item $\alpha_0 \in \{+,-\}$
    \item $\alpha_i, \beta_j \in \{0, 1, \dots, b-1\}$, per $i=1,\dots,m$ e $j=1,\dots,s$, con $\alpha_1 \neq 0$.
\end{itemize}
Questa stringa rappresenta il numero in notazione scientifica normalizzata:
$$ \tilde{x} = \pm S \cdot b^{e-\nu} $$
dove $S$ è la \textbf{mantissa}:
$$ S = \sum_{i=1}^{m} \alpha_i b^{1-i} = (\alpha_1 . \alpha_2 \dots \alpha_m)_b $$
e $e$ è l'\textbf{esponente}, dato da:
$$ e = \sum_{j=1}^{s} \beta_j b^{s-j} $$
A questo si sottrae lo "shift" $\nu$, una costante intera fissata. Lo shift è scelto in modo da poter rappresentare circa lo stesso numero di esponenti positivi e negativi. Poiché $0 \le e \le b^s - 1$, si sceglie tipicamente $\nu \approx \frac{b^s}{2}$.

Per la mantissa, abbiamo che:
$$ 1 \le S < b $$

\begin{definition}[Numeri di Macchina]
L'insieme dei numeri della forma descritta, assieme allo zero, costituisce l'insieme dei \textbf{numeri di macchina} normalizzati, indicato con $\mathcal{M}$.
\end{definition}

\begin{osservazione}
\begin{enumerate}
    \item L'insieme $\mathcal{M}$ è un insieme finito.
    \item Il più piccolo numero di macchina positivo è $r_1 = 1 \cdot b^{0-\nu} = b^{-\nu}$.
    \item Il più grande numero di macchina positivo è $r_2 = (b - b^{1-m}) \cdot b^{(b^s-1)-\nu} \approx b^{b^s-\nu}$.
\end{enumerate}
\end{osservazione}

Tutti i numeri di macchina sono contenuti nell'intervallo:
$$ \mathcal{I} = [-r_2, -r_1] \cup \{0\} \cup [r_1, r_2] $$
Poiché $\mathcal{M}$ è un insieme discreto mentre $\mathcal{I}$ è denso, è necessario definire una funzione di "arrotondamento", detta \textbf{floating}, che associa a ogni numero reale $x \in \mathcal{I}$ un numero di macchina $\tilde{x} \in \mathcal{M}$.
$$ fl: x \in \mathcal{I} \to \tilde{x} = fl(x) \in \mathcal{M} $$
La quantità $fl(x)-x$ è l'errore di rappresentazione. Per costruzione, valgono le seguenti proprietà:
\begin{itemize}
    \item $fl(0) = 0$
    \item Se $x \in \mathcal{M}$, allora $fl(x) = x$
    \item Per $x>0$, $fl(-x) = -fl(x)$
\end{itemize}

Dato un generico $x \in \mathcal{I}$, $x > 0$, scritto come:
$$ x = (\alpha_1 . \alpha_2 \dots \alpha_m \alpha_{m+1} \dots)_b \cdot b^{e-\nu} $$
esistono due modi principali per implementare $fl(x)$:
\begin{enumerate}
    \item \textbf{Troncamento}: si tagliano le cifre della mantissa dopo la $m$-esima.
    $$ fl(x) = (\alpha_1 . \alpha_2 \dots \alpha_m)_b \cdot b^{e-\nu} $$
    
    \item \textbf{Arrotondamento}: si considera la prima cifra scartata, $\alpha_{m+1}$.
    $$ fl(x) = (\alpha_1 . \alpha_2 \dots \alpha_{m-1} \tilde{\alpha}_m)_b \cdot b^{e-\nu} $$
    con
    \[ \tilde{\alpha}_m = 
    \begin{cases} 
    \alpha_m & \text{se } \alpha_{m+1} < b/2 \\
    \alpha_m + 1 & \text{se } \alpha_{m+1} \ge b/2
    \end{cases}
    \]
\end{enumerate}

\begin{teorema}
Per i numeri $x \in \mathcal{I}$, l'errore relativo di rappresentazione è maggiorato da una costante $u$, detta \textbf{precisione di macchina}.
$$ \epsilon_x = \frac{|fl(x)-x|}{|x|} \le u = 
\begin{cases} 
b^{1-m} & \text{in caso di troncamento} \\
\frac{1}{2}b^{1-m} & \text{in caso di arrotondamento}
\end{cases}
$$
\end{teorema}
\begin{proof}[Dimostrazione (solo per il troncamento)]
$$ \epsilon_x = \frac{|x-fl(x)|}{|x|} = \frac{|(0.0 \dots 0 \alpha_{m+1} \dots)_b \cdot b^{e-\nu}|}{|(\alpha_1 . \alpha_2 \dots)_b \cdot b^{e-\nu}|} = \frac{|(\alpha_{m+1}.\alpha_{m+2}\dots)_b \cdot b^{-m}|}{|(\alpha_1.\alpha_2\dots)_b|} \le \frac{b \cdot b^{-m}}{1} = b^{1-m} $$
\end{proof}

\begin{osservazione}
    Pertanto, concludiamo che la precisione di macchina di un'aritmetica finita è una maggiorazione uniforme dell'errore relativo di rappresentazione.
    \end{osservazione}

\subsubsection{Overflow e Underflow}
Cosa succede se $x \notin \mathcal{I}$?
\begin{itemize}
    \item Se $x > r_2$, si ha una condizione di \textbf{overflow}. La recovery standard è porre $fl(x) = \pm\infty$.
    \item Se $0 < x < r_1$, si ha una condizione di \textbf{underflow}. Esistono due tipi di recovery:
    \begin{enumerate}
        \item \textbf{Store to zero}: si pone $fl(x) = 0$.
        \item \textbf{Gradual underflow}: si permette alla prima cifra della mantissa, $\alpha_1$, di essere zero (numero denormalizzato). Questo estende l'intervallo di rappresentabilità vicino allo zero, ma a discapito della precisione (il Teorema precedente non è più valido).
    \end{enumerate}
\end{itemize}

In conclusione, per la funzione $fl(x)$ vale che:
\[
fl(x) = 
\begin{cases} 
x, & \text{se } x \in \mathcal{M} \\
-fl(-x), & \text{se } x < 0 \\
\text{underflow}, & \text{se } 0 < |x| < r_1 \\
\text{overflow}, & \text{se } |x| > r_2
\end{cases}
\]

\subsubsection{Lo standard IEEE 754}
Lo standard IEEE 754 definisce un formato comune per l'aritmetica in virgola mobile, garantendo che i calcoli producano gli stessi risultati su piattaforme diverse.
Le sue caratteristiche principali sono:
\begin{itemize}
    \item \textbf{Base binaria} ($b=2$).
    \item \textbf{Arrotondamento "round to even"}: in caso di ambiguità (quando la prima cifra scartata è esattamente a metà), si sceglie il numero di macchina la cui ultima cifra della mantissa sia pari (cioè 0). Nonostante questa particolarità, la maggiorazione dell'errore relativo $u = \frac{1}{2}b^{1-m}$ continua a valere.
    \item \textbf{Gradual underflow}: viene implementata la gestione dei numeri denormalizzati.
\end{itemize}
Dato che la base è binaria, la parte intera della mantissa di un numero è sempre nota:
\begin{itemize}
    \item È **1** per i numeri \textbf{normalizzati} (forma $1.f$).
    \item È **0** per i numeri \textbf{denormalizzati} (forma $0.f$).
\end{itemize}
Questo permette di non memorizzare la parte intera, ma solo la parte frazionaria $f$, risparmiando 1 bit.

Lo standard prevede due formati principali:

\paragraph{Singola Precisione (32 bit - 4 byte)}
I 32 bit sono così ripartiti:
\begin{itemize}
    \item \textbf{1 bit} per il segno della mantissa ($\alpha_0$).
    \item \textbf{8 bit} per l'esponente ($s=8$), quindi $0 \le e \le 255$.
    \item \textbf{23 bit} per la parte frazionaria $f$ (la mantissa ha $m=24$ bit).
\end{itemize}
La precisione di macchina in singola precisione è $u = \frac{1}{2} \cdot 2^{1-24} = 2^{-24} \approx 5.96 \times 10^{-8}$.

L'esponente $e$ assume significati speciali:
\begin{itemize}
    \item Se $0 < e < 255$: numero \textbf{normalizzato}, con shift $\nu = 127$.
    \item Se $e=0$ e $f=0$: rappresenta lo \textbf{zero}.
    \item Se $e=0$ e $f \neq 0$: numero \textbf{denormalizzato}, con shift $\nu = 126$.
    \item Se $e=255$ e $f=0$: rappresenta $\pm\infty$ (a seconda del segno).
    \item Se $e=255$ e $f \neq 0$: rappresenta un \textbf{NaN} (Not a Number), generato da forme indeterminate come $\infty-\infty$, $0 \cdot \infty$, $\frac{0}{0}$.
\end{itemize}

\begin{osservazione}[Variazione dello shift e contiguità]
    La variazione dello shift tra numeri normalizzati e denormalizzati si spiega osservando la transizione tra i due insiemi, che lo standard rende il più graduale possibile.
    \begin{itemize}
        \item Il più piccolo numero \textbf{normalizzato} positivo si ha con la mantissa minima ($1.0...0$) e l'esponente minimo per i normalizzati ($e=1$), risultando in:
        $$(1.0\dots0)_2 \times 2^{1-127} = 2^{-126}$$
    
        \item Il più grande numero \textbf{denormalizzato} positivo si ha con la mantissa massima ($0.1...1$) e l'esponente fisso per i denormalizzati, che è uguale a quello dei più piccoli normalizzati:
        $$(0.1\dots1)_2 \times 2^{-126} = (1 - 2^{-23}) \times 2^{-126}$$
    \end{itemize}
    Pertanto, i due numeri sono "contigui": il più grande denormalizzato è immediatamente precedente al più piccolo normalizzato, garantendo una transizione fluida verso lo zero (da cui il nome \emph{gradual underflow}).
    \end{osservazione}

\paragraph{Doppia Precisione (64 bit - 8 byte)}
I 64 bit sono così ripartiti:
\begin{itemize}
    \item \textbf{1 bit} per il segno.
    \item \textbf{11 bit} per l'esponente ($s=11$), quindi $0 \le e \le 2047$.
    \item \textbf{52 bit} per la parte frazionaria $f$ ($m=53$ bit).
\end{itemize}
La precisione di macchina è $u = \frac{1}{2} \cdot 2^{1-53} = 2^{-53} \approx 1.11 \times 10^{-16}$, il che significa lavorare con circa 16 cifre decimali significative.

Le regole per l'esponente sono analoghe alla singola precisione:
\begin{itemize}
    \item Se $0 < e < 2047$: numero \textbf{normalizzato}, con shift $\nu = 1023$.
    \item Se $e=0$ e $f=0$: rappresenta lo \textbf{zero}.
    \item Se $e=0$ e $f \neq 0$: numero \textbf{denormalizzato}, con shift $\nu = 1022$.
    \item Se $e=2047$ e $f=0$: rappresenta $\pm\infty$.
    \item Se $e=2047$ e $f \neq 0$: rappresenta un \textbf{NaN}.
\end{itemize}

\subsubsection{Aritmetica Finita}
Le operazioni algebriche elementari $(+, -, *, /)$ in aritmetica finita sono definite come segue, per due numeri reali $x, y \in \mathbb{R}$:
$$ x \oplus y = fl(fl(x) + fl(y)) $$
Questo implica che le comuni proprietà algebriche (associativa, distributiva) in genere non valgono più.

\begin{osservazione}[Esempi in Matlab]
I seguenti comandi Matlab mostrano la perdita della proprietà associativa:
\begin{lstlisting}
    >> r2 = realmax; % Il più grande numero in doppia precisione
    >> [(r2 - r2) + 1, r2 - (r2 + 1)]
    ans =
         1     0
    \end{lstlisting}
E la gestione di operazioni con Infinito:
\begin{lstlisting}
>> [(r2 - r2) * 2, r2 * 2 - r2 * 2]
ans =
     0   NaN
\end{lstlisting}
\end{osservazione}


\subsubsection{Conversione tra tipi diversi}
La conversione tra tipi di dati numerici, come tra diverse precisioni di numeri reali o tra reali e interi, è un'operazione delicata che può introdurre errori significativi.

\paragraph{Conversione tra Reali (doppia $\leftrightarrow$ singola precisione)}
Consideriamo una variabile `x` in doppia precisione e una `y` in singola precisione.
\begin{itemize}
    \item \textbf{Da doppia a singola}: Se si assegna un valore in doppia precisione (es. $\pi$) a una variabile in singola, il valore verrà memorizzato con l'accuratezza massima consentita dalla singola precisione, perdendo le cifre eccedenti.
    \begin{verbatim}
    x = pi; % Doppia precisione
    y = single(x); % y contiene pi con accuratezza ridotta
    \end{verbatim}
    
    \item \textbf{Da singola a doppia}: Se si esegue l'operazione inversa, la precisione persa non viene recuperata. La nuova variabile a doppia precisione manterrà l'accuratezza limitata del dato di partenza.
    \begin{verbatim}
    y = single(pi);
    x = double(y); % x ha la stessa (bassa) accuratezza di y
    \end{verbatim}
\end{itemize}

\paragraph{Conversione Reale $\leftrightarrow$ Intero}
\begin{itemize}
    \item \textbf{Da Intero a Reale}: Questa conversione è in genere innocua. Poiché l'insieme dei numeri reali rappresentabili è molto più ampio e denso, un intero può quasi sempre essere convertito in un reale. Si può avere una perdita di precisione solo se l'intero ha più cifre significative di quante la mantissa del tipo reale possa contenere.
    
    \item \textbf{Da Reale a Intero}: Questa è un'operazione \textbf{molto pericolosa}. L'intervallo di rappresentabilità degli interi (es. $[-32768, 32767]$ per interi a 2 byte) è estremamente più ristretto di quello dei numeri reali. Se il numero reale da convertire è al di fuori di questo intervallo, si verifica un errore di overflow.
\end{itemize}

\subsection{Condizionamento di un problema}
Supponiamo di voler calcolare la soluzione di un problema che, per semplicità, formalizziamo come:
\begin{equation}
    y = f(x)
\end{equation}
dove $f: \mathbb{R} \to \mathbb{R}$ è una funzione sufficientemente regolare, $x$ è il dato di ingresso e $y$ è il risultato atteso.

In un contesto reale, invece di lavorare con i dati e la funzione esatti, spesso si ha a che fare con un dato perturbato $\tilde{x}$ e, a causa dell'aritmetica finita del calcolatore, si utilizza una funzione perturbata $\tilde{f}$. Questo porta a un risultato anch'esso perturbato:
\begin{equation}
    \tilde{y} = \tilde{f}(\tilde{x})
\end{equation}

\begin{osservazione}
Analizzare la differenza completa tra il risultato esatto e quello calcolato (cioè tra l'equazione (1) e la (2)) è in genere molto complesso. Ci limiteremo a un'analisi più semplice, studiando come le perturbazioni sui soli dati di ingresso influenzino il risultato, assumendo di utilizzare un'aritmetica esatta:
\begin{equation}
    \tilde{y} = f(\tilde{x})
\end{equation}
Lo studio della differenza tra il risultato dell'equazione (3) e quello dell'equazione (1) costituisce l'analisi del \textbf{condizionamento del problema}.
\end{osservazione}

Per $y \neq 0$, l'analisi è più efficace se condotta in termini di errori relativi. Poniamo:
\[
\begin{cases}
    \tilde{x} = x(1 + \epsilon_x) \\
    \tilde{y} = y(1 + \epsilon_y)
\end{cases}
\]
dove $\epsilon_x$ e $\epsilon_y$ sono gli errori relativi sul dato di ingresso e sul risultato, rispettivamente. Vogliamo stabilire come $\epsilon_x$ si propaga su $\epsilon_y$, supponendo $|\epsilon_x| \ll 1$.

Sostituendo le definizioni nella (3) e usando lo sviluppo di Taylor, otteniamo:
$$ y(1 + \epsilon_y) = f(x(1+\epsilon_x)) \approx f(x) + f'(x) \cdot (x \epsilon_x) $$
Dato che $y = f(x)$, si ha:
$$ y + y \epsilon_y \approx y + f'(x) x \epsilon_x \implies y \epsilon_y \approx f'(x) x \epsilon_x $$
Da cui si ricava la relazione tra gli errori relativi:
$$ \epsilon_y \approx \frac{f'(x)x}{y} \epsilon_x $$
Questo ci porta a definire il numero di condizione.

\begin{definition}[Numero di Condizione]
Il fattore di amplificazione dell'errore
$$ K = \left| \frac{f'(x)x}{y} \right| $$
è detto \textbf{numero di condizione} del problema. Vale la relazione:
$$ |\epsilon_y| \approx K \cdot |\epsilon_x| $$
\end{definition}

Un problema si dice:
\begin{itemize}
    \item \textbf{ben condizionato}, se $K \approx 1$;
    \item \textbf{mal condizionato}, se $K \gg 1$ (molto maggiore).
\end{itemize}

\begin{osservazione}
Se si lavora in un'aritmetica finita con precisione di macchina $u$, l'errore relativo sul dato di ingresso sarà almeno $|\epsilon_x| \approx u$. Se il problema ha un numero di condizione $K \approx u^{-1}$, l'errore relativo sul risultato sarà $|\epsilon_y| \approx 1$. Questo significa una perdita totale di cifre significative, rendendo il risultato inattendibile.
\end{osservazione}

\subsubsection*{Condizionamento delle Operazioni Algebriche Elementari}

\paragraph{Moltiplicazione}
Per il problema $y = x_1 x_2$, perturbando gli input si ha:
$$ y(1+\epsilon_y) = x_1(1+\epsilon_1) x_2(1+\epsilon_2) = x_1 x_2 (1 + \epsilon_1 + \epsilon_2 + \epsilon_1 \epsilon_2) $$
Trascurando i termini di ordine superiore, $y(1+\epsilon_y) \approx y(1+\epsilon_1+\epsilon_2)$, da cui:
$$ |\epsilon_y| \approx |\epsilon_1 + \epsilon_2| \le |\epsilon_1| + |\epsilon_2| \le 2 \max\{|\epsilon_1|, |\epsilon_2|\} $$
La moltiplicazione è un'operazione \textbf{ben condizionata}, con $K=2$.

\paragraph{Divisione}
Per il problema $y = x_1 / x_2$, si ha:
$$ y(1+\epsilon_y) = \frac{x_1(1+\epsilon_1)}{x_2(1+\epsilon_2)} = \frac{x_1}{x_2}(1+\epsilon_1)(1+\epsilon_2)^{-1} $$
Usando lo sviluppo $(1+\epsilon)^{-1} \approx 1 - \epsilon$, otteniamo:
$$ y(1+\epsilon_y) \approx y(1+\epsilon_1)(1-\epsilon_2) \approx y(1+\epsilon_1 - \epsilon_2) $$
$$ |\epsilon_y| \approx |\epsilon_1 - \epsilon_2| \le 2 \max\{|\epsilon_1|, |\epsilon_2|\} $$
Anche la divisione è un'operazione \textbf{ben condizionata}, con $K=2$.

\paragraph{Somma Algebrica}
Per il problema $y = x_1 + x_2$, la perturbazione porta a:
$$ y(1+\epsilon_y) = x_1(1+\epsilon_1) + x_2(1+\epsilon_2) = x_1+x_2 + x_1\epsilon_1 + x_2\epsilon_2 $$
$$ y\epsilon_y = x_1\epsilon_1 + x_2\epsilon_2 \implies \epsilon_y = \frac{x_1}{x_1+x_2}\epsilon_1 + \frac{x_2}{x_1+x_2}\epsilon_2 $$
Il numero di condizione è:
$$ K = \frac{|x_1| + |x_2|}{|x_1+x_2|} $$
Si distinguono due casi:
\begin{itemize}
    \item \textbf{Addendi concordi ($x_1 x_2 > 0$)}: In questo caso $|x_1+x_2| = |x_1|+|x_2|$, quindi $K=1$. La somma di numeri concordi è \textbf{ben condizionata}.
    \item \textbf{Addendi discordi ($x_1 x_2 < 0$)}: Se $x_1 \approx -x_2$, il denominatore $|x_1+x_2|$ diventa molto piccolo, e $K$ può diventare arbitrariamente grande. La somma di numeri quasi opposti è un'operazione \textbf{mal condizionata}. Questo porta al fenomeno della \textbf{cancellazione numerica}, in cui il risultato può essere molto inaccurato anche se gli addendi sono noti con grande precisione.
\end{itemize}

\subsubsection*{Esempio di Cancellazione Numerica}

Si vuole approssimare la derivata di $f(x) = x^{10}$ in $x=1$ (valore esatto $f'(1)=10$) usando la formula:
$$ \frac{(1+\text{eps})^{10} - 1}{\text{eps}} $$
Al diminuire di \texttt{eps}, l'errore di troncamento si riduce, ma l'errore di round-off dovuto alla cancellazione numerica (sottrazione tra due numeri quasi uguali: $(1+\text{eps})^{10}$ e $1$) aumenta, fino a dominare e distruggere il risultato.

\begin{table}[h!]
\centering
\caption{Approssimazione della derivata di $f(x)=x^{10}$ in $x=1$}
\begin{tabular}{ll}
\toprule
\textbf{eps} & \textbf{Valore Calcolato: $((1+\text{eps})^{10}-1)/\text{eps}$} \\
\midrule
1.00e-01 & 15.937424601000023 \\
1.00e-02 & 10.462212541120453 \\
1.00e-03 & 10.045120210251168 \\
1.00e-04 & 10.004501200209237 \\
1.00e-05 & 10.000450012070949 \\
1.00e-06 & 10.000044999403102 \\
1.00e-07 & 10.000004506682814 \\
\midrule
\multicolumn{2}{c}{\textit{--- Inizia la cancellazione numerica ---}} \\
\midrule
1.00e-08 & 10.000000383314500 \\
1.00e-09 & 10.000000827403710 \\
1.00e-10 & 10.000000827403710 \\
1.00e-11 & 10.000000827403708 \\
1.00e-12 & 10.000889005823410 \\
1.00e-13 & 9.992007221626409 \\
1.00e-14 & 9.992007221626409 \\
1.00e-15 & 11.102230246251565 \\
1.00e-16 & 0.000000000000000 \\
\bottomrule
\end{tabular}
\end{table}

\section{Matlab (Matrix Laboratory)}
Matlab è un linguaggio interpretato orientato alla gestione efficiente delle matrici. Tutte le variabili sono definite in un'area di lavoro (workspace) e non è necessario dichiararne il tipo o la dimensione in anticipo.

\subsection{Gestione delle Variabili e della Workspace}
Il dimensionamento delle variabili è automatico e si adatta alle operazioni. Ad esempio, eseguendo `A(6,7) = 3.14`, Matlab crea una matrice `A` di dimensioni $6 \times 7$. Gli elementi non specificati vengono inizializzati a zero. La rappresentazione interna dei numeri è conforme allo standard IEEE 754 in doppia precisione.

Per ispezionare la workspace si usano i comandi:
\begin{itemize}
    \item \texttt{who}: mostra i nomi delle variabili presenti.
    \item \texttt{whos}: mostra una tabella dettagliata con nome, dimensione, tipo e memoria occupata.
\end{itemize}

È buona norma, per ragioni di efficienza, pre-allocare lo spazio per le matrici se le loro dimensioni finali sono note. Ad esempio:
\begin{lstlisting}
% Inizializza una matrice 200x300 con tutti zeri
A = zeros(200, 300);
\end{lstlisting}
Per sopprimere la visualizzazione del risultato di un'operazione, si termina la riga con un punto e virgola (\texttt{;}).

Per salvare o caricare la workspace si usano i comandi:
\begin{itemize}
    \item \texttt{save <nomefile>}: salva tutte le variabili nel file \texttt{nomefile.mat}.
    \item \texttt{load <nomefile>}: carica le variabili dal file specificato.
\end{itemize}

\paragraph{Formato di Visualizzazione}
Il comando \texttt{format} controlla come i valori numerici vengono visualizzati nella finestra di comando. È importante sottolineare che questo comando modifica solo la visualizzazione e non la precisione con cui i numeri sono memorizzati (che rimane in doppia precisione).

Alcune delle opzioni più comuni sono:
\begin{itemize}
    \item \texttt{format short}: formato a punto fisso con 5 cifre (default).
    \item \texttt{format long}: formato a punto fisso con 15 cifre.
    \item \texttt{format short e}: notazione scientifica con 5 cifre.
    \item \texttt{format long e}: notazione scientifica con 15 cifre.
\end{itemize}

\subsection{Creazione e Manipolazione di Matrici}
Uno dei comandi più utili in Matlab è \lstinline|help|. Questo comando permette di visualizzare la documentazione di una qualsiasi funzione direttamente nella finestra di comando. Ad esempio, per ottenere informazioni sulla funzione \lstinline|zeros|, si digita \lstinline|help zeros|.
\paragraph{Funzioni di base}
\begin{itemize}
    \item \texttt{zeros(m, n)}: crea una matrice $m \times n$ di zeri.
    \item \texttt{eye(n)}: crea una matrice identità $n \times n$.
    \item \texttt{rand(m, n)}: crea una matrice $m \times n$ con elementi casuali da una distribuzione uniformeme.
\end{itemize}

\paragraph{Operatore Colon (:)}
L'operatore \texttt{:} è estremamente versatile per creare vettori e selezionare sottomatrici.
\begin{lstlisting}
v1 = 1:5;       % Crea il vettore [1 2 3 4 5]
v2 = 5:-1:1;    % Crea il vettore [5 4 3 2 1]
v3 = 2:2:20;    % Crea il vettore dei numeri pari da 2 a 20
\end{lstlisting}

\paragraph{Indicizzazione e Concatenazione}
Si possono estrarre righe, colonne o sottomatrici e concatenare matrici esistenti.
\begin{lstlisting}
A = rand(3);
B = rand(3,1);
C = rand(2,4);

% Concatenazione orizzontale e verticale
D = [A B]; % Concatena A e B orizzontalmente (devono avere stesso numero di righe)
E = [A; rand(1,3)]; % Concatena A e una nuova riga verticalmente

% Estrazione
seconda_riga = D(2, :);
terza_colonna = D(:, 3);
sottomatrice = D(1:2, 1:3); % Estrae le prime due righe e le prime tre colonne
\end{lstlisting}

\subsection{Operatori ed Espressioni}
\begin{itemize}
    \item \textbf{Operatori Aritmetici}: \verb|+|, \verb|-|, \verb|*|, \verb|/|, \verb|\|
    \item \textbf{Operazioni con Scalari}: Le operazioni tra una matrice e uno scalare sono applicate a ogni elemento. Es: \texttt{C = A + 1;}, \texttt{C = 2 * A;}.
    \item \textbf{Operatori Element-wise}: Se preceduti da un punto (\texttt{.}), gli operatori agiscono elemento per elemento. Richiedono che le matrici abbiano le stesse dimensioni.
    \begin{itemize}
        \item \texttt{C = A .* B} $\implies C(i,j) = A(i,j) * B(i,j)$
        \item \texttt{C = A ./ B} $\implies C(i,j) = A(i,j) / B(i,j)$
    \end{itemize}
    \item \textbf{Divisione Matriciale}:
    \begin{itemize}
        \item \verb|C = A \ B| (backslash) risolve il sistema lineare $AX = B$, equivalente a $A^{-1}B$.
        \item \texttt{C = A / B} (slash) è equivalente a $AB^{-1}$.
    \end{itemize}
\end{itemize}

\subsection{Funzioni Utili}
Le funzioni in Matlab possono essere raggruppate in:
\begin{itemize}
    \item \textbf{Funzioni orientate a scalari}: operano su ogni elemento di una matrice. Es: \texttt{sin(A)}, \texttt{cos(A)}, \texttt{exp(A)}, \texttt{log(A)}, \texttt{sqrt(A)}. Anche le funzioni di arrotondamento come \texttt{round}, \texttt{floor}, \texttt{ceil}, \texttt{fix} rientrano in questa categoria.
    \item \textbf{Funzioni orientate a vettori}: se applicate a una matrice, operano colonna per colonna restituendo un vettore riga. Es: \texttt{max(A)}, \texttt{min(A)}, \texttt{mean(A)}, \texttt{sum(A)}, \texttt{sort(A)}.
    \item \textbf{Funzioni orientate a matrici}: operano sull'intera matrice. Es: \texttt{size(A)}, \texttt{det(A)}, \texttt{inv(A)}, \texttt{diag(A)}.
\end{itemize}

\subsection{Grafica 2D}
Matlab offre potenti strumenti per la visualizzazione grafica.
\begin{itemize}
    \item \texttt{plot(x, y)}: disegna la spezzata che congiunge i punti definiti dai vettori \texttt{x} e \texttt{y}.
    \item \texttt{semilogx}, \texttt{semilogy}, \texttt{loglog}: disegnano grafici con scale logaritmiche sugli assi.
    \item \texttt{xlabel('testo')}, \texttt{ylabel('testo')}, \texttt{title('titolo')}: aggiungono etichette agli assi e un titolo al grafico.
    \item \texttt{legend('curva1', 'curva2')}: aggiunge una legenda.
    \item \texttt{axis([xmin xmax ymin ymax])}: imposta i limiti degli assi.
    \item \texttt{figure}: apre una nuova finestra grafica.
\end{itemize}

\subsection{Controllo del Flusso e m-files}
Matlab fornisce costrutti per il controllo del flusso di esecuzione (selezioni e iterazioni) e permette di salvare sequenze di comandi in file con estensione \texttt{.m}, detti \textbf{m-files}.

\subsubsection{Selezioni}
Il costrutto di selezione principale è \texttt{if-elseif-else}.
\begin{lstlisting}[frame=none, numbers=none]
if <espressione_booleana_1>
    % blocco di istruzioni 1
elseif <espressione_booleana_2>
    % blocco di istruzioni 2
else
    % blocco di istruzioni 3
end
\end{lstlisting}
Le espressioni booleane si ottengono combinando operatori relazionali (\verb|>|, \verb|<|, \verb|>=|, \verb|<=|, \verb|==|, \verb|~=|) e operatori booleani (\verb|&&| per AND, \verb#||# per OR, \verb|~| per NOT). È buona norma usare le parentesi \verb|()| per definire chiaramente l'ordine di valutazione.

\subsubsection{Iterazioni (Cicli)}
I cicli principali sono \texttt{for} e \texttt{while}.
\paragraph{Il ciclo \texttt{for}} Esegue un blocco di istruzioni per ogni elemento di un vettore.
\begin{lstlisting}
% Esempio: calcola il quadrato dei numeri pari da 10 a 2
a = zeros(10,1);
for i = 10:-2:2
    a(i) = i^2;
end
\end{lstlisting}
Il comando \texttt{break} permette di uscire immediatamente dal ciclo che lo contiene.

\paragraph{Il ciclo \texttt{while}} Esegue un blocco di istruzioni finché una condizione booleana rimane vera.
\begin{lstlisting}[frame=none, numbers=none]
while <condizione_booleana>
    % blocco di istruzioni
    % (eventuale `break` per uscita forzata)
end
\end{lstlisting}

\subsubsection{m-files: Script e Function}
Esistono due tipi di m-files.
\begin{itemize}
    \item \textbf{Script-file}: È una semplice sequenza di comandi. Quando eseguito, opera direttamente sulle variabili presenti nella workspace (le sue variabili sono \textbf{globali}).
    \item \textbf{Function-file}: È un blocco di codice più strutturato che comunica con l'esterno solo tramite parametri di input e output. Tutte le variabili definite al suo interno sono \textbf{locali} e vengono distrutte al termine dell'esecuzione.
\end{itemize}

\paragraph{Anatomia di una Function}
Una function è definita dalla seguente intestazione:
\begin{lstlisting}[frame=none, numbers=none]
function [out1, out2, ...] = nomeFunzione(in1, in2, ...)
\end{lstlisting}
All'interno di una function si possono usare:
\begin{itemize}
    \item \texttt{nargin}: restituisce il numero di argomenti di input forniti alla chiamata.
    \item \texttt{nargout}: restituisce il numero di argomenti di output richiesti alla chiamata.
    \item \texttt{\%}: introduce un commento (il resto della riga viene ignorato).
    \item \texttt{...} (tre punti): permettono di spezzare una singola istruzione su più righe.
    \item \texttt{return}: termina l'esecuzione della function.
\end{itemize}

\paragraph{Help in linea}
Le prime righe di commento consecutive dopo l'intestazione di una function costituiscono il suo \textbf{help in linea}, che viene visualizzato quando si digita \texttt{help nomeFunzione}.

\paragraph{Esempio di Function}
Questo esempio definisce una function che calcola media e varianza di un vettore di dati, gestendo il numero di input e output.

\begin{lstlisting}
function [media, varianza] = esempio(dati)
% [media, varianza] = esempio(dati)
% Calcola la media e la varianza di un insieme di dati.
%
% Input:
%   dati - vettore con i dati di ingresso;
% Output:
%   media    - media dei dati;
%   varianza - varianza dei dati.

% Controllo degli input
if nargin < 1
    error('Dati di ingresso insufficienti');
else
    if isempty(dati), error('Il vettore di dati è vuoto'), end
end

media = mean(dati);

% Calcola la varianza solo se richiesta in output
if nargout > 1
    varianza = var(dati);
end

return
\end{lstlisting}

Digitando \texttt{help esempio} nella console di Matlab, si otterrà il testo scritto nelle prime righe di commento della funzione.


\section{Radici di una funzione}

Data una funzione $f: [a,b] \subseteq \mathbb{R} \to \mathbb{R}$, vogliamo determinare un valore $x^* \in [a,b]$ tale che:
\[
f(x^*) = 0
\]
Diremo che $x^*$ è una \textbf{radice} (o uno \textbf{zero}) della funzione $f(x)$.

In generale, una radice $x^*$ può:
\begin{itemize}
    \item Esistere ed essere unica.
    \item Esistere ma non essere unica (es. $f(x) = \sin(x)$).
    \item Non esistere (es. $f(x) = e^x$).
\end{itemize}

I metodi che esamineremo, se la radice esiste, forniranno un'approssimazione di una di esse.  
Una caratteristica comune a tutti questi metodi è di essere di tipo \textbf{iterativo}:  
a partire da un'approssimazione iniziale $x_0$, viene prodotta una successione di approssimazioni $\{x_n\}_{n \ge 0}$ che, se il metodo è convergente, tende alla radice $x^*$:
\[
\lim_{n \to \infty} x_n = x^*
\]

\subsection{Il metodo di bisezione}
Il primo metodo che analizziamo è il metodo di bisezione.  
Si basa su due assunzioni fondamentali:
\begin{enumerate}
    \item La funzione $f(x)$ è \textbf{continua} in un intervallo chiuso e limitato $[a,b]$.
    \item Agli estremi dell'intervallo, la funzione assume valori di segno opposto, ovvero $f(a) \cdot f(b) < 0$.
\end{enumerate}

\begin{figure}[ht!]
    \centering
    \begin{tikzpicture}[scale=1.5]
        % Assi cartesiani
        \draw[->] (-0.5,0) -- (4,0) node[below] {$x$};
        \draw[->] (0,-1.5) -- (0,1.5) node[left] {$f(x)$};
        
        % Disegna la funzione
        \draw[thick, color=green!50!black] plot[smooth, domain=0.5:3.5] (\x, {cos(\x*60)+0.1*\x-0.5});
        
        % Punti e etichette
        \draw (1,0.05) -- (1,-0.05) node[below] {$a$};
        \draw (3,0.05) -- (3,-0.05) node[below] {$b$};
        \node[below, color=red] at (2.05, -0.05) {$x^*$};
        \fill[red] (2.05,0) circle (1.5pt);
        
        % Linee tratteggiate e valori
        \draw[dashed] (1,0) -- (1, {cos(60)+0.1*1-0.5});
        \draw[dashed] (0, {cos(60)+0.1*1-0.5}) -- (1, {cos(60)+0.1*1-0.5});
        \node[left] at (0, {cos(60)+0.1*1-0.5}) {$f(a) > 0$};
        
        \draw[dashed] (3,0) -- (3, {cos(180)+0.1*3-0.5});
        \draw[dashed] (0, {cos(180)+0.1*3-0.5}) -- (3, {cos(180)+0.1*3-0.5});
        \node[left] at (0, {cos(180)+0.1*3-0.5}) {$f(b) < 0$};
    \end{tikzpicture}
    \caption{Illustrazione del teorema degli zeri.}
    \label{fig:teorema_zeri}
\end{figure}

In base al \textbf{teorema degli zeri}, queste ipotesi garantiscono l'esistenza di almeno una radice $x^*$ nell'intervallo $(a,b)$.

Non sapendo dove si trovi $x^*$ all'interno di $[a,b]$, la migliore stima iniziale che possiamo fare è il punto medio dell'intervallo:
\[
x_1 = \frac{a+b}{2}
\]
A questo punto, si possono verificare tre casi:
\begin{enumerate}
    \item $f(x_1) = 0$: abbiamo trovato la radice.
    \item $f(a) \cdot f(x_1) < 0$: la radice si trova nell'intervallo $[a, x_1]$.
    \item $f(x_1) \cdot f(b) < 0$: la radice si trova nell'intervallo $[x_1, b]$.
\end{enumerate}
Ad ogni iterazione, l'ampiezza dell'intervallo di confidenza viene dimezzata.

\subsection{Implementazione e criteri di arresto}
Un'implementazione ``naive''\footnote{Per "implementazione naive" si intende una versione semplice e diretta dell'algoritmo che ignora problemi pratici di efficienza o stabilità numerica.} del metodo potrebbe essere:

\begin{lstlisting}
fa = f(a);
fb = f(b);
while true
    x1 = (a+b)/2;
    f1 = f(x1);
    if f1 == 0
        break;
    elseif fa*f1 < 0
        b = x1;
        fb = f1;
    else
        a = x1;
        fa = f1;
    end
end
\end{lstlisting}

Questo criterio di arresto non è robusto: a causa dell'aritmetica finita, la condizione `f1 == 0` potrebbe non verificarsi mai, anche se $x_1$ è molto vicino alla radice, portando a un ciclo infinito.

\paragraph{Esempio: Errore di valutazione}
Consideriamo il polinomio $p(x)=(x-1.1)^{20}(x-\pi)$, che ha una radice in $x=\pi$.  
Valutandolo in Matlab:
\begin{lstlisting}
p_coeffs = poly([1.1*ones(1,20), pi]);
polyval(p_coeffs, pi)
% ans = -5.5213e-05
\end{lstlisting}
Il risultato non è zero, e l'algoritmo naive andrebbe in loop.

\paragraph{Criterio basato sul numero di iterazioni}
Possiamo calcolare a priori il numero massimo di iterazioni necessarie per raggiungere una data tolleranza.  
Se $x_i$ è l'approssimazione al passo $i$, l'errore assoluto è maggiorato dalla semi-ampiezza dell'intervallo:
\[
|x^* - x_i| \le \frac{b_i - a_i}{2} = \frac{b-a}{2^i}
\]
Per garantire un'accuratezza `tol`, imponiamo:
\[
\frac{b-a}{2^i} \le \text{tol}
\quad \Rightarrow \quad
i \ge \log_2\!\left(\frac{b-a}{\text{tol}}\right)
\]
Il numero massimo di iterazioni sarà quindi:
\[
i_{\max} = \left\lceil \log_2\!\left(\frac{b-a}{\text{tol}}\right) \right\rceil
\]
\footnote{Questa formula calcola il numero massimo di iterazioni necessarie per garantire un errore inferiore a `tol`. L'operatore $\lceil \cdot \rceil$ (ceiling) arrotonda all'intero superiore.}

\begin{lstlisting}
fa = f(a);
fb = f(b);
imax = ceil(log2(b-a) - log2(tol));

for i = 1:imax
    x1 = (a+b)/2;
    f1 = f(x1);
    
    if f1 == 0
        break;
    elseif fa*f1 < 0
        b = x1;
        fb = f1;
    else
        a = x1;
        fa = f1;
    end
end
\end{lstlisting}

\paragraph{Criterio basato sul residuo (controllo efficace)}
Un criterio di arresto più efficiente si basa sulla "piccolezza" del valore della funzione, $|f(x)|$.  
Lo sviluppo di Taylor di $f(x)$ attorno alla radice $x^*$ è:
\[
f(x) = f(x^*) + f'(x^*)(x-x^*) + O(|x-x^*|^2)
\]
Dato che $f(x^*)=0$, otteniamo:
\[
|x-x^*| \approx \frac{|f(x)|}{|f'(x^*)|}
\]
e imponendo $|x-x^*| \le \text{tol}$, si ha:
\[
|f(x)| \le \text{tol} \cdot |f'(x^*)|
\]
Poiché $f'(x^*)$ non è noto, lo approssimiamo come:
\[
f'(x^*) \approx \frac{f(b_i)-f(a_i)}{b_i - a_i}
\]

\paragraph{Algoritmo di bisezione ottimizzato}
\begin{lstlisting}
% Versione ottimizzata del metodo di bisezione
fa = f(a);
fb = f(b);
imax = ceil(log2(b-a) - log2(tol));

for i = 1:imax
    x1 = (a+b)/2;
    f1 = f(x1);
    
    % Stima della derivata e criterio di arresto sul residuo
    df = abs(fb-fa)/(b-a);
    if abs(f1) <= tol*df
        break;
    end
    
    % Aggiornamento dell'intervallo
    if fa*f1 < 0
        b = x1; fb = f1;
    else
        a = x1; fa = f1;
    end
end
% x1 contiene la radice approssimata
\end{lstlisting}


\subsection{Ordine di Convergenza}
Abbiamo esaminato il metodo di bisezione, definito sotto le ipotesi $f \in C([a,b])$ e $f(a)f(b)<0$. L'errore $e_i = x^* - x_i$ al passo $i$ soddisfa:
$$ |e_i| \le \epsilon_i = \frac{b-a}{2^i} $$
con $\frac{\epsilon_{i+1}}{\epsilon_i} = \frac{1}{2}$, il che implica $\lim_{i \to \infty} \frac{\epsilon_{i+1}}{\epsilon_i} = \frac{1}{2}$.

\begin{definition}[Ordine di Convergenza]
Un generico metodo iterativo si dice \textbf{convergente} se $\lim_{i \to \infty} e_i = 0$.
Si dice che ha \textbf{ordine di convergenza $p$} se $p$ è il più grande numero reale positivo tale che esista una costante $C > 0$ (costante asintotica dell'errore) per cui:
$$ \lim_{i \to \infty} \frac{|e_{i+1}|}{|e_i|^p} = C < \infty $$
\end{definition}

Per $i$ sufficientemente grande, vale $|e_{i+1}| \approx C |e_i|^p$.
\begin{itemize}
    \item Affinché ci sia convergenza, deve aversi $p \ge 1$.
    \item Se $p=1$, la convergenza è \textbf{lineare}. In questo caso, per $i$ grande, abbiamo:
          \begin{align*}
              |e_{i+1}| &\approx C |e_i| \\
              |e_{i+2}| &\approx C |e_{i+1}| \approx C^2 |e_i| \\
              &\vdots \\
              |e_{i+k}| &\approx C^k |e_i| 
          \end{align*}
          Questa successione tende a zero per $k \to \infty$ se e solo se $C < 1$. Pertanto, un metodo di ordine 1 è convergente se e solo se la sua costante asintotica dell'errore è minore di 1.
    \item Se $p=2$, la convergenza è \textbf{quadratica}.
    
\end{itemize}

\begin{osservazione}
Il metodo di bisezione, se applicabile, è sempre convergente, ha ordine $p=1$ e costante asintotica $C=1/2$.
\end{osservazione}

Un ordine $p$ più elevato implica una convergenza più rapida. Confrontiamo due metodi con $C=0.1$ e $|e_0|=0.1$:

\begin{table}[ht!]
\centering
\caption{Confronto tra convergenza lineare e quadratica ($|e_{i+1}| \approx C|e_i|^p$)}
\begin{tabular}{ccc}
\toprule
$i$ & Errore ($p=1$) & Errore ($p=2$) \\
\midrule
0 & $10^{-1}$ & $10^{-1}$ \\
1 & $10^{-2}$ & $10^{-3}$ \\
2 & $10^{-3}$ & $10^{-7}$ \\
3 & $10^{-4}$ & $10^{-15}$ \\
\bottomrule
\end{tabular}
\end{table}
È evidente che è bene ricercare metodi di ordine più elevato.

\paragraph{Condizionamento del Problema}
Vogliamo determinare $x^*$ tale che $f(x^*) = 0$. Se invece abbiamo una soluzione perturbata $\tilde{x}$ tale che $f(\tilde{x}) \neq 0$, studiamo come la perturbazione sul valore $f(\tilde{x})$ influenza l'errore $|\tilde{x} - x^*|$.
Usando Taylor: $f(\tilde{x}) \approx f'(x^*)(\tilde{x} - x^*)$, da cui:
$$ |\tilde{x} - x^*| \approx \frac{|f(\tilde{x})|}{|f'(x^*)|} $$
Il fattore $K = \frac{1}{|f'(x^*)|}$ è il \textbf{numero di condizione} del problema.
\begin{itemize}
    \item Se $|f'(x^*)| \ >> 1$ (non vicino a zero), il problema è \textbf{ben condizionato}.
    \item Se $|f'(x^*)| \approx 0$, il problema è \textbf{mal condizionato}.
\end{itemize}

\begin{definition}[Molteplicità di una radice]
Una radice $x^*$ ha \textbf{molteplicità $m \ge 1$} se $f(x^*) = f'(x^*) = \dots = f^{(m-1)}(x^*) = 0$ e $f^{(m)}(x^*) \neq 0$.
Se $m=1$, la radice è \textbf{semplice}. Se $m>1$, la radice è \textbf{multipla}.
\end{definition}

\begin{osservazione}
L'approssimazione di una radice multipla dà origine ad un problema \textbf{sempre mal condizionato}.
\end{osservazione}

\subsection{Metodo di Newton}
Il metodo di Newton è un metodo iterativo che, a partire da un'approssimazione $x_0$, costruisce la successione $\{x_i\}$ utilizzando l'interpretazione geometrica della derivata.

L'equazione della retta tangente al grafico di $f(x)$ nel punto $(x_0, f(x_0))$ è data da $y - f(x_0) = f'(x_0)(x - x_0)$. La nuova approssimazione $x_1$ si trova intersecando questa retta con l'asse delle $x$ (retta di equazione $y=0$). Dobbiamo quindi risolvere il sistema:
\[
\begin{cases} 
y - f(x_0) = f'(x_0)(x - x_0) & \leftarrow \text{retta tangente} \\
y = 0 & \leftarrow \text{asse x}
\end{cases}
\]
Sostituendo $y=0$ nella prima equazione e chiamando la soluzione $x_1$, otteniamo:
$$ -f(x_0) = f'(x_0)(x_1 - x_0) $$
Da cui si ricava la nuova approssimazione:
$$ x_1 = x_0 - \frac{f(x_0)}{f'(x_0)} $$
In generale, iterando questo procedimento, si ottiene la formula del metodo di Newton:
$$ x_{i+1} = x_i - \frac{f(x_i)}{f'(x_i)}, \quad i=0, 1, \dots $$

% --- INSERISCI QUI IL GRAFICO ---
\begin{figure}[ht!] % Ho cambiato [h!] in [ht!] come suggerito dal warning
\centering
\begin{tikzpicture}[scale=1.2]
    % Assi
    \draw[->] (-0.5,0) -- (4.5,0) node[below] {$x$};
    \draw[->] (0,-0.5) -- (0,3) node[left] {$f(x)$};
    
    % Curva f(x)
    \draw[thick, color=red!80!black] plot[smooth, domain=0:4] (\x, {0.2*(\x-1)^2 + 0.1*\x});
    \node[above, color=red!80!black] at (3.5, 2) {$f(x)$};
    
    % Punto x0 e tangente
    \coordinate (x0) at (3.5, {0.2*(3.5-1)^2 + 0.1*3.5});
    \fill (x0) circle (1.5pt);
    \node[below right] at (x0) {$(x_0, f(x_0))$};
    \draw (3.5,-0.1) node[below] {$x_0$};
    
    % Calcolo della derivata in x0 (pendenza)
    % f'(x) = 0.4*(x-1) + 0.1 -> f'(3.5) = 0.4*2.5 + 0.1 = 1.1
    % Retta: y - f(x0) = 1.1 * (x - x0)
    \draw[color=green!60!black] (x0) -- +(-1.5, -1.5*1.1) coordinate (endtan0); % Disegna un pezzo della tangente
    \draw[color=green!60!black] (x0) -- +(0.5, 0.5*1.1); % Altro pezzo
    
    % Intersezione x1
    % y=0 => -f(x0) = 1.1 * (x1 - x0) => x1 = x0 - f(x0)/1.1
    % f(3.5) = 0.2*2.5^2 + 0.1*3.5 = 0.2*6.25 + 0.35 = 1.25 + 0.35 = 1.6
    % x1 = 3.5 - 1.6/1.1 = 3.5 - 1.4545... = 2.045...
    \coordinate (x1_inter) at (3.5 - 1.6/1.1, 0);
    \draw[dashed, color=green!60!black] (x0) -- (x1_inter);
    \draw (x1_inter)+(0,-0.1) node[below] {$x_1$};
    \fill (x1_inter) circle (1.5pt);

    % Punto x1 sulla curva e tangente
    \coordinate (x1) at (3.5 - 1.6/1.1, {0.2*((3.5 - 1.6/1.1)-1)^2 + 0.1*(3.5 - 1.6/1.1)});
    \fill (x1) circle (1.5pt);
    % f'(x1) = 0.4*(x1-1) + 0.1 = 0.4*(1.045...) + 0.1 = 0.418 + 0.1 = 0.518
    \draw[color=green!60!black] (x1) -- +(-1.0, -1.0*0.518);
    \draw[color=green!60!black] (x1) -- +(0.5, 0.5*0.518);
    
    % Intersezione x* (radice) - approssimata
    \coordinate (x_star) at (1,0); % f(1)=0.1, ma facciamo finta che sia la radice per semplicità
    \draw (x_star)+(0,-0.1) node[below] {$x^*$};
    \fill[red] (x_star) circle (1.5pt);

\end{tikzpicture}
\caption{Interpretazione geometrica del metodo di Newton.}
\label{fig:newton}
\end{figure}
% --- FINE GRAFICO ---

\paragraph{Considerazioni sul Metodo di Newton}
\begin{enumerate}
    \item È richiesta la derivabilità di $f(x)$ (almeno $f \in C^2$ in un intorno della radice per l'analisi).
    \item Il costo per iterazione è 1 valutazione di $f(x)$ e 1 valutazione di $f'(x)$.
\end{enumerate}

\begin{teorema}[Convergenza di Newton]
    Sia $f(x) \in C^{(2)}$ in un intorno della radice $x^*$. Supponiamo che il metodo di Newton converga a $x^*$, e che $x^*$ sia semplice. Allora l'ordine di convergenza è (almeno) 2.
\end{teorema}
\begin{proof}[Dimostrazione (schizzo)]
Sviluppando $f(x^*)$ in Taylor attorno a $x_i$:
$$ 0 = f(x^*) = f(x_i) + f'(x_i)(x^* - x_i) + \frac{1}{2} f''(\xi_i)(x^* - x_i)^2, \quad \xi_i \in (x_i, x^*) $$
Ponendo $e_i = x^* - x_i$ e $e_{i+1} = x^* - x_{i+1}$:
$$ 0 = f(x_i) + f'(x_i)e_i + \frac{1}{2} f''(\xi_i)e_i^2 $$
Dividendo per $f'(x_i)$ e usando $x_{i+1} = x_i - f(x_i)/f'(x_i)$:
$$ 0 = -(x_{i+1} - x_i) + e_i + \frac{1}{2} \frac{f''(\xi_i)}{f'(x_i)} e_i^2 $$
$$ 0 = -(x_{i+1} - x_i) + (x^* - x_i) + \frac{1}{2} \frac{f''(\xi_i)}{f'(x_i)} e_i^2 = (x^* - x_{i+1}) + \frac{1}{2} \frac{f''(\xi_i)}{f'(x_i)} e_i^2 $$
$$ e_{i+1} = - \frac{1}{2} \frac{f''(\xi_i)}{f'(x_i)} e_i^2 $$
Passando al limite per $i \to \infty$:
$$ \lim_{i \to \infty} \frac{|e_{i+1}|}{|e_i|^2} =  \frac{1}{2} \left|\frac{f''(x^*)}{f'(x^*)} \right| = C $$
Questo dimostra la convergenza quadratica ($p=2$).
\end{proof}

\begin{osservazione}[Radici multiple]
    Nel caso di una radice multipla, con molteplicità $m > 1$, si può dimostrare che:
    $$ \lim_{i \to \infty} \frac{|e_{i+1}|}{|e_i|} = \frac{m-1}{m} $$
    ovvero, l'ordine di convergenza del metodo di Newton diventa lineare ($p=1$), come conseguenza del mal condizionamento del problema.
\end{osservazione}

\subsection*{Riepilogo dei Metodi Visti}
\paragraph{Metodo di Bisezione}
\begin{itemize}
    \item Applicabile se $f \in C([a,b])$ e $f(a)f(b) < 0$.
    \item Ordine di convergenza lineare ($p=1$), con costante asintotica $C=1/2$.
    \item Il numero massimo di iterazioni per raggiungere una data accuratezza (\texttt{tol}) è noto a priori: $\text{imax} = \lceil \log_2(b-a) - \log_2(\text{tol}) \rceil$.
\end{itemize}

\paragraph{Metodo di Newton}
Formula iterativa: $x_{i+1} = x_i - \frac{f(x_i)}{f'(x_i)}$.
\begin{itemize}
    \item Richiede che $f(x)$ sia derivabile.
    \item Se $f \in C^2$ in un intorno di una radice semplice, allora si ha, se convergente alla radice, convergenza \textbf{quadratica} ($p=2$).
    \item Se converge a una radice \textbf{multipla} $x^*$ (molteplicità $m>1$), la convergenza è solo \textbf{lineare} ($p=1$) con $C = \frac{m-1}{m}$, riflettendo il mal condizionamento del problema.
\end{itemize}

\subsection{Convergenza Locale vs Globale}
A differenza del metodo di bisezione, per il metodo di Newton non è in generale possibile garantire la convergenza a partire da un punto iniziale $x_0$ qualsiasi.

\begin{esempio}[Non convergenza di Newton]
Consideriamo $f(x) = x^3 - 5x = x(x^2-5)$, che ha tre radici semplici: $0, \pm\sqrt{5}$. Applichiamo il metodo di Newton partendo da $x_0=1$.
$f'(x) = 3x^2 - 5$.
$$ x_1 = x_0 - \frac{f(x_0)}{f'(x_0)} = 1 - \frac{1^3 - 5(1)}{3(1)^2 - 5} = 1 - \frac{-4}{-2} = 1 - 2 = -1 $$
$$ x_2 = x_1 - \frac{f(x_1)}{f'(x_1)} = -1 - \frac{(-1)^3 - 5(-1)}{3(-1)^2 - 5} = -1 - \frac{-1 + 5}{3 - 5} = -1 - \frac{4}{-2} = -1 + 2 = 1 $$
La successione generata è $\{1, -1, 1, -1, \dots\}$, che evidentemente non converge.
\end{esempio}

% --- GRAFICO ESEMPIO NEWTON ---
\begin{figure}[H] % Usiamo [H] dal pacchetto float per forzare la posizione
\centering
\begin{tikzpicture}[scale=1]
    % Assi
    \draw[->] (-2.7,0) -- (2.7,0) node[below] {$x$};
    \draw[->] (0,-5) -- (0,5) node[left] {$y$};
    \node[above right] at (1.5,3) {$f(x) = x^3 - 5x$};
    
    % Curva f(x)
    \draw[thick, color=teal, domain=-2.5:2.5, samples=100] plot (\x, {\x*\x*\x - 5*\x});
    
    % Punti x0=1 e x1=-1
    \coordinate (p1) at (1, -4);  % (1, f(1))
    \coordinate (m1) at (-1, 4);  % (-1, f(-1))
    
    % Cerchietti sui punti
    \draw[red] (p1) circle (3pt);
    \draw[red] (m1) circle (3pt);
    
    % Linee tratteggiate verticali
    \draw[dashed] (1,0) node[below] {$1$} -- (p1);
    \draw[dashed] (-1,0) node[below] {$-1$} -- (m1);
    
    % Tangenti (visualizzate come segmenti tra i punti)
    \draw[thick, color=magenta] (p1) -- (m1); 
    
\end{tikzpicture}
\caption{Metodo di Newton per $f(x)=x^3-5x$ partendo da $x_0=1$. Gli iterati oscillano tra 1 e -1.}
\label{fig:newton_oscillante}
\end{figure}
% --- FINE GRAFICO ---

La conclusione è che la convergenza del metodo di Newton è garantita solo in un opportuno \textbf{intorno} della radice. Si parla in questo caso di \textbf{convergenza locale}. Il metodo di bisezione, invece, ha \textbf{convergenza globale} (se applicabile, converge sempre).

\subsection{Teoria del Punto Fisso e Convergenza Locale}
Formalizziamo questo concetto per un generico metodo iterativo:
\begin{equation}
    x_{i+1} = \Phi(x_i), \quad i=0, 1, \dots
\end{equation}
dove $\Phi(x)$ è detta \textbf{funzione di iterazione}. Ad esempio, per Newton, $\Phi(x) = x - \frac{f(x)}{f'(x)}$.

Se il metodo serve a determinare la radice $x^*$ di $f(x)$, allora $x^*$ deve soddisfare la \textbf{proprietà di consistenza}:
\begin{equation}
    x^* = \Phi(x^*)
\end{equation}
ovvero, $x^*$ deve essere un \textbf{punto fisso} della funzione di iterazione $\Phi(x)$. Il problema di trovare uno zero di $f(x)$ è quindi equivalente a trovare un punto fisso di $\Phi(x)$.

Vogliamo capire sotto quali condizioni, partendo da un intorno del suo punto fisso, la successione (4) converge a $x^*$.

\begin{teorema}[del Punto Fisso / delle Contrazioni]
Sia $x^*$ un punto fisso di $\Phi(x)$. Se esiste $\delta > 0$ tale che per ogni $x, y \in I(x^*) = [x^*-\delta, x^*+\delta]$ la funzione $\Phi(x)$ è \textbf{Lipschitziana} con costante $L < 1$, cioè:
$$ |\Phi(x) - \Phi(y)| \le L |x - y| $$
allora:
\leavevmode % Aggiunto per risolvere potenziali conflitti
\begin{enumerate}
    \item Se $x_0 \in I(x^*)$, tutti i successivi iterati $x_i$ rimangono in $I(x^*)$ ($x_i \in I(x^*)$ per ogni $i \ge 0$).
    \item La successione $\{x_i\}$ converge a $x^*$.
\end{enumerate}
\end{teorema}
\begin{proof}
Dimostriamo per induzione che $x_i \in I(x^*)$. È vero per $i=0$. Supponiamo $x_i \in I(x^*)$. Allora:
$$ |x^* - x_{i+1}| = |\Phi(x^*) - \Phi(x_i)| \le L |x^* - x_i| $$
Poiché $L<1$ e $|x^* - x_i| \le \delta$, segue che $|x^* - x_{i+1}| < \delta$, quindi $x_{i+1} \in I(x^*)$.
Inoltre, applicando ricorsivamente la disuguaglianza:
$$ |x^* - x_i| \le L |x^* - x_{i-1}| \le L^2 |x^* - x_{i-2}| \le \dots \le L^i |x^* - x_0| $$
Dato che $L<1$, $L^i \to 0$ per $i \to \infty$, quindi $\lim_{i \to \infty} |x^* - x_i| = 0$.
\end{proof}

\begin{corollario}
    Se $\exists \delta > 0$ tale che $\forall x \in [x^*-\delta, x^*+\delta] = I(x^*)$ si ha $|\Phi'(x)| \le L < 1$, allora la successione $x_{i+1} = \Phi(x_i)$ converge a $x^*$ per $i \to \infty$.
\end{corollario}
\begin{proof}
Dal teorema del valor medio, per $x, y \in I(x^*)$, esiste $\xi$ tra $x$ e $y$ tale che:
$$ |\Phi(x) - \Phi(y)| = |\Phi'(\xi)(x-y)| = |\Phi'(\xi)| |x-y| \le L |x-y| $$
Quindi $\Phi(x)$ è Lipschitziana con costante $L<1$ e si applica il teorema precedente.
\end{proof}

\paragraph{Applicazione al Metodo di Newton}
Vediamo come si applica il risultato precedente al metodo di Newton. La funzione di iterazione è:
$$ \Phi(x) = x - \frac{f(x)}{f'(x)} $$
Verifichiamo la consistenza nel punto fisso $x^*$:
$$ \Phi(x^*) = x^* - \frac{f(x^*)}{f'(x^*)} = x^* - \frac{0}{f'(x^*)} = x^* $$
Calcoliamo ora la derivata prima di $\Phi(x)$:
$$ \Phi'(x) = \frac{d}{dx}\left(x - \frac{f(x)}{f'(x)}\right) = 1 - \frac{f'(x)f'(x) - f(x)f''(x)}{[f'(x)]^2} = \frac{[f'(x)]^2 - [f'(x)]^2 + f(x)f''(x)}{[f'(x)]^2} = \frac{f(x)f''(x)}{[f'(x)]^2} $$
Valutiamo la derivata nel punto fisso $x^*$:
\begin{itemize}
    \item \textbf{Se $x^*$ è una radice semplice}: In questo caso $f(x^*) = 0$ ma $f'(x^*) \neq 0$. Quindi:
    $$ \Phi'(x^*) = \frac{f(x^*)f''(x^*)}{[f'(x^*)]^2} = \frac{0 \cdot f''(x^*)}{[f'(x^*)]^2} = 0 $$
    Poiché $\Phi'(x^*) = 0 < 1$, per la continuità di $\Phi'(x)$, esisterà un intorno $I(x^*)$ tale che $|\Phi'(x)| \le L < 1$ per $x \in I(x^*)$. Di conseguenza, per il corollario precedente, il metodo di Newton converge localmente. Il fatto che $\Phi'(x^*) = 0$ è condizione sufficiente per la convergenza quadratica.
    
    % --- GRAFICO DERIVATA PHI ---
    \begin{figure}[H]
    \centering
    \begin{tikzpicture}[scale=1]
        % Assi
        \draw[->] (-2.5,0) -- (2.5,0) node[below] {$x$};
        \draw[->] (0,-1.5) -- (0,1.5) node[left] {}; % Asse y senza etichetta
        \node[above right] at (1.5,1) {$\Phi'(x)$};
        
        % Curva Phi'(x) passante per x* con derivata 0
        \draw[thick, domain=-2:2, samples=50] plot (\x, {0.5*(\x-0.5)}); % Curva generica con Phi'(x*)=0
        
        % Punto fisso x*
        \coordinate (xstar) at (0.5, 0);
        \fill[red] (xstar) circle (1.5pt);
        \node[below, red] at (xstar) {$x^*$};
        
        % Linee per L e -L (con L < 1)
        \draw[dashed, color=green!60!black] (-2,1) node[left] {$L$} -- (2,1);
        \draw[dashed, color=green!60!black] (-2,-1) node[left] {$-L$} -- (2,-1);
        
        % Intervallo I(x*) dove |Phi'(x)| < L
        % Trova i punti x tali che Phi'(x) = L e Phi'(x) = -L
        % 0.5*(x-0.5) = 1 => x-0.5=2 => x=2.5
        % 0.5*(x-0.5) = -1 => x-0.5=-2 => x=-1.5
        \draw[green!60!black] (-1.5, -1) -- (-1.5, {0.5*(-1.5-0.5)}); % Linea verticale sinistra
        \draw[green!60!black] (2.5, 1) -- (2.5, {0.5*(2.5-0.5)});    % Linea verticale destra (fuori asse)
                
        % Indicazione dell'intervallo sull'asse x
        \draw[|<->|,thick, green!60!black] (-1.5, -1.2) -- (2.5, -1.2) node[midway, below] {$I(x^*)$}; 
        
    \end{tikzpicture}
    \caption{Intorno $I(x^*)$ di una radice semplice dove $|\Phi'(x)| \le L < 1$, garantendo la convergenza locale.}
    \label{fig:phi_derivata_newton}
    \end{figure}
    % --- FINE GRAFICO ---

    \item \textbf{Se $x^*$ è una radice multipla} di molteplicità $m > 1$: Si può dimostrare che:
    $$ \Phi'(x^*) = \frac{m-1}{m} $$
    Anche in questo caso $|\Phi'(x^*)| < 1$, quindi il metodo converge ancora localmente (per il corollario). Tuttavia, poiché $\Phi'(x^*) \neq 0$, la convergenza diventa meno favorevole.
\end{itemize}
\subsection{Criteri di Arresto}
Per un metodo iterativo $x_{i+1} = \Phi(x_i)$, cerchiamo un criterio per fermare l'iterazione.

\paragraph{Criterio basato sul residuo}
Come visto per la bisezione, potremmo richiedere $|f(x_i)| \le \text{tol} \cdot |f'(x_i)|$ (approssimando $f'(x^*)$ con $f'(x_i)$). Dalla formula di Newton, questo è equivalente a:
$$ \left| \frac{f(x_i)}{f'(x_i)} \right| \le \text{tol} \implies |x_{i+1} - x_i| \le \text{tol} $$

\paragraph{Criterio basato sulla differenza tra iterati}
Il criterio $|x_{i+1} - x_i| \le \text{tol}$ controlla l'errore assoluto. Tuttavia, se $|x^*| \gg 1$, sarebbe più significativo controllare l'errore relativo:
$$ \frac{|x^* - x_i|}{|x^*|} \le \text{tol} $$
che può essere approssimato da:
$$ \frac{|x_{i+1} - x_i|}{|x_i|} \le \text{tol} \quad (\text{o } |x_{i+1}| \text{ al denominatore}) $$

\paragraph{Criterio combinato (auto-scalante)}
Un criterio pratico che combina i due è:
$$ |x_{i+1} - x_i| \le (1 + |x_i|) \cdot \text{tol} \quad \text{o equivalentemente} \quad \frac{|x_{i+1} - x_i|}{1 + |x_i|} \le \text{tol} $$
Questo criterio "auto-scalante" si comporta come un controllo sull'errore assoluto quando $|x_i| \approx 0$ e come un controllo sull'errore relativo quando $|x_i| \gg 1$, adattandosi alla grandezza della radice a cui si sta convergendo. \textbf{N.B.}: Da usare nei nostri elaborati.

\end{document}