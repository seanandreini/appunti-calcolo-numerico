\documentclass{article}
\usepackage[utf8]{inputenc}
\usepackage[italian]{babel}
\usepackage{tikz}
\usepackage{amsmath}
\usepackage{amssymb}
\usepackage{amsthm}
\usepackage{geometry}
\usepackage{listings} % <-- QUESTA È LA RIGA CHE RISOLVE L'ERRORE
\usepackage{xcolor}   % <-- UTILE PER COLORARE IL CODICE
\usepackage{booktabs}
\usepackage{float}
\geometry{a4paper, margin=1in}


% --- Stili e Ambienti Personalizzati ---
\lstset{
  language=Matlab,
  basicstyle=\small\ttfamily,
  keywordstyle=\color{blue},
  commentstyle=\color{green!60!black},
  stringstyle=\color{purple},
  showstringspaces=false,
  breaklines=true,
  frame=single,
  numbers=left,
  literate=%
    {à}{{\`a}}1
    {è}{{\`e}}1
    {ì}{{\`i}}1
    {ò}{{\`o}}1
    {ù}{{\`u}}1
}

\newtheoremstyle{break}
  {\topsep}{\topsep}%
  {\normalfont}{}%
  {\bfseries}{.}%
  {\newline}{}%
  
\theoremstyle{break}
\newtheorem{osservazione}{Osservazione}[section]

\theoremstyle{definition}
\newtheorem{definition}{Definizione}[section]
\newtheorem{esercizio}{Esercizio}[section]
\newtheorem{teorema}{Teorema}[section]
\newtheorem{corollario}{Corollario}[section]
\newtheorem{esempio}{Esempio}[section]
% --- Fine Stili ---

\title{Appunti di Calcolo Numerico}
\author{} % Lasciato vuoto o da compilare
\date{\today}

\begin{document}

\maketitle

% Comando per generare l'indice
\tableofcontents

\newpage % Opzionale: per iniziare l'introduzione in una nuova pagina

% Sezione numerata che verrà aggiunta all'indice
\section{Introduzione}

Molti problemi derivanti da applicazioni pratiche sono descritti tramite un modello matematico. Una volta che le equazioni del modello sono risolte, è possibile fare inferenza sul fenomeno studiato. Tuttavia, la soluzione di tali equazioni non è quasi mai direttamente disponibile, rendendo necessario l'utilizzo di metodi numerici e tecniche di approssimazione.

Questi metodi devono soddisfare alcuni requisiti fondamentali:
\begin{itemize}
    \item \textbf{Accuratezza}: La soluzione approssimata deve essere sufficientemente "vicina" alla soluzione esatta, in base alle specifiche del problema.
    \item \textbf{Facilità di implementazione}: La formulazione del metodo deve consentire una semplice traduzione in un algoritmo da codificare in un opportuno linguaggio di programmazione.
\end{itemize}

\section{Errori ed aritmetica finita}

Il risultato fornito da un metodo numerico è quasi sempre affetto da errore. L'errore commesso è determinato da più cause, spesso intercalate tra loro.

\subsection{Misure dell'Errore}
Supponiamo che $x \in \mathbb{R}$ sia il dato esatto e $\tilde{x}$ la sua approssimazione.

\begin{definition}[Errore Assoluto]
L'errore assoluto è definito come la differenza:
$$ \Delta x = \tilde{x} - x $$
da cui segue $\tilde{x} = x + \Delta x$.
\end{definition}

Questa misura da sola non è completamente esaustiva. Ad esempio, un errore $|\Delta x| = 10^{-6}$ potrebbe essere considerato "grande" o "piccolo" solo rapportandolo al valore esatto $x$.

\begin{definition}[Errore Relativo]
Per ovviare a questo problema, se $x \neq 0$, si introduce l'errore relativo:
$$ \epsilon_x = \frac{\Delta x}{x} = \frac{\tilde{x} - x}{x} $$
\end{definition}

Dalla definizione segue che:
$$ \tilde{x} = x(1 + \epsilon_x) \quad \text{ovvero} \quad \frac{\tilde{x}}{x} = 1 + \epsilon_x $$
Questo mostra che l'errore relativo $\epsilon_x$ deve essere confrontato con 1. Un errore relativo $|\epsilon_x|=10^{-6}$ fornisce un'informazione più "assoluta" sulla qualità dell'approssimazione.

\subsection{Tipologie Principali di Errore}
È possibile individuare, almeno a livello concettuale, tre tipologie principali di errore:
\begin{enumerate}
    \item Errori di troncamento;
    \item Errori di iterazione;
    \item Errori di round-off (o arrotondamento).
\end{enumerate}

\subsubsection{Errori di Discretizzazione (o Troncamento)}
Questi errori nascono quando un problema matematico formulato nel continuo viene sostituito da un problema discreto che lo approssima.

Ad esempio, per calcolare la derivata di una funzione $f(x)$ in un punto $x_0$, si parte dalla definizione:
$$ f'(x_0) = \lim_{h \to 0} \frac{f(x_0+h) - f(x_0)}{h} $$
Per il calcolo numerico, si utilizza un valore di $\bar{h}>0$ (fissato) "piccolo" ma finito, approssimando la derivata con il rapporto incrementale:
$$ f'(x_0) \approx \frac{f(x_0+\bar{h}) - f(x_0)}{\bar{h}} $$
L'errore di troncamento commesso è la differenza tra le due quantità: $f'(x) - \frac{f(x+\bar{h})-f(x)}{\bar{h}}$.
Considerando lo sviluppo in serie di Taylor di $f(x+\bar{h})$ centrato in $x$:
$$ f(x+\bar{h}) = f(x) + \bar{h} f'(x) + \frac{\bar{h}^2}{2} f''(x) + \dots $$
si può isolare il rapporto incrementale e stimare l'errore:
$$ \frac{f(x+\bar{h}) - f(x)}{\bar{h}} = f'(x) + O(\bar{h}) $$
L'errore di troncamento è quindi del primo ordine rispetto ad $h$, ovvero $O(h)$.



\subsubsection{Errori di Convergenza (o Iterazione)}
Molti metodi numerici sono di tipo iterativo: non forniscono la soluzione esatta $x^*$, ma generano una successione di approssimazioni $\{x_n\}$. A partire da un'approssimazione iniziale $x_0$, le approssimazioni successive sono definite da una funzione di iterazione $\Phi(x)$:
$$ x_{n+1} = \Phi(x_n), \quad n=0, 1, 2, \dots $$

\begin{definition}[Metodo Convergente]
Un metodo iterativo si dice convergente se la successione delle approssimazioni tende alla soluzione esatta:
$$ \lim_{n \to \infty} x_n = x^* $$
\end{definition}

Poiché è possibile eseguire solo un numero finito di iterazioni, il processo viene arrestato a un indice $N$, e si utilizza $x_N$ come approssimazione di $x^*$. L'errore $x^* - x_N$ è detto errore di iterazione. L'indice $N$ è tipicamente determinato dinamicamente tramite un criterio di arresto.

\begin{osservazione}
    \begin{enumerate}
        \item L'errore di iterazione è legato all'utilizzo del metodo di base ($\Phi(x)$).
        \item Praticamente sempre, l'indice $N$ a cui si interrompe l'iterazione è determinato dinamicamente mediante un opportuno criterio di arresto.
    \end{enumerate}
\end{osservazione}

\subsubsection{Errori di Round-off}
Questi errori sono dovuti all'utilizzo dell'aritmetica finita di un calcolatore. In particolare, gli errori di rappresentazione nascono dal fatto che non tutti i numeri possono essere rappresentati esattamente nella memoria di un computer.
Analizzeremo la rappresentazione di numeri interi e reali.

\paragraph{Numeri Interi}
Fissata una base di rappresentazione $b$ e un numero di cifre $N$, un numero intero viene memorizzato tramite una stringa del tipo:
$$ \alpha_0 \alpha_1 \dots \alpha_N $$
dove $\alpha_0 \in \{+,-\}$ è il segno e $\alpha_i \in \{0, 1, \dots, b-1\}$ sono le cifre. A questa stringa corrisponde il valore:
\[ n = 
\begin{cases} 
\sum_{i=1}^{N} \alpha_i b^{N-i} & \text{se } \alpha_0 = + \\
\sum_{i=1}^{N} \alpha_i b^{N-i} - b^N & \text{se } \alpha_0 = - \quad \text{(es. complemento a 2)}
\end{cases}
\]

Ad esempio, con $b=2$ e $N=15$ (16 bit totali, cioè 2 byte), è possibile rappresentare tutti gli interi nell'intervallo $[-32768, 32767]$. Se un intero rimane in questo intervallo, non ci sono errori di rappresentazione.

\subsubsection{Numeri reali}
Un numero "reale" è rappresentato in memoria da una stringa del tipo:
$$ \alpha_0 \alpha_1 \dots \alpha_m \beta_1 \dots \beta_s $$
Fissata una base di rappresentazione $b \in \mathbb{N}$ (pari), le cifre sono così definite:
\begin{itemize}
    \item $\alpha_0 \in \{+,-\}$
    \item $\alpha_i, \beta_j \in \{0, 1, \dots, b-1\}$, per $i=1,\dots,m$ e $j=1,\dots,s$, con $\alpha_1 \neq 0$.
\end{itemize}
Questa stringa rappresenta il numero in notazione scientifica normalizzata:
$$ \tilde{x} = \pm S \cdot b^{e-\nu} $$
dove $S$ è la \textbf{mantissa}:
$$ S = \sum_{i=1}^{m} \alpha_i b^{1-i} = (\alpha_1 . \alpha_2 \dots \alpha_m)_b $$
e $e$ è l'\textbf{esponente}, dato da:
$$ e = \sum_{j=1}^{s} \beta_j b^{s-j} $$
A questo si sottrae lo "shift" $\nu$, una costante intera fissata. Lo shift è scelto in modo da poter rappresentare circa lo stesso numero di esponenti positivi e negativi. Poiché $0 \le e \le b^s - 1$, si sceglie tipicamente $\nu \approx \frac{b^s}{2}$.

Per la mantissa, abbiamo che:
$$ 1 \le S < b $$

\begin{definition}[Numeri di Macchina]
L'insieme dei numeri della forma descritta, assieme allo zero, costituisce l'insieme dei \textbf{numeri di macchina} normalizzati, indicato con $\mathcal{M}$.
\end{definition}

\begin{osservazione}
\begin{enumerate}
    \item L'insieme $\mathcal{M}$ è un insieme finito.
    \item Il più piccolo numero di macchina positivo è $r_1 = 1 \cdot b^{0-\nu} = b^{-\nu}$.
    \item Il più grande numero di macchina positivo è $r_2 = (b - b^{1-m}) \cdot b^{(b^s-1)-\nu} \approx b^{b^s-\nu}$.
\end{enumerate}
\end{osservazione}

Tutti i numeri di macchina sono contenuti nell'intervallo:
$$ \mathcal{I} = [-r_2, -r_1] \cup \{0\} \cup [r_1, r_2] $$
Poiché $\mathcal{M}$ è un insieme discreto mentre $\mathcal{I}$ è denso, è necessario definire una funzione di "arrotondamento", detta \textbf{floating}, che associa a ogni numero reale $x \in \mathcal{I}$ un numero di macchina $\tilde{x} \in \mathcal{M}$.
$$ fl: x \in \mathcal{I} \to \tilde{x} = fl(x) \in \mathcal{M} $$
La quantità $fl(x)-x$ è l'errore di rappresentazione. Per costruzione, valgono le seguenti proprietà:
\begin{itemize}
    \item $fl(0) = 0$
    \item Se $x \in \mathcal{M}$, allora $fl(x) = x$
    \item Per $x>0$, $fl(-x) = -fl(x)$
\end{itemize}

Dato un generico $x \in \mathcal{I}$, $x > 0$, scritto come:
$$ x = (\alpha_1 . \alpha_2 \dots \alpha_m \alpha_{m+1} \dots)_b \cdot b^{e-\nu} $$
esistono due modi principali per implementare $fl(x)$:
\begin{enumerate}
    \item \textbf{Troncamento}: si tagliano le cifre della mantissa dopo la $m$-esima.
    $$ fl(x) = (\alpha_1 . \alpha_2 \dots \alpha_m)_b \cdot b^{e-\nu} $$
    
    \item \textbf{Arrotondamento}: si considera la prima cifra scartata, $\alpha_{m+1}$.
    $$ fl(x) = (\alpha_1 . \alpha_2 \dots \alpha_{m-1} \tilde{\alpha}_m)_b \cdot b^{e-\nu} $$
    con
    \[ \tilde{\alpha}_m = 
    \begin{cases} 
    \alpha_m & \text{se } \alpha_{m+1} < b/2 \\
    \alpha_m + 1 & \text{se } \alpha_{m+1} \ge b/2
    \end{cases}
    \]
\end{enumerate}

\begin{teorema}
Per i numeri $x \in \mathcal{I}$, l'errore relativo di rappresentazione è maggiorato da una costante $u$, detta \textbf{precisione di macchina}.
$$ \epsilon_x = \frac{|fl(x)-x|}{|x|} \le u = 
\begin{cases} 
b^{1-m} & \text{in caso di troncamento} \\
\frac{1}{2}b^{1-m} & \text{in caso di arrotondamento}
\end{cases}
$$
\end{teorema}
\begin{proof}[Dimostrazione (solo per il troncamento)]
$$ \epsilon_x = \frac{|x-fl(x)|}{|x|} = \frac{|(0.0 \dots 0 \alpha_{m+1} \dots)_b \cdot b^{e-\nu}|}{|(\alpha_1 . \alpha_2 \dots)_b \cdot b^{e-\nu}|} = \frac{|(\alpha_{m+1}.\alpha_{m+2}\dots)_b \cdot b^{-m}|}{|(\alpha_1.\alpha_2\dots)_b|} \le \frac{b \cdot b^{-m}}{1} = b^{1-m} $$
\end{proof}

\begin{osservazione}
    Pertanto, concludiamo che la precisione di macchina di un'aritmetica finita è una maggiorazione uniforme dell'errore relativo di rappresentazione.
    \end{osservazione}

\subsubsection{Overflow e Underflow}
Cosa succede se $x \notin \mathcal{I}$?
\begin{itemize}
    \item Se $x > r_2$, si ha una condizione di \textbf{overflow}. La recovery standard è porre $fl(x) = \pm\infty$.
    \item Se $0 < x < r_1$, si ha una condizione di \textbf{underflow}. Esistono due tipi di recovery:
    \begin{enumerate}
        \item \textbf{Store to zero}: si pone $fl(x) = 0$.
        \item \textbf{Gradual underflow}: si permette alla prima cifra della mantissa, $\alpha_1$, di essere zero (numero denormalizzato). Questo estende l'intervallo di rappresentabilità vicino allo zero, ma a discapito della precisione (il Teorema precedente non è più valido).
    \end{enumerate}
\end{itemize}

In conclusione, per la funzione $fl(x)$ vale che:
\[
fl(x) = 
\begin{cases} 
x, & \text{se } x \in \mathcal{M} \\
-fl(-x), & \text{se } x < 0 \\
\text{underflow}, & \text{se } 0 < |x| < r_1 \\
\text{overflow}, & \text{se } |x| > r_2
\end{cases}
\]

\subsubsection{Lo standard IEEE 754}
Lo standard IEEE 754 definisce un formato comune per l'aritmetica in virgola mobile, garantendo che i calcoli producano gli stessi risultati su piattaforme diverse.
Le sue caratteristiche principali sono:
\begin{itemize}
    \item \textbf{Base binaria} ($b=2$).
    \item \textbf{Arrotondamento "round to even"}: in caso di ambiguità (quando la prima cifra scartata è esattamente a metà), si sceglie il numero di macchina la cui ultima cifra della mantissa sia pari (cioè 0). Nonostante questa particolarità, la maggiorazione dell'errore relativo $u = \frac{1}{2}b^{1-m}$ continua a valere.
    \item \textbf{Gradual underflow}: viene implementata la gestione dei numeri denormalizzati.
\end{itemize}
Dato che la base è binaria, la parte intera della mantissa di un numero è sempre nota:
\begin{itemize}
    \item È **1** per i numeri \textbf{normalizzati} (forma $1.f$).
    \item È **0** per i numeri \textbf{denormalizzati} (forma $0.f$).
\end{itemize}
Questo permette di non memorizzare la parte intera, ma solo la parte frazionaria $f$, risparmiando 1 bit.

Lo standard prevede due formati principali:

\paragraph{Singola Precisione (32 bit - 4 byte)}
I 32 bit sono così ripartiti:
\begin{itemize}
    \item \textbf{1 bit} per il segno della mantissa ($\alpha_0$).
    \item \textbf{8 bit} per l'esponente ($s=8$), quindi $0 \le e \le 255$.
    \item \textbf{23 bit} per la parte frazionaria $f$ (la mantissa ha $m=24$ bit).
\end{itemize}
La precisione di macchina in singola precisione è $u = \frac{1}{2} \cdot 2^{1-24} = 2^{-24} \approx 5.96 \times 10^{-8}$.

L'esponente $e$ assume significati speciali:
\begin{itemize}
    \item Se $0 < e < 255$: numero \textbf{normalizzato}, con shift $\nu = 127$.
    \item Se $e=0$ e $f=0$: rappresenta lo \textbf{zero}.
    \item Se $e=0$ e $f \neq 0$: numero \textbf{denormalizzato}, con shift $\nu = 126$.
    \item Se $e=255$ e $f=0$: rappresenta $\pm\infty$ (a seconda del segno).
    \item Se $e=255$ e $f \neq 0$: rappresenta un \textbf{NaN} (Not a Number), generato da forme indeterminate come $\infty-\infty$, $0 \cdot \infty$, $\frac{0}{0}$.
\end{itemize}

\begin{osservazione}[Variazione dello shift e contiguità]
    La variazione dello shift tra numeri normalizzati e denormalizzati si spiega osservando la transizione tra i due insiemi, che lo standard rende il più graduale possibile.
    \begin{itemize}
        \item Il più piccolo numero \textbf{normalizzato} positivo si ha con la mantissa minima ($1.0...0$) e l'esponente minimo per i normalizzati ($e=1$), risultando in:
        $$(1.0\dots0)_2 \times 2^{1-127} = 2^{-126}$$
    
        \item Il più grande numero \textbf{denormalizzato} positivo si ha con la mantissa massima ($0.1...1$) e l'esponente fisso per i denormalizzati, che è uguale a quello dei più piccoli normalizzati:
        $$(0.1\dots1)_2 \times 2^{-126} = (1 - 2^{-23}) \times 2^{-126}$$
    \end{itemize}
    Pertanto, i due numeri sono "contigui": il più grande denormalizzato è immediatamente precedente al più piccolo normalizzato, garantendo una transizione fluida verso lo zero (da cui il nome \emph{gradual underflow}).
    \end{osservazione}

\paragraph{Doppia Precisione (64 bit - 8 byte)}
I 64 bit sono così ripartiti:
\begin{itemize}
    \item \textbf{1 bit} per il segno.
    \item \textbf{11 bit} per l'esponente ($s=11$), quindi $0 \le e \le 2047$.
    \item \textbf{52 bit} per la parte frazionaria $f$ ($m=53$ bit).
\end{itemize}
La precisione di macchina è $u = \frac{1}{2} \cdot 2^{1-53} = 2^{-53} \approx 1.11 \times 10^{-16}$, il che significa lavorare con circa 16 cifre decimali significative.

Le regole per l'esponente sono analoghe alla singola precisione:
\begin{itemize}
    \item Se $0 < e < 2047$: numero \textbf{normalizzato}, con shift $\nu = 1023$.
    \item Se $e=0$ e $f=0$: rappresenta lo \textbf{zero}.
    \item Se $e=0$ e $f \neq 0$: numero \textbf{denormalizzato}, con shift $\nu = 1022$.
    \item Se $e=2047$ e $f=0$: rappresenta $\pm\infty$.
    \item Se $e=2047$ e $f \neq 0$: rappresenta un \textbf{NaN}.
\end{itemize}

\subsubsection{Aritmetica Finita}
Le operazioni algebriche elementari $(+, -, *, /)$ in aritmetica finita sono definite come segue, per due numeri reali $x, y \in \mathbb{R}$:
$$ x \oplus y = fl(fl(x) + fl(y)) $$
Questo implica che le comuni proprietà algebriche (associativa, distributiva) in genere non valgono più.

\begin{osservazione}[Esempi in Matlab]
I seguenti comandi Matlab mostrano la perdita della proprietà associativa:
\begin{lstlisting}
    >> r2 = realmax; % Il più grande numero in doppia precisione
    >> [(r2 - r2) + 1, r2 - (r2 + 1)]
    ans =
         1     0
    \end{lstlisting}
E la gestione di operazioni con Infinito:
\begin{lstlisting}
>> [(r2 - r2) * 2, r2 * 2 - r2 * 2]
ans =
     0   NaN
\end{lstlisting}
\end{osservazione}


\subsubsection{Conversione tra tipi diversi}
La conversione tra tipi di dati numerici, come tra diverse precisioni di numeri reali o tra reali e interi, è un'operazione delicata che può introdurre errori significativi.

\paragraph{Conversione tra Reali (doppia $\leftrightarrow$ singola precisione)}
Consideriamo una variabile `x` in doppia precisione e una `y` in singola precisione.
\begin{itemize}
    \item \textbf{Da doppia a singola}: Se si assegna un valore in doppia precisione (es. $\pi$) a una variabile in singola, il valore verrà memorizzato con l'accuratezza massima consentita dalla singola precisione, perdendo le cifre eccedenti.
    \begin{verbatim}
    x = pi; % Doppia precisione
    y = single(x); % y contiene pi con accuratezza ridotta
    \end{verbatim}
    
    \item \textbf{Da singola a doppia}: Se si esegue l'operazione inversa, la precisione persa non viene recuperata. La nuova variabile a doppia precisione manterrà l'accuratezza limitata del dato di partenza.
    \begin{verbatim}
    y = single(pi);
    x = double(y); % x ha la stessa (bassa) accuratezza di y
    \end{verbatim}
\end{itemize}

\paragraph{Conversione Reale $\leftrightarrow$ Intero}
\begin{itemize}
    \item \textbf{Da Intero a Reale}: Questa conversione è in genere innocua. Poiché l'insieme dei numeri reali rappresentabili è molto più ampio e denso, un intero può quasi sempre essere convertito in un reale. Si può avere una perdita di precisione solo se l'intero ha più cifre significative di quante la mantissa del tipo reale possa contenere.
    
    \item \textbf{Da Reale a Intero}: Questa è un'operazione \textbf{molto pericolosa}. L'intervallo di rappresentabilità degli interi (es. $[-32768, 32767]$ per interi a 2 byte) è estremamente più ristretto di quello dei numeri reali. Se il numero reale da convertire è al di fuori di questo intervallo, si verifica un errore di overflow.
\end{itemize}

\subsection{Condizionamento di un problema}
Supponiamo di voler calcolare la soluzione di un problema che, per semplicità, formalizziamo come:
\begin{equation}
    y = f(x)
\end{equation}
dove $f: \mathbb{R} \to \mathbb{R}$ è una funzione sufficientemente regolare, $x$ è il dato di ingresso e $y$ è il risultato atteso.

In un contesto reale, invece di lavorare con i dati e la funzione esatti, spesso si ha a che fare con un dato perturbato $\tilde{x}$ e, a causa dell'aritmetica finita del calcolatore, si utilizza una funzione perturbata $\tilde{f}$. Questo porta a un risultato anch'esso perturbato:
\begin{equation}
    \tilde{y} = \tilde{f}(\tilde{x})
\end{equation}

\begin{osservazione}
Analizzare la differenza completa tra il risultato esatto e quello calcolato (cioè tra l'equazione (1) e la (2)) è in genere molto complesso. Ci limiteremo a un'analisi più semplice, studiando come le perturbazioni sui soli dati di ingresso influenzino il risultato, assumendo di utilizzare un'aritmetica esatta:
\begin{equation}
    \tilde{y} = f(\tilde{x})
\end{equation}
Lo studio della differenza tra il risultato dell'equazione (3) e quello dell'equazione (1) costituisce l'analisi del \textbf{condizionamento del problema}.
\end{osservazione}

Per $y \neq 0$, l'analisi è più efficace se condotta in termini di errori relativi. Poniamo:
\[
\begin{cases}
    \tilde{x} = x(1 + \epsilon_x) \\
    \tilde{y} = y(1 + \epsilon_y)
\end{cases}
\]
dove $\epsilon_x$ e $\epsilon_y$ sono gli errori relativi sul dato di ingresso e sul risultato, rispettivamente. Vogliamo stabilire come $\epsilon_x$ si propaga su $\epsilon_y$, supponendo $|\epsilon_x| \ll 1$.

Sostituendo le definizioni nella (3) e usando lo sviluppo di Taylor, otteniamo:
$$ y(1 + \epsilon_y) = f(x(1+\epsilon_x)) \approx f(x) + f'(x) \cdot (x \epsilon_x) $$
Dato che $y = f(x)$, si ha:
$$ y + y \epsilon_y \approx y + f'(x) x \epsilon_x \implies y \epsilon_y \approx f'(x) x \epsilon_x $$
Da cui si ricava la relazione tra gli errori relativi:
$$ \epsilon_y \approx \frac{f'(x)x}{y} \epsilon_x $$
Questo ci porta a definire il numero di condizione.

\begin{definition}[Numero di Condizione]
Il fattore di amplificazione dell'errore
$$ K = \left| \frac{f'(x)x}{y} \right| $$
è detto \textbf{numero di condizione} del problema. Vale la relazione:
$$ |\epsilon_y| \approx K \cdot |\epsilon_x| $$
\end{definition}

Un problema si dice:
\begin{itemize}
    \item \textbf{ben condizionato}, se $K \approx 1$;
    \item \textbf{mal condizionato}, se $K \gg 1$ (molto maggiore).
\end{itemize}

\begin{osservazione}
Se si lavora in un'aritmetica finita con precisione di macchina $u$, l'errore relativo sul dato di ingresso sarà almeno $|\epsilon_x| \approx u$. Se il problema ha un numero di condizione $K \approx u^{-1}$, l'errore relativo sul risultato sarà $|\epsilon_y| \approx 1$. Questo significa una perdita totale di cifre significative, rendendo il risultato inattendibile.
\end{osservazione}

\subsubsection*{Condizionamento delle Operazioni Algebriche Elementari}

\paragraph{Moltiplicazione}
Per il problema $y = x_1 x_2$, perturbando gli input si ha:
$$ y(1+\epsilon_y) = x_1(1+\epsilon_1) x_2(1+\epsilon_2) = x_1 x_2 (1 + \epsilon_1 + \epsilon_2 + \epsilon_1 \epsilon_2) $$
Trascurando i termini di ordine superiore, $y(1+\epsilon_y) \approx y(1+\epsilon_1+\epsilon_2)$, da cui:
$$ |\epsilon_y| \approx |\epsilon_1 + \epsilon_2| \le |\epsilon_1| + |\epsilon_2| \le 2 \max\{|\epsilon_1|, |\epsilon_2|\} $$
La moltiplicazione è un'operazione \textbf{ben condizionata}, con $K=2$.

\paragraph{Divisione}
Per il problema $y = x_1 / x_2$, si ha:
$$ y(1+\epsilon_y) = \frac{x_1(1+\epsilon_1)}{x_2(1+\epsilon_2)} = \frac{x_1}{x_2}(1+\epsilon_1)(1+\epsilon_2)^{-1} $$
Usando lo sviluppo $(1+\epsilon)^{-1} \approx 1 - \epsilon$, otteniamo:
$$ y(1+\epsilon_y) \approx y(1+\epsilon_1)(1-\epsilon_2) \approx y(1+\epsilon_1 - \epsilon_2) $$
$$ |\epsilon_y| \approx |\epsilon_1 - \epsilon_2| \le 2 \max\{|\epsilon_1|, |\epsilon_2|\} $$
Anche la divisione è un'operazione \textbf{ben condizionata}, con $K=2$.

\paragraph{Somma Algebrica}
Per il problema $y = x_1 + x_2$, la perturbazione porta a:
$$ y(1+\epsilon_y) = x_1(1+\epsilon_1) + x_2(1+\epsilon_2) = x_1+x_2 + x_1\epsilon_1 + x_2\epsilon_2 $$
$$ y\epsilon_y = x_1\epsilon_1 + x_2\epsilon_2 \implies \epsilon_y = \frac{x_1}{x_1+x_2}\epsilon_1 + \frac{x_2}{x_1+x_2}\epsilon_2 $$
Il numero di condizione è:
$$ K = \frac{|x_1| + |x_2|}{|x_1+x_2|} $$
Si distinguono due casi:
\begin{itemize}
    \item \textbf{Addendi concordi ($x_1 x_2 > 0$)}: In questo caso $|x_1+x_2| = |x_1|+|x_2|$, quindi $K=1$. La somma di numeri concordi è \textbf{ben condizionata}.
    \item \textbf{Addendi discordi ($x_1 x_2 < 0$)}: Se $x_1 \approx -x_2$, il denominatore $|x_1+x_2|$ diventa molto piccolo, e $K$ può diventare arbitrariamente grande. La somma di numeri quasi opposti è un'operazione \textbf{mal condizionata}. Questo porta al fenomeno della \textbf{cancellazione numerica}, in cui il risultato può essere molto inaccurato anche se gli addendi sono noti con grande precisione.
\end{itemize}

\subsubsection*{Esempio di Cancellazione Numerica}

Si vuole approssimare la derivata di $f(x) = x^{10}$ in $x=1$ (valore esatto $f'(1)=10$) usando la formula:
$$ \frac{(1+\text{eps})^{10} - 1}{\text{eps}} $$
Al diminuire di \texttt{eps}, l'errore di troncamento si riduce, ma l'errore di round-off dovuto alla cancellazione numerica (sottrazione tra due numeri quasi uguali: $(1+\text{eps})^{10}$ e $1$) aumenta, fino a dominare e distruggere il risultato.

\begin{table}[h!]
\centering
\caption{Approssimazione della derivata di $f(x)=x^{10}$ in $x=1$}
\begin{tabular}{ll}
\toprule
\textbf{eps} & \textbf{Valore Calcolato: $((1+\text{eps})^{10}-1)/\text{eps}$} \\
\midrule
1.00e-01 & 15.937424601000023 \\
1.00e-02 & 10.462212541120453 \\
1.00e-03 & 10.045120210251168 \\
1.00e-04 & 10.004501200209237 \\
1.00e-05 & 10.000450012070949 \\
1.00e-06 & 10.000044999403102 \\
1.00e-07 & 10.000004506682814 \\
\midrule
\multicolumn{2}{c}{\textit{--- Inizia la cancellazione numerica ---}} \\
\midrule
1.00e-08 & 10.000000383314500 \\
1.00e-09 & 10.000000827403710 \\
1.00e-10 & 10.000000827403710 \\
1.00e-11 & 10.000000827403708 \\
1.00e-12 & 10.000889005823410 \\
1.00e-13 & 9.992007221626409 \\
1.00e-14 & 9.992007221626409 \\
1.00e-15 & 11.102230246251565 \\
1.00e-16 & 0.000000000000000 \\
\bottomrule
\end{tabular}
\end{table}

\section{Matlab (Matrix Laboratory)}
Matlab è un linguaggio interpretato orientato alla gestione efficiente delle matrici. Tutte le variabili sono definite in un'area di lavoro (workspace) e non è necessario dichiararne il tipo o la dimensione in anticipo.

\subsection{Gestione delle Variabili e della Workspace}
Il dimensionamento delle variabili è automatico e si adatta alle operazioni. Ad esempio, eseguendo `A(6,7) = 3.14`, Matlab crea una matrice `A` di dimensioni $6 \times 7$. Gli elementi non specificati vengono inizializzati a zero. La rappresentazione interna dei numeri è conforme allo standard IEEE 754 in doppia precisione.

Per ispezionare la workspace si usano i comandi:
\begin{itemize}
    \item \texttt{who}: mostra i nomi delle variabili presenti.
    \item \texttt{whos}: mostra una tabella dettagliata con nome, dimensione, tipo e memoria occupata.
\end{itemize}

È buona norma, per ragioni di efficienza, pre-allocare lo spazio per le matrici se le loro dimensioni finali sono note. Ad esempio:
\begin{lstlisting}
% Inizializza una matrice 200x300 con tutti zeri
A = zeros(200, 300);
\end{lstlisting}
Per sopprimere la visualizzazione del risultato di un'operazione, si termina la riga con un punto e virgola (\texttt{;}).

Per salvare o caricare la workspace si usano i comandi:
\begin{itemize}
    \item \texttt{save <nomefile>}: salva tutte le variabili nel file \texttt{nomefile.mat}.
    \item \texttt{load <nomefile>}: carica le variabili dal file specificato.
\end{itemize}

\paragraph{Formato di Visualizzazione}
Il comando \texttt{format} controlla come i valori numerici vengono visualizzati nella finestra di comando. È importante sottolineare che questo comando modifica solo la visualizzazione e non la precisione con cui i numeri sono memorizzati (che rimane in doppia precisione).

Alcune delle opzioni più comuni sono:
\begin{itemize}
    \item \texttt{format short}: formato a punto fisso con 5 cifre (default).
    \item \texttt{format long}: formato a punto fisso con 15 cifre.
    \item \texttt{format short e}: notazione scientifica con 5 cifre.
    \item \texttt{format long e}: notazione scientifica con 15 cifre.
\end{itemize}

\subsection{Creazione e Manipolazione di Matrici}
Uno dei comandi più utili in Matlab è \lstinline|help|. Questo comando permette di visualizzare la documentazione di una qualsiasi funzione direttamente nella finestra di comando. Ad esempio, per ottenere informazioni sulla funzione \lstinline|zeros|, si digita \lstinline|help zeros|.
\paragraph{Funzioni di base}
\begin{itemize}
    \item \texttt{zeros(m, n)}: crea una matrice $m \times n$ di zeri.
    \item \texttt{eye(n)}: crea una matrice identità $n \times n$.
    \item \texttt{rand(m, n)}: crea una matrice $m \times n$ con elementi casuali da una distribuzione uniformeme.
\end{itemize}

\paragraph{Operatore Colon (:)}
L'operatore \texttt{:} è estremamente versatile per creare vettori e selezionare sottomatrici.
\begin{lstlisting}
v1 = 1:5;       % Crea il vettore [1 2 3 4 5]
v2 = 5:-1:1;    % Crea il vettore [5 4 3 2 1]
v3 = 2:2:20;    % Crea il vettore dei numeri pari da 2 a 20
\end{lstlisting}

\paragraph{Indicizzazione e Concatenazione}
Si possono estrarre righe, colonne o sottomatrici e concatenare matrici esistenti.
\begin{lstlisting}
A = rand(3);
B = rand(3,1);
C = rand(2,4);

% Concatenazione orizzontale e verticale
D = [A B]; % Concatena A e B orizzontalmente (devono avere stesso numero di righe)
E = [A; rand(1,3)]; % Concatena A e una nuova riga verticalmente

% Estrazione
seconda_riga = D(2, :);
terza_colonna = D(:, 3);
sottomatrice = D(1:2, 1:3); % Estrae le prime due righe e le prime tre colonne
\end{lstlisting}

\subsection{Operatori ed Espressioni}
\begin{itemize}
    \item \textbf{Operatori Aritmetici}: \verb|+|, \verb|-|, \verb|*|, \verb|/|, \verb|\|
    \item \textbf{Operazioni con Scalari}: Le operazioni tra una matrice e uno scalare sono applicate a ogni elemento. Es: \texttt{C = A + 1;}, \texttt{C = 2 * A;}.
    \item \textbf{Operatori Element-wise}: Se preceduti da un punto (\texttt{.}), gli operatori agiscono elemento per elemento. Richiedono che le matrici abbiano le stesse dimensioni.
    \begin{itemize}
        \item \texttt{C = A .* B} $\implies C(i,j) = A(i,j) * B(i,j)$
        \item \texttt{C = A ./ B} $\implies C(i,j) = A(i,j) / B(i,j)$
    \end{itemize}
    \item \textbf{Divisione Matriciale}:
    \begin{itemize}
        \item \verb|C = A \ B| (backslash) risolve il sistema lineare $AX = B$, equivalente a $A^{-1}B$.
        \item \texttt{C = A / B} (slash) è equivalente a $AB^{-1}$.
    \end{itemize}
\end{itemize}

\subsection{Funzioni Utili}
Le funzioni in Matlab possono essere raggruppate in:
\begin{itemize}
    \item \textbf{Funzioni orientate a scalari}: operano su ogni elemento di una matrice. Es: \texttt{sin(A)}, \texttt{cos(A)}, \texttt{exp(A)}, \texttt{log(A)}, \texttt{sqrt(A)}. Anche le funzioni di arrotondamento come \texttt{round}, \texttt{floor}, \texttt{ceil}, \texttt{fix} rientrano in questa categoria.
    \item \textbf{Funzioni orientate a vettori}: se applicate a una matrice, operano colonna per colonna restituendo un vettore riga. Es: \texttt{max(A)}, \texttt{min(A)}, \texttt{mean(A)}, \texttt{sum(A)}, \texttt{sort(A)}.
    \item \textbf{Funzioni orientate a matrici}: operano sull'intera matrice. Es: \texttt{size(A)}, \texttt{det(A)}, \texttt{inv(A)}, \texttt{diag(A)}.
\end{itemize}

\subsection{Grafica 2D}
Matlab offre potenti strumenti per la visualizzazione grafica.
\begin{itemize}
    \item \texttt{plot(x, y)}: disegna la spezzata che congiunge i punti definiti dai vettori \texttt{x} e \texttt{y}.
    \item \texttt{semilogx}, \texttt{semilogy}, \texttt{loglog}: disegnano grafici con scale logaritmiche sugli assi.
    \item \texttt{xlabel('testo')}, \texttt{ylabel('testo')}, \texttt{title('titolo')}: aggiungono etichette agli assi e un titolo al grafico.
    \item \texttt{legend('curva1', 'curva2')}: aggiunge una legenda.
    \item \texttt{axis([xmin xmax ymin ymax])}: imposta i limiti degli assi.
    \item \texttt{figure}: apre una nuova finestra grafica.
\end{itemize}

\subsection{Controllo del Flusso e m-files}
Matlab fornisce costrutti per il controllo del flusso di esecuzione (selezioni e iterazioni) e permette di salvare sequenze di comandi in file con estensione \texttt{.m}, detti \textbf{m-files}.

\subsubsection{Selezioni}
Il costrutto di selezione principale è \texttt{if-elseif-else}.
\begin{lstlisting}[frame=none, numbers=none]
if <espressione_booleana_1>
    % blocco di istruzioni 1
elseif <espressione_booleana_2>
    % blocco di istruzioni 2
else
    % blocco di istruzioni 3
end
\end{lstlisting}
Le espressioni booleane si ottengono combinando operatori relazionali (\verb|>|, \verb|<|, \verb|>=|, \verb|<=|, \verb|==|, \verb|~=|) e operatori booleani (\verb|&&| per AND, \verb#||# per OR, \verb|~| per NOT). È buona norma usare le parentesi \verb|()| per definire chiaramente l'ordine di valutazione.

\subsubsection{Iterazioni (Cicli)}
I cicli principali sono \texttt{for} e \texttt{while}.
\paragraph{Il ciclo \texttt{for}} Esegue un blocco di istruzioni per ogni elemento di un vettore.
\begin{lstlisting}
% Esempio: calcola il quadrato dei numeri pari da 10 a 2
a = zeros(10,1);
for i = 10:-2:2
    a(i) = i^2;
end
\end{lstlisting}
Il comando \texttt{break} permette di uscire immediatamente dal ciclo che lo contiene.

\paragraph{Il ciclo \texttt{while}} Esegue un blocco di istruzioni finché una condizione booleana rimane vera.
\begin{lstlisting}[frame=none, numbers=none]
while <condizione_booleana>
    % blocco di istruzioni
    % (eventuale `break` per uscita forzata)
end
\end{lstlisting}

\subsubsection{m-files: Script e Function}
Esistono due tipi di m-files.
\begin{itemize}
    \item \textbf{Script-file}: È una semplice sequenza di comandi. Quando eseguito, opera direttamente sulle variabili presenti nella workspace (le sue variabili sono \textbf{globali}).
    \item \textbf{Function-file}: È un blocco di codice più strutturato che comunica con l'esterno solo tramite parametri di input e output. Tutte le variabili definite al suo interno sono \textbf{locali} e vengono distrutte al termine dell'esecuzione.
\end{itemize}

\paragraph{Anatomia di una Function}
Una function è definita dalla seguente intestazione:
\begin{lstlisting}[frame=none, numbers=none]
function [out1, out2, ...] = nomeFunzione(in1, in2, ...)
\end{lstlisting}
All'interno di una function si possono usare:
\begin{itemize}
    \item \texttt{nargin}: restituisce il numero di argomenti di input forniti alla chiamata.
    \item \texttt{nargout}: restituisce il numero di argomenti di output richiesti alla chiamata.
    \item \texttt{\%}: introduce un commento (il resto della riga viene ignorato).
    \item \texttt{...} (tre punti): permettono di spezzare una singola istruzione su più righe.
    \item \texttt{return}: termina l'esecuzione della function.
\end{itemize}

\paragraph{Help in linea}
Le prime righe di commento consecutive dopo l'intestazione di una function costituiscono il suo \textbf{help in linea}, che viene visualizzato quando si digita \texttt{help nomeFunzione}.

\paragraph{Esempio di Function}
Questo esempio definisce una function che calcola media e varianza di un vettore di dati, gestendo il numero di input e output.

\begin{lstlisting}
function [media, varianza] = esempio(dati)
% [media, varianza] = esempio(dati)
% Calcola la media e la varianza di un insieme di dati.
%
% Input:
%   dati - vettore con i dati di ingresso;
% Output:
%   media    - media dei dati;
%   varianza - varianza dei dati.

% Controllo degli input
if nargin < 1
    error('Dati di ingresso insufficienti');
else
    if isempty(dati), error('Il vettore di dati è vuoto'), end
end

media = mean(dati);

% Calcola la varianza solo se richiesta in output
if nargout > 1
    varianza = var(dati);
end

return
\end{lstlisting}

Digitando \texttt{help esempio} nella console di Matlab, si otterrà il testo scritto nelle prime righe di commento della funzione.


\section{Radici di una funzione}

Data una funzione $f: [a,b] \subseteq \mathbb{R} \to \mathbb{R}$, vogliamo determinare un valore $x^* \in [a,b]$ tale che:
\[
f(x^*) = 0
\]
Diremo che $x^*$ è una \textbf{radice} (o uno \textbf{zero}) della funzione $f(x)$.

In generale, una radice $x^*$ può:
\begin{itemize}
    \item Esistere ed essere unica.
    \item Esistere ma non essere unica (es. $f(x) = \sin(x)$).
    \item Non esistere (es. $f(x) = e^x$).
\end{itemize}

I metodi che esamineremo, se la radice esiste, forniranno un'approssimazione di una di esse.  
Una caratteristica comune a tutti questi metodi è di essere di tipo \textbf{iterativo}:  
a partire da un'approssimazione iniziale $x_0$, viene prodotta una successione di approssimazioni $\{x_n\}_{n \ge 0}$ che, se il metodo è convergente, tende alla radice $x^*$:
\[
\lim_{n \to \infty} x_n = x^*
\]

\subsection{Il metodo di bisezione}
Il primo metodo che analizziamo è il metodo di bisezione.  
Si basa su due assunzioni fondamentali:
\begin{enumerate}
    \item La funzione $f(x)$ è \textbf{continua} in un intervallo chiuso e limitato $[a,b]$.
    \item Agli estremi dell'intervallo, la funzione assume valori di segno opposto, ovvero $f(a) \cdot f(b) < 0$.
\end{enumerate}

\begin{figure}[ht!]
    \centering
    \begin{tikzpicture}[scale=1.5]
        % Assi cartesiani
        \draw[->] (-0.5,0) -- (4,0) node[below] {$x$};
        \draw[->] (0,-1.5) -- (0,1.5) node[left] {$f(x)$};
        
        % Disegna la funzione
        \draw[thick, color=green!50!black] plot[smooth, domain=0.5:3.5] (\x, {cos(\x*60)+0.1*\x-0.5});
        
        % Punti e etichette
        \draw (1,0.05) -- (1,-0.05) node[below] {$a$};
        \draw (3,0.05) -- (3,-0.05) node[below] {$b$};
        \node[below, color=red] at (2.05, -0.05) {$x^*$};
        \fill[red] (2.05,0) circle (1.5pt);
        
        % Linee tratteggiate e valori
        \draw[dashed] (1,0) -- (1, {cos(60)+0.1*1-0.5});
        \draw[dashed] (0, {cos(60)+0.1*1-0.5}) -- (1, {cos(60)+0.1*1-0.5});
        \node[left] at (0, {cos(60)+0.1*1-0.5}) {$f(a) > 0$};
        
        \draw[dashed] (3,0) -- (3, {cos(180)+0.1*3-0.5});
        \draw[dashed] (0, {cos(180)+0.1*3-0.5}) -- (3, {cos(180)+0.1*3-0.5});
        \node[left] at (0, {cos(180)+0.1*3-0.5}) {$f(b) < 0$};
    \end{tikzpicture}
    \caption{Illustrazione del teorema degli zeri.}
    \label{fig:teorema_zeri}
\end{figure}

In base al \textbf{teorema degli zeri}, queste ipotesi garantiscono l'esistenza di almeno una radice $x^*$ nell'intervallo $(a,b)$.

Non sapendo dove si trovi $x^*$ all'interno di $[a,b]$, la migliore stima iniziale che possiamo fare è il punto medio dell'intervallo:
\[
x_1 = \frac{a+b}{2}
\]
A questo punto, si possono verificare tre casi:
\begin{enumerate}
    \item $f(x_1) = 0$: abbiamo trovato la radice.
    \item $f(a) \cdot f(x_1) < 0$: la radice si trova nell'intervallo $[a, x_1]$.
    \item $f(x_1) \cdot f(b) < 0$: la radice si trova nell'intervallo $[x_1, b]$.
\end{enumerate}
Ad ogni iterazione, l'ampiezza dell'intervallo di confidenza viene dimezzata.

\subsection{Implementazione e criteri di arresto}
Un'implementazione ``naive''\footnote{Per "implementazione naive" si intende una versione semplice e diretta dell'algoritmo che ignora problemi pratici di efficienza o stabilità numerica.} del metodo potrebbe essere:

\begin{lstlisting}
fa = f(a);
fb = f(b);
while true
    x1 = (a+b)/2;
    f1 = f(x1);
    if f1 == 0
        break;
    elseif fa*f1 < 0
        b = x1;
        fb = f1;
    else
        a = x1;
        fa = f1;
    end
end
\end{lstlisting}

Questo criterio di arresto non è robusto: a causa dell'aritmetica finita, la condizione `f1 == 0` potrebbe non verificarsi mai, anche se $x_1$ è molto vicino alla radice, portando a un ciclo infinito.

\paragraph{Esempio: Errore di valutazione}
Consideriamo il polinomio $p(x)=(x-1.1)^{20}(x-\pi)$, che ha una radice in $x=\pi$.  
Valutandolo in Matlab:
\begin{lstlisting}
p_coeffs = poly([1.1*ones(1,20), pi]);
polyval(p_coeffs, pi)
% ans = -5.5213e-05
\end{lstlisting}
Il risultato non è zero, e l'algoritmo naive andrebbe in loop.

\paragraph{Criterio basato sul numero di iterazioni}
Possiamo calcolare a priori il numero massimo di iterazioni necessarie per raggiungere una data tolleranza.  
Se $x_i$ è l'approssimazione al passo $i$, l'errore assoluto è maggiorato dalla semi-ampiezza dell'intervallo:
\[
|x^* - x_i| \le \frac{b_i - a_i}{2} = \frac{b-a}{2^i}
\]
Per garantire un'accuratezza `tol`, imponiamo:
\[
\frac{b-a}{2^i} \le \text{tol}
\quad \Rightarrow \quad
i \ge \log_2\!\left(\frac{b-a}{\text{tol}}\right)
\]
Il numero massimo di iterazioni sarà quindi:
\[
i_{\max} = \left\lceil \log_2\!\left(\frac{b-a}{\text{tol}}\right) \right\rceil
\]
\footnote{Questa formula calcola il numero massimo di iterazioni necessarie per garantire un errore inferiore a `tol`. L'operatore $\lceil \cdot \rceil$ (ceiling) arrotonda all'intero superiore.}

\begin{lstlisting}
fa = f(a);
fb = f(b);
imax = ceil(log2(b-a) - log2(tol));

for i = 1:imax
    x1 = (a+b)/2;
    f1 = f(x1);
    
    if f1 == 0
        break;
    elseif fa*f1 < 0
        b = x1;
        fb = f1;
    else
        a = x1;
        fa = f1;
    end
end
\end{lstlisting}

\paragraph{Criterio basato sul residuo (controllo efficace)}
Un criterio di arresto più efficiente si basa sulla "piccolezza" del valore della funzione, $|f(x)|$.  
Lo sviluppo di Taylor di $f(x)$ attorno alla radice $x^*$ è:
\[
f(x) = f(x^*) + f'(x^*)(x-x^*) + O(|x-x^*|^2)
\]
Dato che $f(x^*)=0$, otteniamo:
\[
|x-x^*| \approx \frac{|f(x)|}{|f'(x^*)|}
\]
e imponendo $|x-x^*| \le \text{tol}$, si ha:
\[
|f(x)| \le \text{tol} \cdot |f'(x^*)|
\]
Poiché $f'(x^*)$ non è noto, lo approssimiamo come:
\[
f'(x^*) \approx \frac{f(b_i)-f(a_i)}{b_i - a_i}
\]

\paragraph{Algoritmo di bisezione ottimizzato}
\begin{lstlisting}
% Versione ottimizzata del metodo di bisezione
fa = f(a);
fb = f(b);
imax = ceil(log2(b-a) - log2(tol));

for i = 1:imax
    x1 = (a+b)/2;
    f1 = f(x1);
    
    % Stima della derivata e criterio di arresto sul residuo
    df = abs(fb-fa)/(b-a);
    if abs(f1) <= tol*df
        break;
    end
    
    % Aggiornamento dell'intervallo
    if fa*f1 < 0
        b = x1; fb = f1;
    else
        a = x1; fa = f1;
    end
end
% x1 contiene la radice approssimata
\end{lstlisting}


\subsection{Ordine di Convergenza}
Abbiamo esaminato il metodo di bisezione, definito sotto le ipotesi $f \in C([a,b])$ e $f(a)f(b)<0$. L'errore $e_i = x^* - x_i$ al passo $i$ soddisfa:
$$ |e_i| \le \epsilon_i = \frac{b-a}{2^i} $$
con $\frac{\epsilon_{i+1}}{\epsilon_i} = \frac{1}{2}$, il che implica $\lim_{i \to \infty} \frac{\epsilon_{i+1}}{\epsilon_i} = \frac{1}{2}$.

\begin{definition}[Ordine di Convergenza]
Un generico metodo iterativo si dice \textbf{convergente} se $\lim_{i \to \infty} e_i = 0$.
Si dice che ha \textbf{ordine di convergenza $p$} se $p$ è il più grande numero reale positivo tale che esista una costante $C > 0$ (costante asintotica dell'errore) per cui:
$$ \lim_{i \to \infty} \frac{|e_{i+1}|}{|e_i|^p} = C < \infty $$
\end{definition}

Per $i$ sufficientemente grande, vale $|e_{i+1}| \approx C |e_i|^p$.
\begin{itemize}
    \item Affinché ci sia convergenza, deve aversi $p \ge 1$.
    \item Se $p=1$, la convergenza è \textbf{lineare}. In questo caso, per $i$ grande, abbiamo:
          \begin{align*}
              |e_{i+1}| &\approx C |e_i| \\
              |e_{i+2}| &\approx C |e_{i+1}| \approx C^2 |e_i| \\
              &\vdots \\
              |e_{i+k}| &\approx C^k |e_i| 
          \end{align*}
          Questa successione tende a zero per $k \to \infty$ se e solo se $C < 1$. Pertanto, un metodo di ordine 1 è convergente se e solo se la sua costante asintotica dell'errore è minore di 1.
    \item Se $p=2$, la convergenza è \textbf{quadratica}.
    
\end{itemize}

\begin{osservazione}
Il metodo di bisezione, se applicabile, è sempre convergente, ha ordine $p=1$ e costante asintotica $C=1/2$.
\end{osservazione}

Un ordine $p$ più elevato implica una convergenza più rapida. Confrontiamo due metodi con $C=0.1$ e $|e_0|=0.1$:

\begin{table}[ht!]
\centering
\caption{Confronto tra convergenza lineare e quadratica ($|e_{i+1}| \approx C|e_i|^p$)}
\begin{tabular}{ccc}
\toprule
$i$ & Errore ($p=1$) & Errore ($p=2$) \\
\midrule
0 & $10^{-1}$ & $10^{-1}$ \\
1 & $10^{-2}$ & $10^{-3}$ \\
2 & $10^{-3}$ & $10^{-7}$ \\
3 & $10^{-4}$ & $10^{-15}$ \\
\bottomrule
\end{tabular}
\end{table}
È evidente che è bene ricercare metodi di ordine più elevato.

\paragraph{Condizionamento del Problema}
Vogliamo determinare $x^*$ tale che $f(x^*) = 0$. Se invece abbiamo una soluzione perturbata $\tilde{x}$ tale che $f(\tilde{x}) \neq 0$, studiamo come la perturbazione sul valore $f(\tilde{x})$ influenza l'errore $|\tilde{x} - x^*|$.
Usando Taylor: $f(\tilde{x}) \approx f'(x^*)(\tilde{x} - x^*)$, da cui:
$$ |\tilde{x} - x^*| \approx \frac{|f(\tilde{x})|}{|f'(x^*)|} $$
Il fattore $K = \frac{1}{|f'(x^*)|}$ è il \textbf{numero di condizione} del problema.
\begin{itemize}
    \item Se $|f'(x^*)| \ >> 1$ (non vicino a zero), il problema è \textbf{ben condizionato}.
    \item Se $|f'(x^*)| \approx 0$, il problema è \textbf{mal condizionato}.
\end{itemize}

\begin{definition}[Molteplicità di una radice]
Una radice $x^*$ ha \textbf{molteplicità $m \ge 1$} se $f(x^*) = f'(x^*) = \dots = f^{(m-1)}(x^*) = 0$ e $f^{(m)}(x^*) \neq 0$.
Se $m=1$, la radice è \textbf{semplice}. Se $m>1$, la radice è \textbf{multipla}.
\end{definition}

\begin{osservazione}
L'approssimazione di una radice multipla dà origine ad un problema \textbf{sempre mal condizionato}.
\end{osservazione}

\subsection{Metodo di Newton}
Il metodo di Newton è un metodo iterativo che, a partire da un'approssimazione $x_0$, costruisce la successione $\{x_i\}$ utilizzando l'interpretazione geometrica della derivata.

L'equazione della retta tangente al grafico di $f(x)$ nel punto $(x_0, f(x_0))$ è data da $y - f(x_0) = f'(x_0)(x - x_0)$. La nuova approssimazione $x_1$ si trova intersecando questa retta con l'asse delle $x$ (retta di equazione $y=0$). Dobbiamo quindi risolvere il sistema:
\[
\begin{cases} 
y - f(x_0) = f'(x_0)(x - x_0) & \leftarrow \text{retta tangente} \\
y = 0 & \leftarrow \text{asse x}
\end{cases}
\]
Sostituendo $y=0$ nella prima equazione e chiamando la soluzione $x_1$, otteniamo:
$$ -f(x_0) = f'(x_0)(x_1 - x_0) $$
Da cui si ricava la nuova approssimazione:
$$ x_1 = x_0 - \frac{f(x_0)}{f'(x_0)} $$
In generale, iterando questo procedimento, si ottiene la formula del metodo di Newton:
$$ x_{i+1} = x_i - \frac{f(x_i)}{f'(x_i)}, \quad i=0, 1, \dots $$

% --- INSERISCI QUI IL GRAFICO ---
\begin{figure}[ht!] % Ho cambiato [h!] in [ht!] come suggerito dal warning
\centering
\begin{tikzpicture}[scale=1.2]
    % Assi
    \draw[->] (-0.5,0) -- (4.5,0) node[below] {$x$};
    \draw[->] (0,-0.5) -- (0,3) node[left] {$f(x)$};
    
    % Curva f(x)
    \draw[thick, color=red!80!black] plot[smooth, domain=0:4] (\x, {0.2*(\x-1)^2 + 0.1*\x});
    \node[above, color=red!80!black] at (3.5, 2) {$f(x)$};
    
    % Punto x0 e tangente
    \coordinate (x0) at (3.5, {0.2*(3.5-1)^2 + 0.1*3.5});
    \fill (x0) circle (1.5pt);
    \node[below right] at (x0) {$(x_0, f(x_0))$};
    \draw (3.5,-0.1) node[below] {$x_0$};
    
    % Calcolo della derivata in x0 (pendenza)
    % f'(x) = 0.4*(x-1) + 0.1 -> f'(3.5) = 0.4*2.5 + 0.1 = 1.1
    % Retta: y - f(x0) = 1.1 * (x - x0)
    \draw[color=green!60!black] (x0) -- +(-1.5, -1.5*1.1) coordinate (endtan0); % Disegna un pezzo della tangente
    \draw[color=green!60!black] (x0) -- +(0.5, 0.5*1.1); % Altro pezzo
    
    % Intersezione x1
    % y=0 => -f(x0) = 1.1 * (x1 - x0) => x1 = x0 - f(x0)/1.1
    % f(3.5) = 0.2*2.5^2 + 0.1*3.5 = 0.2*6.25 + 0.35 = 1.25 + 0.35 = 1.6
    % x1 = 3.5 - 1.6/1.1 = 3.5 - 1.4545... = 2.045...
    \coordinate (x1_inter) at (3.5 - 1.6/1.1, 0);
    \draw[dashed, color=green!60!black] (x0) -- (x1_inter);
    \draw (x1_inter)+(0,-0.1) node[below] {$x_1$};
    \fill (x1_inter) circle (1.5pt);

    % Punto x1 sulla curva e tangente
    \coordinate (x1) at (3.5 - 1.6/1.1, {0.2*((3.5 - 1.6/1.1)-1)^2 + 0.1*(3.5 - 1.6/1.1)});
    \fill (x1) circle (1.5pt);
    % f'(x1) = 0.4*(x1-1) + 0.1 = 0.4*(1.045...) + 0.1 = 0.418 + 0.1 = 0.518
    \draw[color=green!60!black] (x1) -- +(-1.0, -1.0*0.518);
    \draw[color=green!60!black] (x1) -- +(0.5, 0.5*0.518);
    
    % Intersezione x* (radice) - approssimata
    \coordinate (x_star) at (1,0); % f(1)=0.1, ma facciamo finta che sia la radice per semplicità
    \draw (x_star)+(0,-0.1) node[below] {$x^*$};
    \fill[red] (x_star) circle (1.5pt);

\end{tikzpicture}
\caption{Interpretazione geometrica del metodo di Newton.}
\label{fig:newton}
\end{figure}
% --- FINE GRAFICO ---

\paragraph{Considerazioni sul Metodo di Newton}
\begin{enumerate}
    \item È richiesta la derivabilità di $f(x)$ (almeno $f \in C^2$ in un intorno della radice per l'analisi).
    \item Il costo per iterazione è 1 valutazione di $f(x)$ e 1 valutazione di $f'(x)$.
\end{enumerate}

\begin{teorema}[Convergenza di Newton]
    Sia $f(x) \in C^{(2)}$ in un intorno della radice $x^*$. Supponiamo che il metodo di Newton converga a $x^*$, e che $x^*$ sia semplice. Allora l'ordine di convergenza è (almeno) 2.
\end{teorema}
\begin{proof}[Dimostrazione (schizzo)]
Sviluppando $f(x^*)$ in Taylor attorno a $x_i$:
$$ 0 = f(x^*) = f(x_i) + f'(x_i)(x^* - x_i) + \frac{1}{2} f''(\xi_i)(x^* - x_i)^2, \quad \xi_i \in (x_i, x^*) $$
Ponendo $e_i = x^* - x_i$ e $e_{i+1} = x^* - x_{i+1}$:
$$ 0 = f(x_i) + f'(x_i)e_i + \frac{1}{2} f''(\xi_i)e_i^2 $$
Dividendo per $f'(x_i)$ e usando $x_{i+1} = x_i - f(x_i)/f'(x_i)$:
$$ 0 = -(x_{i+1} - x_i) + e_i + \frac{1}{2} \frac{f''(\xi_i)}{f'(x_i)} e_i^2 $$
$$ 0 = -(x_{i+1} - x_i) + (x^* - x_i) + \frac{1}{2} \frac{f''(\xi_i)}{f'(x_i)} e_i^2 = (x^* - x_{i+1}) + \frac{1}{2} \frac{f''(\xi_i)}{f'(x_i)} e_i^2 $$
$$ e_{i+1} = - \frac{1}{2} \frac{f''(\xi_i)}{f'(x_i)} e_i^2 $$
Passando al limite per $i \to \infty$:
$$ \lim_{i \to \infty} \frac{|e_{i+1}|}{|e_i|^2} =  \frac{1}{2} \left|\frac{f''(x^*)}{f'(x^*)} \right| = C $$
Questo dimostra la convergenza quadratica ($p=2$).
\end{proof}

\begin{osservazione}[Radici multiple]
    Nel caso di una radice multipla, con molteplicità $m > 1$, si può dimostrare che:
    $$ \lim_{i \to \infty} \frac{|e_{i+1}|}{|e_i|} = \frac{m-1}{m} $$
    ovvero, l'ordine di convergenza del metodo di Newton diventa lineare ($p=1$), come conseguenza del mal condizionamento del problema.
\end{osservazione}

\subsection*{Riepilogo dei Metodi Visti}
\paragraph{Metodo di Bisezione}
\begin{itemize}
    \item Applicabile se $f \in C([a,b])$ e $f(a)f(b) < 0$.
    \item Ordine di convergenza lineare ($p=1$), con costante asintotica $C=1/2$.
    \item Il numero massimo di iterazioni per raggiungere una data accuratezza (\texttt{tol}) è noto a priori: $\text{imax} = \lceil \log_2(b-a) - \log_2(\text{tol}) \rceil$.
\end{itemize}

\paragraph{Metodo di Newton}
Formula iterativa: $x_{i+1} = x_i - \frac{f(x_i)}{f'(x_i)}$.
\begin{itemize}
    \item Richiede che $f(x)$ sia derivabile.
    \item Se $f \in C^2$ in un intorno di una radice semplice, allora si ha, se convergente alla radice, convergenza \textbf{quadratica} ($p=2$).
    \item Se converge a una radice \textbf{multipla} $x^*$ (molteplicità $m>1$), la convergenza è solo \textbf{lineare} ($p=1$) con $C = \frac{m-1}{m}$, riflettendo il mal condizionamento del problema.
\end{itemize}

\subsection{Convergenza Locale vs Globale}
A differenza del metodo di bisezione, per il metodo di Newton non è in generale possibile garantire la convergenza a partire da un punto iniziale $x_0$ qualsiasi.

\begin{esempio}[Non convergenza di Newton]
Consideriamo $f(x) = x^3 - 5x = x(x^2-5)$, che ha tre radici semplici: $0, \pm\sqrt{5}$. Applichiamo il metodo di Newton partendo da $x_0=1$.
$f'(x) = 3x^2 - 5$.
$$ x_1 = x_0 - \frac{f(x_0)}{f'(x_0)} = 1 - \frac{1^3 - 5(1)}{3(1)^2 - 5} = 1 - \frac{-4}{-2} = 1 - 2 = -1 $$
$$ x_2 = x_1 - \frac{f(x_1)}{f'(x_1)} = -1 - \frac{(-1)^3 - 5(-1)}{3(-1)^2 - 5} = -1 - \frac{-1 + 5}{3 - 5} = -1 - \frac{4}{-2} = -1 + 2 = 1 $$
La successione generata è $\{1, -1, 1, -1, \dots\}$, che evidentemente non converge.
\end{esempio}

% --- GRAFICO ESEMPIO NEWTON ---
\begin{figure}[H] % Usiamo [H] dal pacchetto float per forzare la posizione
\centering
\begin{tikzpicture}[scale=1]
    % Assi
    \draw[->] (-2.7,0) -- (2.7,0) node[below] {$x$};
    \draw[->] (0,-5) -- (0,5) node[left] {$y$};
    \node[above right] at (1.5,3) {$f(x) = x^3 - 5x$};
    
    % Curva f(x)
    \draw[thick, color=teal, domain=-2.5:2.5, samples=100] plot (\x, {\x*\x*\x - 5*\x});
    
    % Punti x0=1 e x1=-1
    \coordinate (p1) at (1, -4);  % (1, f(1))
    \coordinate (m1) at (-1, 4);  % (-1, f(-1))
    
    % Cerchietti sui punti
    \draw[red] (p1) circle (3pt);
    \draw[red] (m1) circle (3pt);
    
    % Linee tratteggiate verticali
    \draw[dashed] (1,0) node[below] {$1$} -- (p1);
    \draw[dashed] (-1,0) node[below] {$-1$} -- (m1);
    
    % Tangenti (visualizzate come segmenti tra i punti)
    \draw[thick, color=magenta] (p1) -- (m1); 
    
\end{tikzpicture}
\caption{Metodo di Newton per $f(x)=x^3-5x$ partendo da $x_0=1$. Gli iterati oscillano tra 1 e -1.}
\label{fig:newton_oscillante}
\end{figure}
% --- FINE GRAFICO ---

La conclusione è che la convergenza del metodo di Newton è garantita solo in un opportuno \textbf{intorno} della radice. Si parla in questo caso di \textbf{convergenza locale}. Il metodo di bisezione, invece, ha \textbf{convergenza globale} (se applicabile, converge sempre).

\subsection{Teoria del Punto Fisso e Convergenza Locale}
Formalizziamo questo concetto per un generico metodo iterativo:
\begin{equation}
    x_{i+1} = \Phi(x_i), \quad i=0, 1, \dots
\end{equation}
dove $\Phi(x)$ è detta \textbf{funzione di iterazione}. Ad esempio, per Newton, $\Phi(x) = x - \frac{f(x)}{f'(x)}$.

Se il metodo serve a determinare la radice $x^*$ di $f(x)$, allora $x^*$ deve soddisfare la \textbf{proprietà di consistenza}:
\begin{equation}
    x^* = \Phi(x^*)
\end{equation}
ovvero, $x^*$ deve essere un \textbf{punto fisso} della funzione di iterazione $\Phi(x)$. Il problema di trovare uno zero di $f(x)$ è quindi equivalente a trovare un punto fisso di $\Phi(x)$.

Vogliamo capire sotto quali condizioni, partendo da un intorno del suo punto fisso, la successione (4) converge a $x^*$.

\begin{teorema}[del Punto Fisso / delle Contrazioni]
Sia $x^*$ un punto fisso di $\Phi(x)$. Se esiste $\delta > 0$ tale che per ogni $x, y \in I(x^*) = [x^*-\delta, x^*+\delta]$ la funzione $\Phi(x)$ è \textbf{Lipschitziana} con costante $L < 1$, cioè:
$$ |\Phi(x) - \Phi(y)| \le L |x - y| $$
allora:
\leavevmode % Aggiunto per risolvere potenziali conflitti
\begin{enumerate}
    \item Se $x_0 \in I(x^*)$, tutti i successivi iterati $x_i$ rimangono in $I(x^*)$ ($x_i \in I(x^*)$ per ogni $i \ge 0$).
    \item La successione $\{x_i\}$ converge a $x^*$.
\end{enumerate}
\end{teorema}
\begin{proof}
Dimostriamo per induzione che $x_i \in I(x^*)$. È vero per $i=0$. Supponiamo $x_i \in I(x^*)$. Allora:
$$ |x^* - x_{i+1}| = |\Phi(x^*) - \Phi(x_i)| \le L |x^* - x_i| $$
Poiché $L<1$ e $|x^* - x_i| \le \delta$, segue che $|x^* - x_{i+1}| < \delta$, quindi $x_{i+1} \in I(x^*)$.
Inoltre, applicando ricorsivamente la disuguaglianza:
$$ |x^* - x_i| \le L |x^* - x_{i-1}| \le L^2 |x^* - x_{i-2}| \le \dots \le L^i |x^* - x_0| $$
Dato che $L<1$, $L^i \to 0$ per $i \to \infty$, quindi $\lim_{i \to \infty} |x^* - x_i| = 0$.
\end{proof}

\begin{corollario}
    Se $\exists \delta > 0$ tale che $\forall x \in [x^*-\delta, x^*+\delta] = I(x^*)$ si ha $|\Phi'(x)| \le L < 1$, allora la successione $x_{i+1} = \Phi(x_i)$ converge a $x^*$ per $i \to \infty$.
\end{corollario}
\begin{proof}
Dal teorema del valor medio, per $x, y \in I(x^*)$, esiste $\xi$ tra $x$ e $y$ tale che:
$$ |\Phi(x) - \Phi(y)| = |\Phi'(\xi)(x-y)| = |\Phi'(\xi)| |x-y| \le L |x-y| $$
Quindi $\Phi(x)$ è Lipschitziana con costante $L<1$ e si applica il teorema precedente.
\end{proof}

\paragraph{Applicazione al Metodo di Newton}
Vediamo come si applica il risultato precedente al metodo di Newton. La funzione di iterazione è:
$$ \Phi(x) = x - \frac{f(x)}{f'(x)} $$
Verifichiamo la consistenza nel punto fisso $x^*$:
$$ \Phi(x^*) = x^* - \frac{f(x^*)}{f'(x^*)} = x^* - \frac{0}{f'(x^*)} = x^* $$
Calcoliamo ora la derivata prima di $\Phi(x)$:
$$ \Phi'(x) = \frac{d}{dx}\left(x - \frac{f(x)}{f'(x)}\right) = 1 - \frac{f'(x)f'(x) - f(x)f''(x)}{[f'(x)]^2} = \frac{[f'(x)]^2 - [f'(x)]^2 + f(x)f''(x)}{[f'(x)]^2} = \frac{f(x)f''(x)}{[f'(x)]^2} $$
Valutiamo la derivata nel punto fisso $x^*$:
\begin{itemize}
    \item \textbf{Se $x^*$ è una radice semplice}: In questo caso $f(x^*) = 0$ ma $f'(x^*) \neq 0$. Quindi:
    $$ \Phi'(x^*) = \frac{f(x^*)f''(x^*)}{[f'(x^*)]^2} = \frac{0 \cdot f''(x^*)}{[f'(x^*)]^2} = 0 $$
    Poiché $\Phi'(x^*) = 0 < 1$, per la continuità di $\Phi'(x)$, esisterà un intorno $I(x^*)$ tale che $|\Phi'(x)| \le L < 1$ per $x \in I(x^*)$. Di conseguenza, per il corollario precedente, il metodo di Newton converge localmente. Il fatto che $\Phi'(x^*) = 0$ è condizione sufficiente per la convergenza quadratica.
    
    % --- GRAFICO DERIVATA PHI ---
    \begin{figure}[H]
    \centering
    \begin{tikzpicture}[scale=1]
        % Assi
        \draw[->] (-2.5,0) -- (2.5,0) node[below] {$x$};
        \draw[->] (0,-1.5) -- (0,1.5) node[left] {}; % Asse y senza etichetta
        \node[above right] at (1.5,1) {$\Phi'(x)$};
        
        % Curva Phi'(x) passante per x* con derivata 0
        \draw[thick, domain=-2:2, samples=50] plot (\x, {0.5*(\x-0.5)}); % Curva generica con Phi'(x*)=0
        
        % Punto fisso x*
        \coordinate (xstar) at (0.5, 0);
        \fill[red] (xstar) circle (1.5pt);
        \node[below, red] at (xstar) {$x^*$};
        
        % Linee per L e -L (con L < 1)
        \draw[dashed, color=green!60!black] (-2,1) node[left] {$L$} -- (2,1);
        \draw[dashed, color=green!60!black] (-2,-1) node[left] {$-L$} -- (2,-1);
        
        % Intervallo I(x*) dove |Phi'(x)| < L
        % Trova i punti x tali che Phi'(x) = L e Phi'(x) = -L
        % 0.5*(x-0.5) = 1 => x-0.5=2 => x=2.5
        % 0.5*(x-0.5) = -1 => x-0.5=-2 => x=-1.5
        \draw[green!60!black] (-1.5, -1) -- (-1.5, {0.5*(-1.5-0.5)}); % Linea verticale sinistra
        \draw[green!60!black] (2.5, 1) -- (2.5, {0.5*(2.5-0.5)});    % Linea verticale destra (fuori asse)
                
        % Indicazione dell'intervallo sull'asse x
        \draw[|<->|,thick, green!60!black] (-1.5, -1.2) -- (2.5, -1.2) node[midway, below] {$I(x^*)$}; 
        
    \end{tikzpicture}
    \caption{Intorno $I(x^*)$ di una radice semplice dove $|\Phi'(x)| \le L < 1$, garantendo la convergenza locale.}
    \label{fig:phi_derivata_newton}
    \end{figure}
    % --- FINE GRAFICO ---

    \item \textbf{Se $x^*$ è una radice multipla} di molteplicità $m > 1$: Si può dimostrare che:
    $$ \Phi'(x^*) = \frac{m-1}{m} $$
    Anche in questo caso $|\Phi'(x^*)| < 1$, quindi il metodo converge ancora localmente (per il corollario). Tuttavia, poiché $\Phi'(x^*) \neq 0$, la convergenza diventa meno favorevole.
\end{itemize}
\subsection{Criteri di Arresto}
Per un metodo iterativo $x_{i+1} = \Phi(x_i)$, cerchiamo un criterio per fermare l'iterazione.

\paragraph{Criterio basato sul residuo}
Come visto per la bisezione, potremmo richiedere $|f(x_i)| \le \text{tol} \cdot |f'(x_i)|$ (approssimando $f'(x^*)$ con $f'(x_i)$). Dalla formula di Newton, questo è equivalente a:
$$ \left| \frac{f(x_i)}{f'(x_i)} \right| \le \text{tol} \implies |x_{i+1} - x_i| \le \text{tol} $$

\paragraph{Criterio basato sulla differenza tra iterati}
Il criterio $|x_{i+1} - x_i| \le \text{tol}$ controlla l'errore assoluto. Tuttavia, se $|x^*| \gg 1$, sarebbe più significativo controllare l'errore relativo:
$$ \frac{|x^* - x_i|}{|x^*|} \le \text{tol} $$
che può essere approssimato da:
$$ \frac{|x_{i+1} - x_i|}{|x_i|} \le \text{tol} \quad (\text{o } |x_{i+1}| \text{ al denominatore}) $$

\paragraph{Criterio combinato (auto-scalante)}
Un criterio pratico che combina i due è:
$$ |x_{i+1} - x_i| \le (1 + |x_i|) \cdot \text{tol} \quad \text{o equivalentemente} \quad \frac{|x_{i+1} - x_i|}{1 + |x_i|} \le \text{tol} $$
Questo criterio "auto-scalante" si comporta come un controllo sull'errore assoluto quando $|x_i| \approx 0$ e come un controllo sull'errore relativo quando $|x_i| \gg 1$, adattandosi alla grandezza della radice a cui si sta convergendo. \textbf{N.B.}: Da usare nei nostri elaborati.

\subsection{Metodi Quasi-Newton}
Ricordiamo che la derivata prima $f'(x_i)$ può essere approssimata dal rapporto incrementale:
$$ f'(x_i) \approx \frac{f(x_i+h) - f(x_i)}{h} $$
Se usassimo questa approssimazione con un $h$ fisso all'interno della formula di Newton, otterremmo una variante chiamata metodo \textbf{quasi-Newton}. Questo specifico approccio richiederebbe 2 valutazioni di funzione per iterazione ($f(x_i)$ e $f(x_i+h)$).

Per migliorare l'efficienza, possiamo usare le informazioni già calcolate nelle iterazioni precedenti.

\paragraph{Metodo delle Secanti}
Consideriamo la retta \emph{secante} passante per i punti $(x_{i-1}, f(x_{i-1}))$ e $(x_i, f(x_i))$. Il suo coefficiente angolare è:
$$ \frac{f(x_i) - f(x_{i-1})}{x_i - x_{i-1}} $$
Usiamo questo valore come approssimazione di $f'(x_i)$ nella formula di Newton. L'iterazione del \textbf{metodo delle secanti} diventa:
$$ x_{i+1} = x_i - f(x_i) \frac{x_i - x_{i-1}}{f(x_i) - f(x_{i-1})}, \quad i=1, 2, \dots $$

% --- GRAFICO METODO SECANTI ---
\begin{figure}[H]
\centering
\begin{tikzpicture}[scale=1.2]
    % Assi
    \draw[->] (-0.5,0) -- (4.5,0) node[below] {$x$};
    \draw[->] (0,-0.5) -- (0,3) node[left] {};
    \node[above right] at (3.5, 2.5) {$f(x)$};
    
    % Curva f(x)
    \draw[thick, color=green!60!black, domain=0:4.2, samples=50] plot (\x, {0.1*exp(0.8*\x)});
    
    % Radice
    \coordinate (xstar) at (0,0); % Punto fittizio
    \node[below, red] at (-0.2, -0.1) {$x^*$};
    
    % Punti xi, x(i-1) ...
    \coordinate (x0) at (4, {0.1*exp(0.8*4)});
    \coordinate (x1) at (3, {0.1*exp(0.8*3)});
    \coordinate (x2) at (2, {0.1*exp(0.8*2)});
    \coordinate (x3) at (1, {0.1*exp(0.8*1)});
    
    \fill (x0) circle (1.5pt); \draw (4,-0.1) node[below] {$x_0$};
    \fill (x1) circle (1.5pt); \draw (3,-0.1) node[below] {$x_1$};
    \fill (x2) circle (1.5pt); \draw (2,-0.1) node[below] {$x_2$};
    \fill (x3) circle (1.5pt); \draw (1,-0.1) node[below] {$x_3$};
    
    % Rette secanti
    % Secante tra x0 e x1 per trovare x2 (intersezione asse x)
    \draw[black] (x0) -- (x1); 
    % Interpolazione lineare per x2
    \coordinate (x2_inter) at (2.19, 0); % Valore approssimato
    \draw[dashed] (x1) -- (x2_inter);
    
    % Secante tra x1 e x2 per trovare x3
    \draw[black] (x1) -- (x2);
    \coordinate (x3_inter) at (1.1, 0); % Valore approssimato
    \draw[dashed] (x2) -- (x3_inter);

    % Secante tra x2 e x3 per trovare x4
    \draw[black] (x2) -- (x3);
    \coordinate (x4_inter) at (0.2, 0); % Valore approssimato
    \draw[dashed] (x3) -- (x4_inter);
    
\end{tikzpicture}
\caption{Interpretazione geometrica del metodo delle secanti.}
\label{fig:secanti}
\end{figure}
% --- FINE GRAFICO ---

\begin{osservazione}[Metodo delle Secanti]
\begin{enumerate}
    \item Richiede due approssimazioni iniziali ($x_0, x_1$) per iniziare (è un metodo a due passi).
    \item Il costo per iterazione (dopo la prima) è di \textbf{1 sola valutazione} di $f(x)$.
    \item L'ordine di convergenza verso radici \textbf{semplici} è $p = \frac{1+\sqrt{5}}{2} \approx 1.618$ (la sezione aurea), che è superlineare ma inferiore a quello di Newton.
    \item La convergenza verso radici \textbf{multiple} è solo lineare ($p=1$).
    \item Essendo un'approssimazione di Newton, la convergenza è generalmente \textbf{locale}.
\end{enumerate}
\end{osservazione}

\paragraph{Metodo delle Corde}
Un'altra approssimazione di $f'(x_i)$ consiste nell'usare un valore fisso, calcolato solo all'inizio, tipicamente $f'(x_0)$. Si assume che la derivata non vari molto vicino alla radice.
L'iterazione del \textbf{metodo delle corde} è:
$$ x_{i+1} = x_i - \frac{f(x_i)}{f'(x_0)}, \quad i=0, 1, \dots $$

% --- GRAFICO METODO CORDE ---
\begin{figure}[H]
\centering
\begin{tikzpicture}[scale=1.2]
    % Assi
    \draw[->] (-0.5,0) -- (4.5,0) node[below] {$x$};
    \draw[->] (0,-0.5) -- (0,3) node[left] {};
    \node[above right] at (3.5, 2.5) {$f(x)$};
    
    % Curva f(x)
    \draw[thick, color=green!60!black, domain=0:4.2, samples=50] plot (\x, {0.1*exp(0.8*\x)});
    
    % Radice
    \coordinate (xstar) at (0,0); % Punto fittizio
    \node[below, red] at (-0.2, -0.1) {$x^*$};
    
    % Punti x0, x1, x2, x3
    \coordinate (x0) at (4, {0.1*exp(0.8*4)}); \draw (4,-0.1) node[below] {$x_0$}; \fill (x0) circle (1.5pt);
    % Calcolo pendenza in x0: f'(x) = 0.08*exp(0.8*x) => f'(4) = 0.08*exp(3.2) = 1.96
    % x1 = x0 - f(x0)/f'(x0) = 4 - (0.1*exp(3.2))/1.96 = 4 - 2.45 / 1.96 = 4 - 1.25 = 2.75 (circa)
    \coordinate (x1_inter) at (2.75, 0); \draw (2.75,-0.1) node[below] {$x_1$}; \fill (x1_inter) circle (1.5pt);
    \coordinate (x1) at (2.75, {0.1*exp(0.8*2.75)}); \fill (x1) circle (1.5pt);
    % x2 = x1 - f(x1)/f'(x0) = 2.75 - (0.1*exp(2.2))/1.96 = 2.75 - 0.90 / 1.96 = 2.75 - 0.46 = 2.29 (circa)
    \coordinate (x2_inter) at (2.29, 0); \draw (2.29,-0.1) node[below] {$x_2$}; \fill (x2_inter) circle (1.5pt);
    \coordinate (x2) at (2.29, {0.1*exp(0.8*2.29)}); \fill (x2) circle (1.5pt);
    % x3 = x2 - f(x2)/f'(x0) = 2.29 - (0.1*exp(1.83))/1.96 = 2.29 - 0.62 / 1.96 = 2.29 - 0.32 = 1.97 (circa)
    \coordinate (x3_inter) at (1.97, 0); \draw (1.97,-0.1) node[below] {$x_3$}; \fill (x3_inter) circle (1.5pt);
    
    % Rette parallele con pendenza f'(x0)
    \draw[dashed, black] (x0) -- (x1_inter);
    \draw[dashed, black] (x1) -- (x2_inter);
    \draw[dashed, black] (x2) -- (x3_inter);
    
\end{tikzpicture}
\caption{Interpretazione geometrica del metodo delle corde (le rette sono parallele).}
\label{fig:corde}
\end{figure}
% --- FINE GRAFICO ---

\begin{osservazione}[Metodo delle Corde]
\begin{enumerate}
    \item Richiede il calcolo di $f'(x_0)$ solo una volta all'inizio.
    \item Il costo per iterazione è di \textbf{1 sola valutazione} di $f(x)$.
    \item La convergenza è generalmente \textbf{locale} e l'ordine è \textbf{lineare} ($p=1$). È spesso usato quando si dispone già di una buona approssimazione iniziale $x_0$.
\end{enumerate}
\end{osservazione}

\begin{esempio}[Confronto Metodi per $f(x) = x - \cos(x)$]
La tabella mostra le prime iterazioni per trovare la radice di $f(x) = x - \cos(x)$ (vicina a 0.739) partendo da $x_0=1$ (per le secanti, $x_1$ è calcolato con un passo di Newton).

\begin{table}[H]
\centering
\caption{Iterazioni per $f(x)=x-\cos(x)$, $x_0=1$}
\begin{tabular}{cccc}
\toprule
Iter (i) & Newton & Secanti & Corde \\
\midrule
0 & 0 & 0 & 0 \\
1 & 1.000000e+00 & 1.000000e+00 & 1.000000e+00 \\
2 & 7.503639e-01 & 7.503639e-01 & 5.403023e-01 \\ % x1_sec = x1_newton
3 & 7.391129e-01 & 7.374864e-01 & 8.575532e-01 \\ 
4 & 7.390851e-01 & 7.390986e-01 & 6.542898e-01 \\
5 & 7.390851e-01 & 7.390851e-01 & 7.934804e-01 \\
6 & 7.390851e-01 & 7.390851e-01 & 7.013688e-01 \\
7 & 7.390851e-01 & 7.390851e-01 & 7.639597e-01 \\
\bottomrule
\end{tabular}
\end{table}
Si osserva la convergenza più rapida di Newton e Secanti rispetto alle Corde.
\end{esempio}

\subsection{Gestione delle Radici Multiple}
Abbiamo visto che il calcolo di radici multiple è un problema mal condizionato e che l'ordine di convergenza di Newton degrada a lineare. Vediamo come ovviare a questo.

\paragraph{Caso 1: Molteplicità $m$ Nota}
Se $f(x)$ ha una radice $x^*$ di molteplicità $m>1$, significa che $f(x^*) = f'(x^*) = \dots = f^{(m-1)}(x^*) = 0$ e $f^{(m)}(x^*) \neq 0$. In un intorno di $x^*$, la funzione può essere scritta come:
$$ f(x) = (x-x^*)^m g(x) $$
dove $g(x)$ è una funzione tale che $g(x^*) \neq 0$.

Per analizzare il comportamento del metodo di Newton, consideriamo il caso più semplice in cui $g(x)$ è una costante, $g(x) = c \neq 0$. Quindi:
$$ f(x) = c(x-x^*)^m $$
La sua derivata prima è:
$$ f'(x) = cm(x-x^*)^{m-1} $$
Vediamo cosa succede applicando l'iterazione standard di Newton a questa funzione:
$$ x_{i+1} = x_i - \frac{f(x_i)}{f'(x_i)} = x_i - \frac{c(x_i-x^*)^m}{cm(x_i-x^*)^{m-1}} = x_i - \frac{x_i - x^*}{m} $$
Questo mostra che l'errore si riduce solo linearmente.

Se invece utilizzassimo l'iterazione seguente, detta \textbf{Metodo di Newton Modificato}:
$$ x_{i+1} = x_i - m \frac{f(x_i)}{f'(x_i)} $$
Nel caso semplificato $f(x)=c(x-x^*)^m$, otterremmo:
$$ x_{i+1} = x_i - m \frac{c(x_i-x^*)^m}{cm(x_i-x^*)^{m-1}} = x_i - m \frac{x_i - x^*}{m} = x_i - (x_i - x^*) = x^* $$
Si può dimostrare che, anche nel caso generale in cui $g(x)$ non è costante, l'iterazione del Metodo di Newton Modificato ripristina la convergenza \textbf{quadratica} verso la radice multipla. Il costo per iterazione rimane lo stesso del metodo standard (1 valutazione di $f$ e 1 di $f'$).
\paragraph{Caso 2: Molteplicità $m$ Incognita}
Se $m$ non è noto, sappiamo che Newton converge linearmente:
$$ \lim_{i \to \infty} \frac{e_{i+1}}{e_i} = \frac{m-1}{m} = C $$
Per $i$ grande, $e_{i+1} \approx C e_i$ e $e_i \approx C e_{i-1}$. Dividendo membro a membro:
$$ \frac{e_{i+1}}{e_i} \approx \frac{e_i}{e_{i-1}} \implies e_{i+1} e_{i-1} \approx e_i^2 $$
Sostituendo $e_k = x^* - x_k$:
$$ (x^* - x_{i+1})(x^* - x_{i-1}) \approx (x^* - x_i)^2 $$
Trattando l'approssimazione come un'uguaglianza e risolvendo per $x^*$, si ottiene una nuova stima $x_i^*$ della radice:
$$ (x_i^*)^2 - (x_{i+1}+x_{i-1})x_i^* + x_{i+1}x_{i-1} = (x_i^*)^2 - 2x_i x_i^* + x_i^2 $$
$$ x_i^* (2x_i - x_{i+1} - x_{i-1}) = x_i^2 - x_{i+1}x_{i-1} $$
$$ x_i^* = \frac{x_i^2 - x_{i+1}x_{i-1}}{2x_i - x_{i+1} - x_{i-1}} = \frac{x_{i+1}x_{i-1} - x_i^2}{x_{i+1} - 2x_i + x_{i-1}} $$
Questa formula definisce un processo a due livelli noto come \textbf{Metodo di Accelerazione di Aitken ($\Delta^2$)}:
\begin{enumerate}
    \item Si eseguono due passi del metodo originale (es. Newton) per ottenere $x_{i-1}, x_i, x_{i+1}$.
    \item Si usa la formula di Aitken per calcolare $x_i^*$, che diventa il nuovo punto di partenza.
\end{enumerate}
Il costo per iterazione raddoppia (2 passi di Newton + formula di Aitken), ma si può dimostrare che la successione $\{x_i^*\}$ converge \textbf{quadraticamente} a $x^*$, anche per radici multiple. La convergenza rimane locale.

\begin{esempio}[Confronto per $f(x) = (x-1)e^x$]
La funzione ha una radice semplice in $x^*=1$. La tabella confronta Newton, Newton Modificato (con $m=1$, quindi uguale a Newton) e Aitken partendo da $x_0 = 1.1$.

\begin{table}[H]
\centering
\caption{Iterazioni per $f(x)=(x-1)e^x$, $x_0=1.1$}
\begin{tabular}{cccc}
\toprule
Iter (i) & Newton & Newton Mod. (m=1) & Aitken \\
\midrule
0 & 0 & 0 & 0 \\
1 & 1.100000e+00 & 1.100000e+00 & 1.100000e+00 \\
2 & 1.051190e+00 & 1.051190e+00 & 1.004481e+00 \\ % Aitken usa x0, x1, x2 di Newton
3 & 1.025911e+00 & 1.025911e+00 & 1.000010e+00 \\
4 & 1.013038e+00 & 1.013038e+00 & 1.000000e+00 \\
5 & 1.006540e+00 & 1.006540e+00 & 1.000000e+00 \\
6 & 1.003275e+00 & 1.003275e+00 & 1.000000e+00 \\
\bottomrule
\end{tabular}
\end{table}
(Nota: La tabella fornita negli appunti sembra riferirsi a una radice multipla, non $f(x)=(x-1)e^x$. Ho corretto l'esempio per coerenza).
\end{esempio}

\subsection*{Implementazione Matlab del Metodo di Newton Modificato}

\begin{lstlisting}
    function [x, it, xx] = newton(fun, x0, tol, maxit, molt) 
    % [x, it, xx] = newton(fun, x0, tol, maxit, molt)
    %
    % Metodo di Newton per gli zeri di una funzione.
    %
    % Input:
    %   fun  - identificatore function che implementa
    %          il problema da risolvere, del tipo
    %          [fx, f1x] = fun(x),
    %          con fx, f1x rispettivamente i valori della funzione
    %          e della derivata prima, calcolati in x;
    %   x0   - approssimazione iniziale; 
    %   tol  - tolleranza per il criterio di arresto 
    %          (default = 1E-8);
    %   maxit- numero massimo di iterazioni 
    %          (default = 1000);
    %   molt - molteplicità della radice 
    %          (default = 1)
    %
    % Output:
    %   x    - approssimazione trovata;
    %   it   - numero di iterazioni per ottenere la convergenza. Se it = -1, 
    %          convergenza non raggiunta entro maxit;
    %   xx   - (opzionale) vettore con le approssimazioni ottenute.
    %
    % Rel. 2025-10-28.
   
    if nargin < 2
        error('numero argomenti insufficiente');
    end

    if nargin >= 3
        if tol <= 0 
            error('tolleranza non idonea');
        end

        if nargin >= 4
            if maxit < 1 
                error('numero massimo iterazioni errato');
            end
       
            if nargin >= 5
                if molt < 1 || molt ~= fix(molt) 
                    error('molteplicità non corretta');
                end
            else 
                molt = 1;
            end 
            
        else
            maxit = 1000;
            molt = 1
        end

    else 
        tol = 1e-8;
        maxit = 1000;
        molt = 1;
    end
    
    if nargout == 3
        flag = 1; 
        xx = zeros(maxit+1, 1);
        xx(1) = x0;
    else
        flag = 0; 
    end

    for i = 1:maxit
        [fx, f1x] = feval(fun, x0); 
        
        if f1x == 0 && fx == 0
            x = x0;
            if flag, xx = xx(1:i), it=i; end  
        else if f1x == 0
            error('Metodo non applicabile');
            it=-1;
            if flag, xx = xx(1:i), end
        
        else
            x = x0 - molt * fx / f1x;
            if flag, xx(i+1)=x; end
        end

        err = abs(x - x0) / (1 + abs(x0)); 
        if err <= tol
            it = i; 
            if flag, xx = xx(1:i); end 
            break; 
        end
        
        x0 = x; 
    end
 
    if err > tol, it = -1; end
        
    return
  
    \end{lstlisting}

\section{Risoluzione di Sistemi Lineari}
% Riguardare l'Appendice A1 del libro per richiami di algebra lineare.

Il problema consiste nel risolvere un sistema di $m$ equazioni lineari in $n$ incognite:
\[
\begin{cases}
    a_{11}x_1 + a_{12}x_2 + \dots + a_{1n}x_n &= b_1 \\
    a_{21}x_1 + a_{22}x_2 + \dots + a_{2n}x_n &= b_2 \\
    \vdots \\
    a_{m1}x_1 + a_{m2}x_2 + \dots + a_{mn}x_n &= b_m
\end{cases}
\]
dove i coefficienti $a_{ij}$ e i termini noti $b_i$ sono assegnati, mentre le incognite $x_j$ sono da determinare.

Possiamo riscrivere il sistema in forma vettoriale (o matriciale):
\begin{equation} \label{eq:sistema_lineare}
    A\mathbf{x} = \mathbf{b}
\end{equation}
introducendo:
\begin{itemize}
    \item La \textbf{matrice dei coefficienti} $A \in \mathbb{R}^{m \times n}$:
    \[ A = 
    \begin{pmatrix}
        a_{11} & a_{12} & \dots & a_{1n} \\
        a_{21} & a_{22} & \dots & a_{2n} \\
        \vdots & \vdots & \ddots & \vdots \\
        a_{m1} & a_{m2} & \dots & a_{mn}
    \end{pmatrix}
    \]
    \item Il \textbf{vettore dei termini noti} $\mathbf{b} \in \mathbb{R}^m$:
    \[ \mathbf{b} = \begin{pmatrix} b_1 \\ b_2 \\ \vdots \\ b_m \end{pmatrix} \]
    \item Il \textbf{vettore delle incognite} $\mathbf{x} \in \mathbb{R}^n$:
    \[ \mathbf{x} = \begin{pmatrix} x_1 \\ x_2 \\ \vdots \\ x_n \end{pmatrix} \]
\end{itemize}

Nella nostra trattazione, assumeremo sempre che:
\begin{enumerate}
    \item $m \ge n$ (numero di equazioni maggiore o uguale al numero di incognite). Pertanto il numero di colonne della matrice A è $\leq$ del numero di righe:
    \begin{itemize}
        \item La riga $i$-esima di $A$ è il vettore riga: $(a_{i1}, a_{i2}, \dots, a_{in}) \in \mathbb{R}^{1 \times n}$.
        \item La colonna $j$-esima di $A$ è il vettore colonna: $\begin{pmatrix} a_{1j} \\ a_{2j} \\ \vdots \\ a_{mj} \end{pmatrix} \in \mathbb{R}^m$.
        \item $a_{ij}$ è l'elemento che si trova all'intersezione della riga $i$-esima con la colonna $j$-esima.
    \end{itemize}
    \item La matrice $A$ abbia \textbf{rango massimo}, ovvero $\text{rank}(A) = n$. Questo implica che le colonne di $A$ sono vettori linearmente indipendenti.
\end{enumerate}

Distingueremo due casi significativi:
\begin{enumerate}
    \item $m=n \iff$ A è una matrice quadrata;
    \item $m>n \iff$ A è a rango massimo
\end{enumerate}

\subsection{Il Caso Quadrato ($m=n$)}
Se $A \in \mathbb{R}^{n \times n}$ e $\text{rank}(A) = n$, allora $A$ è una matrice \textbf{nonsingolare} (o invertibile). Questo significa che:
\begin{itemize}
    \item Esiste ed è unica la matrice inversa $A^{-1}$ tale che $A^{-1}A = AA^{-1} = I$, dove $I$ è la matrice identità $n \times n$.
    \item Il determinante di $A$ è diverso da zero: $\det(A) \neq 0$.
\end{itemize}
In questo caso, il sistema lineare $A\mathbf{x} = \mathbf{b}$ ammette un'unica soluzione. Moltiplicando entrambi i membri a sinistra per $A^{-1}$, otteniamo:
$$ A^{-1}(A\mathbf{x}) = A^{-1}\mathbf{b} \implies (A^{-1}A)\mathbf{x} = A^{-1}\mathbf{b} \implies I\mathbf{x} = A^{-1}\mathbf{b} $$
Quindi, la soluzione formale è:
$$ \mathbf{x} = A^{-1}\mathbf{b} $$
\begin{osservazione}
Sebbene questa espressione fornisca la soluzione, calcolare esplicitamente l'inversa $A^{-1}$ per poi moltiplicarla per $\mathbf{b}$ non è generalmente efficiente dal punto di vista computazionale. Si preferiscono metodi diversi, che vedremo nel seguito. Useremo questa formula solo in casi molto particolari.
\end{osservazione}

\subsection{Sistemi Lineari: Casi Semplici}
Cominciamo esaminando casi in cui la matrice $A$ ha una struttura particolare che rende la risoluzione del sistema $A\mathbf{x} = \mathbf{b}$ particolarmente semplice. Questi casi serviranno come base per metodi più generali. Le strutture che considereremo sono:
\begin{itemize}
    \item $A$ diagonale
    \item $A$ triangolare
    \item $A$ ortogonale
\end{itemize}
L'ordine di presentazione segue la complessità computazionale crescente, misurata in termini di occupazione di memoria e numero di operazioni algebriche (flops) richieste.

\subsubsection{$A$ diagonale}
In questo caso, $a_{ij} = 0$ per ogni $i \neq j$.
\[ A = 
\begin{pmatrix}
    a_{11} & 0 & \dots & 0 \\
    0 & a_{22} & \dots & 0 \\
    \vdots & \vdots & \ddots & \vdots \\
    0 & 0 & \dots & a_{nn}
\end{pmatrix}
\]
\begin{osservazione}[Struttura Diagonale]
La differenza $k = |j-i|$ indica la diagonale: $k=0$ è la diagonale principale, $k>0$ è la $k$-esima sopradiagonale, $k<0$ è la $(-k)$-esima sottodiagonale. Per una matrice diagonale, solo gli elementi con $k=0$ possono essere non nulli.
\end{osservazione}
Per memorizzare gli elementi significativi di $A$ è sufficiente un vettore di lunghezza $n$. Una matrice diagonale è un caso particolare di \textbf{matrice sparsa} (una matrice con un numero di elementi non nulli molto inferiore a $n^2$).

Il sistema $A\mathbf{x} = \mathbf{b}$ diventa:
\[
\begin{cases}
    a_{11}x_1 &= b_1 \\
    a_{22}x_2 &= b_2 \\
    \vdots \\
    a_{nn}x_n &= b_n
\end{cases}
\]
Poiché $A$ è nonsingolare, $\det(A) = \prod_{i=1}^n a_{ii} \neq 0$, il che implica $a_{ii} \neq 0$ per ogni $i=1, \dots, n$.
Pertanto, la soluzione si ottiene immediatamente con $n$ divisioni:
$$ x_i = \frac{b_i}{a_{ii}}, \quad i=1, \dots, n $$
In conclusione, per risolvere un sistema diagonale $n \times n$ sono sufficienti:
\begin{itemize}
    \item Memoria per 2 vettori di lunghezza $n$ (uno per la diagonale di $A$, uno per $\mathbf{b}$ che viene sovrascritto con $\mathbf{x}$).
    \item $n$ operazioni algebriche (flops).
\end{itemize}


%----LEZIONE 29 OTTOBRE-----%
% Riguardare l'Appendice A del libro
\subsection{Prodotto Matrice-Vettore}
Ricordiamo la notazione: $A = (\mathbf{c}_1 | \dots | \mathbf{c}_n) = \begin{pmatrix} \mathbf{r}_1^T \\ \vdots \\ \mathbf{r}_n^T \end{pmatrix}$.
Definiamo i versori della base canonica di $\mathbb{R}^n$:
$$ \mathbf{e}_i = \begin{pmatrix} 0 \\ \vdots \\ 1 \\ \vdots \\ 0 \end{pmatrix} \leftarrow \text{posizione } i $$
Allora:
\begin{itemize}
    \item $\mathbf{e}_i^T A = \mathbf{r}_i^T$ (seleziona l'$i$-esima riga di $A$)
    \item $A \mathbf{e}_j = \mathbf{c}_j$ (seleziona la $j$-esima colonna di $A$)
    \item $\mathbf{e}_i^T A \mathbf{e}_j = a_{ij}$ (seleziona l'elemento $(i,j)$)
\end{itemize}
Inoltre, la matrice identità $I \in \mathbb{R}^{n \times n}$ si può scrivere come:
$$ I = \sum_{j=1}^n \mathbf{e}_j \mathbf{e}_j^T $$

Calcolare $\mathbf{y} = A\mathbf{x}$ può essere fatto in due modi equivalenti:

\paragraph{1. Prodotto righe per colonne (prodotto scalare)}
L'elemento $i$-esimo di $\mathbf{y}$ è il prodotto scalare tra l'$i$-esima riga di $A$ e il vettore $\mathbf{x}$:
\begin{equation}
    y_i = \mathbf{r}_i^T \mathbf{x} = \sum_{j=1}^n a_{ij} x_j, \quad \text{per } i=1, \dots, n
\end{equation}

\paragraph{2. Combinazione lineare di colonne (operazione "axpy")}
Il vettore $\mathbf{y}$ è una combinazione lineare delle colonne di $A$, con coefficienti gli elementi di $\mathbf{x}$:
\begin{equation}
    \mathbf{y} = A \mathbf{x} = A (I \mathbf{x}) = A \left(\sum_{j=1}^n \mathbf{e}_j \mathbf{e}_j^T\right) \mathbf{x} = \sum_{j=1}^n (A \mathbf{e}_j) (\mathbf{e}_j^T \mathbf{x}) = \sum_{j=1}^n \mathbf{c}_j x_j
\end{equation}

\paragraph{Implementazione e Costo}
Entrambi gli approcci hanno un costo di $O(n^2)$ flops. Per $A \in \mathbb{R}^{n \times n}$, il costo è $\approx 2n^2$ flops (se $A \in \mathbb{R}^{m \times n}$, $\approx 2mn$ flops).

\begin{itemize}
    \item \textbf{Algoritmo (2) - Righe per Colonne (accesso per righe)}:
    \begin{lstlisting}[numbers=none, frame=none, basicstyle=\ttfamily]
for i = 1:n
    % y(i) = 0; (inizializzazione)
    for j = 1:n
        y(i) = y(i) + A(i,j) * x(j); % Prodotto scalare
    end
end
    \end{lstlisting}
    
    \item \textbf{Algoritmo (3) - Combinazione Lineare (accesso per colonne)}:
    \begin{lstlisting}[numbers=none, frame=none, basicstyle=\ttfamily]
% y = zeros(n,1); (inizializzazione)
for j = 1:n
    for i = 1:n
        y(i) = y(i) + A(i,j) * x(j); % Operazione axpy
    end
end
    \end{lstlisting}
\end{itemize}

La scelta tra (2) e (3) dipende dalla modalità di memorizzazione della matrice nel linguaggio di programmazione (per righe o per colonne), per sfruttare al meglio la località dei dati.


\subsection{Matrici Triangolari}
\begin{definition}
Una matrice $A=(a_{ij}) \in \mathbb{R}^{n \times n}$ è:
\begin{itemize}
    \item \textbf{Triangolare inferiore} se $a_{ij} = 0$ per $j > i$ (elementi sopra la diagonale nulli).
    \item \textbf{Triangolare superiore} se $a_{ij} = 0$ per $i > j$ (elementi sotto la diagonale nulli).
\end{itemize}
\end{definition}

\begin{osservazione}
Una matrice che è contemporaneamente triangolare inferiore e superiore è una matrice \textbf{diagonale}.
\end{osservazione}

\begin{teorema}
Se $A$ è una matrice triangolare (inferiore o superiore), il suo determinante è il prodotto degli elementi diagonali:
$$ \det(A) = \prod_{i=1}^n a_{ii} $$
Ne consegue che $A$ è nonsingolare ($\det(A) \neq 0$) se e solo se tutti i suoi elementi diagonali sono non nulli ($a_{ii} \neq 0$ per ogni $i$).
\end{teorema}

\paragraph{Risoluzione di Sistemi Triangolari Inferiori}
Esaminiamo il caso $A\mathbf{x} = \mathbf{b}$ con $A$ triangolare inferiore (il caso superiore è analogo).
\[
\begin{pmatrix}
    a_{11} & 0 & \dots & 0 \\
    a_{21} & a_{22} & \dots & 0 \\
    \vdots & \vdots & \ddots & \vdots \\
    a_{n1} & a_{n2} & \dots & a_{nn}
\end{pmatrix}
\begin{pmatrix} x_1 \\ x_2 \\ \vdots \\ x_n \end{pmatrix} = 
\begin{pmatrix} b_1 \\ b_2 \\ \vdots \\ b_n \end{pmatrix}
\]
Il sistema è:
\[
\begin{array}{lcr}
a_{11}x_1 & = & b_1 \\
a_{21}x_1 + a_{22}x_2 & = & b_2 \\
\vdots & & \vdots \\
a_{n1}x_1 + a_{n2}x_2 + \dots + a_{nn}x_n & = & b_n
\end{array}
\]
Possiamo risolvere il sistema mediante \textbf{sostituzioni successive in avanti}:
\begin{enumerate}
    \item Dalla prima equazione ricaviamo $x_1 = b_1 / a_{11}$.
    \item Sostituiamo $x_1$ nella seconda e ricaviamo $x_2 = (b_2 - a_{21}x_1) / a_{22}$.
    \item E così via...
\end{enumerate}
La formula generale per $i=1, \dots, n$ è:
\begin{equation} \label{eq:sost_avanti}
    x_i = \frac{b_i - \sum_{j=1}^{i-1} a_{ij}x_j}{a_{ii}}
\end{equation}
(con la convenzione $\sum_{j=1}^0 (\dots) = 0$).

\paragraph{Costo Computazionale}
Dalla formula \eqref{eq:sost_avanti} otteniamo che:
\begin{itemize}
    \item L'algoritmo risolutivo è ben definito se e solo se $a_{ii} \neq 0$ per $i=1, \dots, n$, il che è vero poiché $\det(A) \neq 0$ (come abbiamo assunto).
    \item Il numero di operazioni all'iterazione $i$-esima è $2i-1$ flops.
    \item Il costo totale è di $\sum_{i=1}^{n}(2i-1) = 2\sum_{i=1}^{n}i - n = 2\frac{n(n+1)}{2} - n = n^2$ flops.
    \item Anche la memoria richiesta è $\sim \frac{n^2}{2}$ posizioni di memoria.
\end{itemize}
Pertanto la complessità è $O(n^2)$, sia in termini di memoria che di flops.

\paragraph{Algoritmi di Sostituzione in Avanti}
Scriviamo uno pseudo-codice per la formula \eqref{eq:sost_avanti}, in cui supponiamo di avere i vettori $\mathbf{x}$ e $\mathbf{b}$ e la matrice $A$ $n \times n$. Inizializziamo $\mathbf{x} \leftarrow \mathbf{b}$, ovvero assumiamo che il vettore $\mathbf{b}$ sia memorizzato in $\mathbf{x}$ e venga sovrascritto dalla soluzione.

\begin{itemize}
    \item \textbf{Versione "per righe" (accesso per righe ad A)}:
    \begin{lstlisting}[numbers=none, frame=none, basicstyle=\ttfamily]
% Algoritmo (5) - Sostituzione in avanti per righe
for i = 1:n
    for j = 1:i-1
        x(i) = x(i) - A(i,j) * x(j); % (scal)
    end
    x(i) = x(i) / A(i,i);
end
    \end{lstlisting}
    
    \item \textbf{Versione "per colonne" (accesso per colonne ad A)}:
    \begin{lstlisting}[numbers=none, frame=none, basicstyle=\ttfamily]
% Algoritmo (6) - Sostituzione in avanti per colonne
for j = 1:n
    x(j) = x(j) / A(j,j);
    for i = j+1:n
        x(i) = x(i) - A(i,j) * x(j); % (axpy)
    end
end
    \end{lstlisting}
\end{itemize}

\begin{osservazione}
    Osserviamo che i due algoritmi, sebbene algebricamente equivalenti, in aritmetica finita producono generalmente risultati non uguali.
    \end{osservazione}

% --- LEZIONE 4 NOVEMBRE ---

\paragraph{Risoluzione di Sistemi Triangolari Superiori}
Esaminiamo ora il caso in cui $A$ sia triangolare superiore. Il caso è analogo a quello inferiore.
\[ A = 
\begin{pmatrix}
    a_{11} & a_{12} & \dots & a_{1n} \\
    0 & a_{22} & \dots & a_{2n} \\
    \vdots & \vdots & \ddots & \vdots \\
    0 & 0 & \dots & a_{nn}
\end{pmatrix}
\]
Ricordiamo che $A$ è nonsingolare se e solo se $a_{ii} \neq 0$ per ogni $i$.

Il sistema $A\mathbf{x} = \mathbf{b}$ diventa:
\[
\begin{array}{r c c c c c c l}
a_{11}x_1 & + & a_{12}x_2 & + & \dots & + & a_{1n}x_n & = b_1 \\
          &   & a_{22}x_2 & + & \dots & + & a_{2n}x_n & = b_2 \\
          &   &           & \ddots &       & \vdots   & \vdots & \vdots \\
          &   &           &        &       & a_{nn}x_n & = b_n
\end{array}
\]
Questo sistema può essere risolto mediante \textbf{sostituzioni successive all'indietro}. Partendo dall'ultima equazione e risalendo:
$$ x_{n-i} = \frac{b_{n-i} - \sum_{j=n-i+1}^n a_{n-i, j} x_j}{a_{n-i, n-i}}, \quad i=0, \dots, n-1 $$
Il costo computazionale è identico al caso triangolare inferiore, ovvero $n^2$ flops.

\paragraph{Algoritmi di Sostituzione all'Indietro}
Anche in questo caso, possiamo codificare la soluzione in due modi algebricamente equivalenti, con diverso accesso ai dati. 
Sia \texttt{a} un array $n \times n$ contenente gli elementi della matrice $A$, e sia \texttt{x} un vettore di lunghezza $n$, che inizializziamo con il vettore dei termini noti $\mathbf{b}$:
\[ \mathbf{x} \leftarrow \mathbf{b} \]
(Assumiamo $\mathbf{x}$ sovrascriva $\mathbf{b}$).

\begin{itemize}
    \item \textbf{Versione "per righe" (accesso per righe ad A)}:
    \begin{lstlisting}[numbers=none, frame=none, basicstyle=\ttfamily]
% Sostituzione all'indietro (per righe)
for i = n:-1:1
    for j = i+1:n
        x(i) = x(i) - A(i,j) * x(j); % (scal)
    end
    x(i) = x(i) / A(i,i);
end
    \end{lstlisting}
    
    \item \textbf{Versione "per colonne" (accesso per colonne ad A)}:
    \begin{lstlisting}[numbers=none, frame=none, basicstyle=\ttfamily]
% Sostituzione all'indietro (per colonne)
for j = n:-1:1
    x(j) = x(j) / A(j,j);
    for i = 1:j-1
        x(i) = x(i) - A(i,j) * x(j); % (axpy)
    end
end
    \end{lstlisting}
\end{itemize}

\begin{osservazione}[Vettorizzazione in Matlab]
Ove possibile, utilizzare in Matlab la notazione vettoriale. L'ultimo algoritmo (accesso "per colonne") può essere riscritto in modo più efficiente e compatto:
\begin{lstlisting}
% Sostituzione all'indietro (vettorizzata)
for j = n:-1:1
    x(j) = x(j) / A(j,j);
    x(1:j-1) = x(1:j-1) - A(1:j-1, j) * x(j);
end
\end{lstlisting}
Questa ultima scrittura è sostanzialmente più efficiente. Usatela.
\end{osservazione}

\begin{esercizio}[Proprietà delle Matrici Triangolari]
Dimostrare le seguenti proprietà per $A, B$ matrici triangolari $n \times n$:
\begin{enumerate}
    \item[A)] Se $A, B$ sono triangolari inferiori (risp. superiori), allora $C=A+B$ e $C=A \cdot B$ sono anch'esse triangolari inferiori (risp. superiori).
    Inoltre, per gli elementi diagonali vale: $c_{ii} = a_{ii} + b_{ii}$ (per la somma) e $c_{ii} = a_{ii} \cdot b_{ii}$ (per il prodotto).
    
    \item[B)] Se $A, B$ sono triangolari inferiori (risp. superiori) a \textbf{diagonale unitaria}, allora $C=A \cdot B$ è anch'essa triangolare inferiore (risp. superiore) a diagonale unitaria.
    
    \item[C)] Se $A$ è triangolare (inf. o sup.) e nonsingolare, allora $A^{-1}$ è dello stesso tipo (inf. o sup.) e $(A^{-1})_{ii} = a_{ii}^{-1}$.
    
    \item[D)] Se $A$ è triangolare (inf. o sup.) a diagonale unitaria, allora $A^{-1}$ è anch'essa triangolare (inf. o sup.) a diagonale unitaria.
\end{enumerate}
\end{esercizio}

\begin{proof}[Dimostrazione (Proprietà A)]
    Che $C=A+B$ sia triangolare dello stesso tipo di $A$ e $B$, discende dal fatto che $c_{ij} = a_{ij} + b_{ij}$.
    
    Se $C=A \cdot B$, supponiamo che $A$ e $B$ siano \textbf{triangolari inferiori}, ovvero $a_{ij}=b_{ij}=0$ se $j>i$. Dobbiamo dimostrare che: 1) $c_{ij}=0$ se $j>i$, e 2) $c_{ii} = a_{ii} \cdot b_{ii}$.
    
    Infatti, se $\mathbf{e}_i, \mathbf{e}_j \in \mathbb{R}^n$ sono i versori $i$ e $j$:
    \begin{align*}
        c_{ij} &= \mathbf{e}_i^T C \mathbf{e}_j = (\mathbf{e}_i^T A) (B \mathbf{e}_j) \\
             &= 
        \begin{pmatrix}
            a_{i1} & \dots & a_{ii} & \overbrace{0 \dots 0}^{n-i}
        \end{pmatrix}
        \begin{pmatrix}
            0 \\ \vdots \\ 0 \\ b_{jj} \\ \vdots \\ b_{nj}
        \end{pmatrix}
        \left. \vphantom{\begin{matrix} 0 \\ \vdots \\ 0 \end{matrix}} \right\} j-1 \text{ zeri}
        \\
        &=
        \begin{cases} 
            a_{ii} \cdot b_{ii} & \text{se } i=j \\
            0 & \text{se } j>i \text{ (o } i<j \text{)}
        \end{cases}
    \end{align*}
    
    \end{proof}

\subsection{Matrici Ortogonali}
\begin{definition}
Una matrice $A \in \mathbb{R}^{n \times n}$ è \textbf{ortogonale} se $A^T A = A A^T = I$.
Questo significa che l'inversa di $A$ è la sua trasposta: $A^{-1} = A^T$.
\end{definition}
In questo caso, la soluzione del sistema $A\mathbf{x} = \mathbf{b}$ è immediata:
$$ \mathbf{x} = A^{-1}\mathbf{b} = A^T \mathbf{b} $$
La soluzione si ottiene con un prodotto matrice-vettore (costo $\approx 2n^2$ flops).

\subsection{Metodi di Fattorizzazione}
L'analisi dei casi semplici ci permette di affrontare il caso generale $A\mathbf{x} = \mathbf{b}$ (con $A \in \mathbb{R}^{n \times n}$ nonsingolare). I metodi che esamineremo sono detti \textbf{metodi di fattorizzazione}.
Si cercherà una decomposizione (fattorizzazione) di $A$ del tipo:
$$ A = F_1 F_2 \dots F_k $$
dove $k$ è "piccolo" e i fattori $F_i$ sono matrici di tipo "semplice" (diagonali, triangolari o ortogonali), per cui i sistemi lineari con tali fattori sono facilmente risolvibili.

Se $A = F_1 F_2$ (caso $k=2$), il sistema $A\mathbf{x} = \mathbf{b}$ diventa $F_1(F_2 \mathbf{x}) = \mathbf{b}$.
Possiamo risolverlo in due passi:
\begin{enumerate}
    \item Risolvi $F_1 \mathbf{y} = \mathbf{b}$ per trovare $\mathbf{y}$.
    \item Risolvi $F_2 \mathbf{x} = \mathbf{y}$ per trovare $\mathbf{x}$.
\end{enumerate}
In generale, per $A = F_1 \dots F_k$, posto $\mathbf{x}_0 = \mathbf{b}$, si risolvono i $k$ sistemi $F_i \mathbf{x}_i = \mathbf{x}_{i-1}$ per $i=1, \dots, k$. La soluzione finale sarà $\mathbf{x} = \mathbf{x}_k$.

\begin{osservazione}[Implementazione]
\begin{enumerate}
    \item In pratica, non sarà necessario memorizzare esplicitamente i fattori $F_i$, ma si potrà sovrascrivere la matrice $A$ con l'informazione relativa ai suoi fattori.
    \item Non sarà necessario memorizzare le soluzioni intermedie. Lo stesso vettore può contenere il termine noto $\mathbf{b}$ e poi essere sovrascritto con le soluzioni intermedie.
\end{enumerate}
\end{osservazione}

\subsection{Fattorizzazione LU di una matrice}
\begin{definition}
Una matrice $A \in \mathbb{R}^{n \times n}$ (nonsingolare) è \textbf{fattorizzabile LU} se esistono:
\begin{itemize}
    \item $L$: matrice triangolare inferiore a \textbf{diagonale unitaria} ($l_{ii}=1$).
    \item $U$: matrice triangolare superiore.
\end{itemize}
tali che $A = L U$.
\end{definition}

\begin{osservazione}
Se $A=LU$, per risolvere $A\mathbf{x}=\mathbf{b}$ si risolvono i due sistemi di tipo semplice $L\mathbf{y}=\mathbf{b}$ e $U\mathbf{x}=\mathbf{y}$, con un costo totale di $2n^2$ flops.
\end{osservazione}

\begin{teorema}[Unicità della Fattorizzazione LU]
Se $A$ è nonsingolare e la fattorizzazione $A=LU$ (con $L$ a diagonale unitaria) esiste, allora tale fattorizzazione è unica.
\end{teorema}
\begin{proof}
Supponiamo per assurdo che esistano due fattorizzazioni $A = L_1 U_1$ e $A = L_2 U_2$.
Poiché $L_1, L_2$ hanno diagonale unitaria, $\det(L_1) = \det(L_2) = 1$.
Dato che $A$ è nonsingolare, $0 \neq \det(A) = \det(L_1)\det(U_1) = \det(U_1)$. Quindi $U_1$ (e analogamente $U_2$) è nonsingolare.

Dall'uguaglianza $L_1 U_1 = L_2 U_2$, moltiplichiamo a sinistra per $L_2^{-1}$ e a destra per $U_1^{-1}$:
$$ L_2^{-1} (L_1 U_1) U_1^{-1} = L_2^{-1} (L_2 U_2) U_1^{-1} $$
$$ (L_2^{-1} L_1) (U_1 U_1^{-1}) = (L_2^{-1} L_2) (U_2 U_1^{-1}) $$
$$ L_2^{-1} L_1 = U_2 U_1^{-1} $$
La matrice $L_2^{-1} L_1$ è triangolare inferiore a diagonale unitaria (dalle proprietà delle matrici triangolari).
La matrice $U_2 U_1^{-1}$ è triangolare superiore.
L'unica matrice che è contemporaneamente triangolare inferiore e superiore è una matrice diagonale. Poiché $L_2^{-1} L_1$ ha diagonale unitaria, questa matrice diagonale deve essere la matrice identità $I$.
Quindi:
$$ L_2^{-1} L_1 = I \implies L_1 = L_2 $$
$$ U_2 U_1^{-1} = I \implies U_2 = U_1 $$
La fattorizzazione è unica.
\end{proof}

% lezione 5 novembre
\subsubsection{Esistenza della Fattorizzazione LU}
Per dimostrare costruttivamente l'esistenza della fattorizzazione, vediamo prima un problema preliminare. Supponiamo di avere un vettore $\mathbf{v} = (v_1, \dots, v_n)^T \in \mathbb{R}^n$. Vogliamo azzerare le sue componenti dalla $(k+1)$-esima in poi, mediante moltiplicazione a sinistra per una matrice $L$ che sia triangolare inferiore e a diagonale unitaria. Vogliamo cioè definire $L$ tale che:
$$ L\mathbf{v} = \begin{pmatrix} v_1 \\ \vdots \\ v_k \\ 0 \\ \vdots \\ 0 \end{pmatrix} $$
Se $v_k \neq 0$, possiamo definire il \textbf{vettore elementare di Gauss}:
$$ \mathbf{g}_k = \frac{1}{v_k} ( \underbrace{0, \dots, 0}_{k \text{ zeri}}, v_{k+1}, \dots, v_n )^T \in \mathbb{R}^n $$
Detto $\mathbf{e}_k$ il $k$-esimo versore, definiamo la \textbf{matrice elementare di Gauss}:
$$ L_k = I - \mathbf{g}_k \mathbf{e}_k^T = 
\begin{pmatrix}
1 & & & & & & \\
& \ddots & & & & & \\
& & 1 & & & & \\
& & & 1 & & & \\
& & & -v_{k+1}/v_k & 1 & & \\
& & & \vdots & & \ddots & \\
& & & -v_n/v_k & & & 1
\end{pmatrix} \leftarrow \text{riga } k
$$
Questa matrice è, per costruzione, triangolare inferiore a diagonale unitaria. Inoltre:
\begin{align*}
    L_k \mathbf{v} &= (I - \mathbf{g}_k \mathbf{e}_k^T) \mathbf{v} = \mathbf{v} - \mathbf{g}_k (\mathbf{e}_k^T \mathbf{v}) \\
                   &= \mathbf{v} - \mathbf{g}_k (v_k) = \mathbf{v} - \frac{1}{v_k} \begin{pmatrix} 0 \\ \vdots \\ 0 \\ v_{k+1} \\ \vdots \\ v_n \end{pmatrix} \cdot v_k
                   = \begin{pmatrix} v_1 \\ \vdots \\ v_k \\ v_{k+1} \\ \vdots \\ v_n \end{pmatrix} - 
                     \begin{pmatrix} 0 \\ \vdots \\ 0 \\ v_{k+1} \\ \vdots \\ v_n \end{pmatrix}
                   = \begin{pmatrix} v_1 \\ \vdots \\ v_k \\ 0 \\ \vdots \\ 0 \end{pmatrix}
\end{align*}
come richiesto.

\begin{osservazione}
L'inversa di una matrice elementare di Gauss è $L^{-1} = I + \mathbf{g}_k \mathbf{e}_k^T$.
Infatti:
$$ L^{-1} L = (I + \mathbf{g}_k \mathbf{e}_k^T) (I - \mathbf{g}_k \mathbf{e}_k^T) = I - \mathbf{g}_k \mathbf{e}_k^T + \mathbf{g}_k \mathbf{e}_k^T - \mathbf{g}_k (\mathbf{e}_k^T \mathbf{g}_k) \mathbf{e}_k^T = I $$
poiché $\mathbf{e}_k^T \mathbf{g}_k = 0$ (la $k$-esima componente di $\mathbf{g}_k$ è zero per definizione).
\end{osservazione}

\subsubsection{Metodo di Eliminazione di Gauss (MEG)}
Definiamo ora il metodo costruttivo per la fattorizzazione LU. È un metodo semi-iterativo che consta di $n-1$ passi (se $A \in \mathbb{R}^{n \times n}$). Al passo $j$, l'obiettivo è trasformare la $j$-esima colonna della matrice corrente, $A^{(j)}$, in quella di una matrice triangolare superiore, azzerando gli elementi sotto la diagonale.

Sia $A = (a_{ij}) \equiv A^{(1)}$ la matrice da fattorizzare. L'apice $(k)$ denota l'ultimo passo in cui l'elemento $(i,j)$ è stato modificato.

\paragraph{Passo 1:}
Sia $A^{(1)} = \begin{pmatrix}
a_{11}^{(1)} & a_{12}^{(1)} & \dots & a_{1n}^{(1)} \\
a_{21}^{(1)} & a_{22}^{(1)} & \dots & a_{2n}^{(1)} \\
\vdots & \vdots & \ddots & \vdots \\
a_{n1}^{(1)} & a_{n2}^{(1)} & \dots & a_{nn}^{(1)}
\end{pmatrix}$.
Se $a_{11}^{(1)} \neq 0$, possiamo definire il primo vettore elementare di Gauss:
$$ \mathbf{g}_1 = \frac{1}{a_{11}^{(1)}} (0, a_{21}^{(1)}, \dots, a_{n1}^{(1)})^T $$
e la prima matrice elementare di Gauss $L_1 = I - \mathbf{g}_1 \mathbf{e}_1^T$.
Applicandola ad $A^{(1)}$ otteniamo $A^{(2)}$:
$$ L_1 A^{(1)} = A^{(2)} = \begin{pmatrix}
a_{11}^{(1)} & a_{12}^{(1)} & \dots & a_{1n}^{(1)} \\
0 & a_{22}^{(2)} & \dots & a_{2n}^{(2)} \\
\vdots & \vdots & \ddots & \vdots \\
0 & a_{n2}^{(2)} & \dots & a_{nn}^{(2)}
\end{pmatrix} $$

\paragraph{Passo 2:}
Se $a_{22}^{(2)} \neq 0$, definiamo il secondo vettore di Gauss:
$$ \mathbf{g}_2 = \frac{1}{a_{22}^{(2)}} (0, 0, a_{32}^{(2)}, \dots, a_{n2}^{(2)})^T $$
e la matrice $L_2 = I - \mathbf{g}_2 \mathbf{e}_2^T$. Otteniamo:
$$ L_2 A^{(2)} = L_2 L_1 A^{(1)} = A^{(3)} = \begin{pmatrix}
a_{11}^{(1)} & a_{12}^{(1)} & a_{13}^{(1)} & \dots & a_{1n}^{(1)} \\
0 & a_{22}^{(2)} & a_{23}^{(2)} & \dots & a_{2n}^{(2)} \\
0 & 0 & a_{33}^{(3)} & \dots & a_{3n}^{(3)} \\
\vdots & \vdots & \vdots & \ddots & \vdots \\
0 & 0 & a_{n3}^{(3)} & \dots & a_{nn}^{(3)}
\end{pmatrix} $$

\paragraph{Passo $j$ generico:}
Se $a_{jj}^{(j)} \neq 0$, definiamo $\mathbf{g}_j = \frac{1}{a_{jj}^{(j)}} (0, \dots, 0, a_{j+1,j}^{(j)}, \dots, a_{nj}^{(j)})^T$ e $L_j = I - \mathbf{g}_j \mathbf{e}_j^T$.
Si ottiene $L_j....L_1A = A^{(j+1)}$.

\paragraph{Fine del metodo (dopo $n-1$ passi):}
Se la procedura è stata possibile per $j=1, \dots, n-1$ (cioè $a_{jj}^{(j)} \neq 0$ per ogni passo), si ottiene, infine, che:
$$ L_{n-1} \dots L_2 L_1 A = A^{(n)} \equiv U $$
dove $U$ è una matrice triangolare superiore.
Possiamo quindi concludere che questa procedura è definita se e solo se $a_{jj}^{(j)} \neq 0$ per $j=1, \dots, n-1$, ovvero se e solo se $U$ è nonsingolare.

Dall'uguaglianza $L_{n-1} \dots L_1 A = U$, si ottiene la fattorizzazione osservando che:
\begin{enumerate}
    \item Ogni $L_i$ è triangolare inferiore a diagonale unitaria;
    \item Ogni $L_i^{-1}$ è triangolare inferiore a diagonale unitaria (come visto in precedenza);
    \item Il prodotto di matrici triangolari inferiori a diagonale unitaria è una matrice triangolare inferiore a diagonale unitaria (dall'esercizio sulle proprietà).
\end{enumerate}
Si ottiene che possiamo porre:
$$ L_{n-1} \dots L_1 = L^{-1} $$
con $L = (L_{n-1} \dots L_1)^{-1} = L_1^{-1} \dots L_{n-1}^{-1}$, che è anch'essa triangolare inferiore a diagonale unitaria.
Da questo si ottiene:
$$ L^{-1} A = U $$
ovvero:
$$ A = L U $$
che è la fattorizzazione richiesta.


%lezione del 11 novembre

\subsection{Algoritmo di fattorizzazione LU di Gauss}
Dato $A=(a_{ij}) \in \mathbb{R}^{n \times n}$ con $\det(A) \neq 0$.
L'algoritmo di eliminazione di Gauss consiste in una procedura semi-iterativa di $n-1$ passi, in cui al passo $i$-esimo si azzerano selettivamente gli elementi al di sotto di quello diagonale, in colonna $i$.
Un generico elemento è denotato con $a_{ij}^{(k)}$, dove $k$ denota l'ultimo passo in cui l'elemento $(i,j)$ è stato modificato.

Avevamo visto che se $\forall i=1, \dots, n-1 : a_{ii}^{(i)} \neq 0$, allora è definito l'i-esimo vettore elementare di Gauss,
$$ \mathbf{g}_i = \frac{1}{a_{ii}^{(i)}} (0, \dots, 0, a_{i+1,i}^{(i)}, \dots, a_{ni}^{(i)})^T $$
e la matrice elementare di Gauss
$$ L_i = I - \mathbf{g}_i \mathbf{e}_i^T $$
tali che $L_{n-1} \dots L_1 A = A^{(n)} = U$, e da cui si ricava $A = L U$.

Esaminiamo in primis gli aspetti del costo computazionale, supponendo che la fattorizzazione esista.

\subsubsection{Costo Computazionale}

\paragraph{Memoria}
L'idea è quella di sovrascrivere la matrice $A$ con l'informazione dei suoi fattori $L$ e $U$.
\begin{itemize}
    \item La parte triangolare superiore di $U$ (che è $A^{(n)}$) può essere sovrascritta sulla porzione triangolare superiore di $A$.
    \item Riguardo al fattore $L$, ricordiamo che $L = L_1^{-1} \dots L_{n-1}^{-1}$ e $L_i^{-1} = I + \mathbf{g}_i \mathbf{e}_i^T$.
    \item Poiché le prime $i$ componenti di $\mathbf{g}_i$ sono nulle, si ha $\mathbf{e}_j^T \mathbf{g}_i = 0$ per $j \le i$.
\end{itemize}
Consideriamo il prodotto $L = (I + \mathbf{g}_1 \mathbf{e}_1^T) (I + \mathbf{g}_2 \mathbf{e}_2^T) \dots (I + \mathbf{g}_{n-1} \mathbf{e}_{n-1}^T)$.
Per $n=3$, $L = (I + \mathbf{g}_1 \mathbf{e}_1^T) (I + \mathbf{g}_2 \mathbf{e}_2^T) = I + \mathbf{g}_1 \mathbf{e}_1^T + \mathbf{g}_2 \mathbf{e}_2^T + \mathbf{g}_1 (\mathbf{e}_1^T \mathbf{g}_2) \mathbf{e}_2^T$.
Ma $\mathbf{e}_1^T \mathbf{g}_2 = 0$ (poiché la prima componente di $\mathbf{g}_2$ è 0).
Questa proprietà vale in generale, quindi il prodotto si semplifica:
\begin{equation}
    L = I + \sum_{i=1}^{n-1} \mathbf{g}_i \mathbf{e}_i^T 
\end{equation}
Pertanto, al passo $i$-esimo della fattorizzazione possiamo riscrivere gli $(n-i)$ elementi, al di sotto di quello diagonale, in colonna $i$, con gli elementi significativi di $\mathbf{g}_i$.
Di conseguenza, alla fine dell'algoritmo, avremo riscritto gli elementi della porzione strettamente triangolare inferiore di $A$, con la porzione strettamente triangolare inferiore di $L$.
Evidentemente, la diagonale di $L$, che sappiamo a priori essere unitaria, non necessita di essere memorizzata esplicitamente.

In conclusione, la matrice $A$ può essere sovrascritta con l'informazione dei suoi fattori $L$ e $U$.
\paragraph{Numero di Operazioni (Flops)}
L'operazione $A^{(i+1)} = L_i A^{(i)}$ equivale a:
$$ A^{(i+1)} = (I - \mathbf{g}_i \mathbf{e}_i^T) A^{(i)} = A^{(i)} - \mathbf{g}_i (\mathbf{e}_i^T A^{(i)}) $$
Questa operazione aggiorna solo la sottomatrice $(n-i) \times (n-i)$ in basso a destra.
Lo pseudo-codice (supponendo $A$ $n \times n$) è:

\begin{lstlisting}[language=matlab]
% Algoritmo di Fattorizzazione LU (sovrascrive A)
for i = 1:n-1
    % passi di eliminazione
    if A(i,i) == 0
        error('A non fattorizzabile'); 
    end
    
    % Calcolo dei moltiplicatori (vettore g_i)
    % e memorizzazione in A
    A(i+1:n, i) = A(i+1:n, i) / A(i,i);
    
    % Aggiornamento della sottomatrice (operazione rank-1)
    A(i+1:n, i+1:n) = A(i+1:n, i+1:n) - ...
                       A(i+1:n, i) * A(i, i+1:n);
end
\end{lstlisting}

Analizziamo le operazioni all'iterazione $i$:
\begin{itemize}
    \item $(n-i)$ divisioni (per calcolare $\mathbf{g}_i$).
    \item $2(n-i)^2$ flops per l'aggiornamento della sottomatrice (un prodotto esterno $A(i+1:n, i) * A(i, i+1:n)$ che costa $(n-i)^2$ moltiplicazioni, e una sottrazione matriciale che costa $(n-i)^2$ sottrazioni).
\end{itemize}
Il costo totale è dominato dagli aggiornamenti:
$$ \sum_{i=1}^{n-1} 2(n-i)^2 = 2 \sum_{j=1}^{n-1} j^2 \approx 2 \int_1^{n-1} x^2 dx \approx \frac{2(n-1)^3}{3} \approx \frac{2n^3}{3} \text{ flops} $$
(Usando $\sum_{i=1}^{n} i^k \approx \int_1^n x^k dx \approx \frac{n^{k+1}}{k+1}$).

\subsubsection{Teorema di Esistenza della Fattorizzazione LU}
La fattorizzazione $A=LU$ (tramite MEG) esiste se e solo se $a_{ii}^{(i)} \neq 0$ per $i=1, \dots, n-1$, che (essendo $A$ nonsingolare) equivale a $\det(U) \neq 0$.

Denotiamo con $A_k \in \mathbb{R}^{k \times k}$ la \textbf{sottomatrice principale di ordine $k$} di $A$, ottenuta come intersezione delle sue prime $k$ righe e $k$ colonne.
(Es. $A_1 = (a_{11})$, $A_2 = \begin{pmatrix} a_{11} & a_{12} \\ a_{21} & a_{22} \end{pmatrix}$, $A_n = A$).

\begin{osservazione}
Questa sottomatrice può essere estratta usando la moltiplicazione a blocchi. Se $L$ e $U$ sono partizionate come segue (dove $L_k$ e $U_k$ sono le sottomatrici principali di ordine $k$):
$$ L = \begin{pmatrix} L_k & O \\ \dots & \dots \end{pmatrix} \quad \text{e} \quad U = \begin{pmatrix} U_k & \dots \\ O & \dots \end{pmatrix} $$
allora:
\begin{align*}
    A_k &= \begin{bmatrix} I_k & O_{k, n-k} \end{bmatrix} A \begin{bmatrix} I_k \\ O_{n-k, k} \end{bmatrix} \\
        &= \left( \begin{bmatrix} I_k & O_{k, n-k} \end{bmatrix} L \right) \left( U \begin{bmatrix} I_k \\ O_{n-k, k} \end{bmatrix} \right) \\
        &= \begin{bmatrix} L_k & O_{k, n-k} \end{bmatrix} \begin{bmatrix} U_k \\ O_{n-k, k} \end{bmatrix} = L_k U_k
\end{align*}
\end{osservazione}
Definiamo \textbf{minore principale} di ordine $k$ il $\det(A_k)$, cioè il determinante della sottomatrice principale di ordine k.
Segue che:
\begin{align*}
    \det(A_k) &= \det(L_k U_k) \\
              &= \underbrace{\det(L_k)}_{=1} \cdot \det(U_k) \quad \text{(perché L è a diagonale unitaria)} \\
              &= \det(U_k) = \prod_{i=1}^k u_{ii} = \prod_{i=1}^k a_{ii}^{(i)}, \quad \forall k=1, \dots, n
\end{align*}
A questo punto, osserviamo che la condizione $\det(U) \neq 0$ (necessaria per l'esistenza del MEG) è equivalente a:
\begin{align*}
    \det(U) = \prod_{i=1}^n a_{ii}^{(i)} \neq 0 \quad &\Leftrightarrow \quad \forall k=1, \dots, n: \prod_{i=1}^k a_{ii}^{(i)} \neq 0 \\
    &\Leftrightarrow \quad \forall k=1, \dots, n: \det(U_k) \neq 0 \\
    &\Leftrightarrow \quad \forall k=1, \dots, n: \det(A_k) \neq 0
\end{align*}
In altri termini, abbiamo dimostrato il seguente risultato.

\begin{teorema}[Esistenza della Fattorizzazione LU]
Data una matrice nonsingolare $A$, $A$ è fattorizzabile LU (nella forma $A=LU$ con $L$ unitaria) se e solo se tutti i suoi minori principali sono non nulli.
\end{teorema}

\begin{osservazione}
Affinché il sistema $A\mathbf{x} = \mathbf{b}$ ($A \in \mathbb{R}^{n \times n}$) abbia soluzione unica, è necessario e sufficiente che $\det(A) = \det(A_n) \neq 0$.
Tuttavia, se vogliamo fattorizzare $A=LU$ per risolverlo, si richiede la condizione (generalmente molto più restrittiva) che $\det(A_k) \neq 0$ per ogni $k=1, \dots, n$.

Esistono importanti classi di matrici per cui questa condizione è sempre verificata:
\begin{enumerate}
    \item La nonsingolarità di $A$ deriva da una proprietà strutturale della matrice.
    \item Tutte le sottomatrici principali di $A$ godono della medesima proprietà .
\end{enumerate}
Questo avviene per le matrici a \textbf{diagonale dominante} e per le matrici \textbf{simmetriche e definite positive}.
\end{osservazione}

%lezione del 12 novembre
\subsubsection{Matrici a Diagonale Dominante}
\begin{definition}
    Una matrice $A=(a_{ij}) \in \mathbb{R}^{n \times n}$ si dice a \textbf{diagonale dominante}:
    \begin{itemize}
        \item \textbf{per righe} se, per ogni riga, il valore assoluto dell'elemento sulla diagonale principale è strettamente maggiore della somma dei valori assoluti di tutti gli altri elementi su quella riga.
        \[ |a_{ii}| > \sum_{j \neq i} |a_{ij}|, \quad \forall i=1, \dots, n \]
        \textbf{Esempio:}
        \[ A = \begin{pmatrix} -3 & 2 & 0 \\ 4 & -7 & 1 \\ 1 & -5 & 8 \end{pmatrix} \]
        
        \item \textbf{per colonne} se, per ogni colonna, il valore assoluto dell'elemento sulla diagonale principale è strettamente maggiore della somma dei valori assoluti di tutti gli altri elementi in quella colonna.
        \[ |a_{ii}| > \sum_{j \neq i} |a_{ji}|, \quad \forall i=1, \dots, n \]
        \textbf{Esempio:}
        \[ A = \begin{pmatrix} 2 & 8 & 7 \\ 1 & -9 & 0 \\ 0 & 0 & 8 \end{pmatrix} \]
       
    \end{itemize}
    \end{definition}

\paragraph{Proprietà:}
\begin{enumerate}
    \item $A$ è d.d. per righe $\iff A^T$ è d.d. per colonne.
    \item Se $A$ è d.d. (per righe o per colonne), allora $\forall k=1, \dots, n$, la sottomatrice principale $A_k$ è d.d. (per righe o per colonne).
    \item Se $A$ è d.d. (per righe o per colonne), allora $\det(A) \neq 0$ (A è nonsingolare).
\end{enumerate}

Dalle proprietà 2) e 3) segue immediatamente il seguente teorema:
\begin{teorema}
Se $A$ è a diagonale dominante (per righe o per colonne), allora $A$ è fattorizzabile $LU$.
\end{teorema}
\begin{proof}[Dimostrazione (Proprietà 3)]
Poiché $\det(A) = \det(A^T)$ (e per la Prop. 1), possiamo considerare solo il caso in cui $A$ è d.d. per righe. Supponiamo per assurdo che $\det(A)=0$. Ma allora $\exists \mathbf{x} \in \mathbb{R}^n, \mathbf{x} \neq \mathbf{0}$ tale che $A\mathbf{x} = \mathbf{0}$.
Poiché $A(\alpha \mathbf{x}) = \mathbf{0}$ per ogni scalare $\alpha$, possiamo scegliere $\mathbf{x}$ normalizzato in modo che la sua componente di modulo massimo sia 1:
$$ x_k : |x_k| = \max_{i=1,\dots,n} |x_i| = 1 \implies |x_j| \le 1, \forall j $$
Consideriamo la $k$-esima equazione del sistema $A\mathbf{x} = \mathbf{0}$:
$$ (\mathbf{e}_k^T A) \mathbf{x} = 0 \implies \sum_{j=1}^n a_{kj} x_j = 0 $$
Isoliamo il termine diagonale: $a_{kk} x_k = - \sum_{j \neq k} a_{kj} x_j$. Passando ai moduli e usando $x_k=1$:
$$ |a_{kk}| = |a_{kk} \cdot 1| = \left| - \sum_{j \neq k} a_{kj} x_j \right| = \left| \sum_{j \neq k} a_{kj} x_j \right| \le \sum_{j \neq k} |a_{kj}| |x_j| $$
Poiché $|x_j| \le 1$ per ogni $j$:
$$ |a_{kk}| \le \sum_{j \neq k} |a_{kj}| \cdot 1 = \sum_{j \neq k} |a_{kj}| $$
Questo contraddice l'ipotesi che $A$ sia a diagonale dominante per righe (in particolare sulla riga $k$-esima: $|a_{kk}| > \sum_{j \neq k} |a_{kj}|$). Pertanto, deve aversi $\det(A) \neq 0$.
\end{proof}

\subsubsection{Matrici Simmetriche e Definite Positive (sdp)}
\begin{definition}
Diremo che $A=(a_{ij}) \in \mathbb{R}^{n \times n}$ è \textbf{sdp} se:
\begin{enumerate}
    \item $A = A^T$ (simmetria, ovvero $a_{ij} = a_{ji}$);
    \item $\forall \mathbf{x} \in \mathbb{R}^n, \mathbf{x} \neq \mathbf{0} : \mathbf{x}^T A \mathbf{x} > 0$ (definita positività).
\end{enumerate}
\end{definition}

\paragraph{Proprietà:}
\begin{enumerate}
    \item $A \text{ sdp} \implies \forall k=1, \dots, n : A_k \text{ è sdp}$.
    \item $A \text{ sdp} \implies \det(A) \neq 0$.
    \item Dalle 1) e 2) segue che: $A \text{ sdp} \implies A = LU$.
\end{enumerate}

\begin{proof}[Dimostrazione (Proprietà 2)]
Supponiamo per assurdo $\det(A)=0$. Allora $\exists \mathbf{x} \neq \mathbf{0}$ tale che $A\mathbf{x} = \mathbf{0}$. Ma allora $\mathbf{x}^T A \mathbf{x} = \mathbf{x}^T \mathbf{0} = 0$, il che contraddice l'ipotesi che $A$ sia definita positiva. Pertanto $\det(A) \neq 0$.
\end{proof}

\begin{proof}[Dimostrazione (Proprietà 1)]
Per un generico $k$, partizioniamo $A$ a blocchi: $A = \begin{pmatrix} A_k & B \\ C & D \end{pmatrix}$.
Dalla simmetria $A=A^T$, si ha $\begin{pmatrix} A_k & B \\ C & D \end{pmatrix} = \begin{pmatrix} A_k^T & C^T \\ B^T & D^T \end{pmatrix}$. Uguagliando i blocchi omologhi: $A_k = A_k^T$ (quindi $A_k$ è simmetrica).
Rimane da dimostrare che $A_k$ è definita positiva. Scegliamo un generico $\mathbf{y} \in \mathbb{R}^k, \mathbf{y} \neq \mathbf{0}$. Costruiamo $\mathbf{x} = \begin{pmatrix} \mathbf{y} \\ \mathbf{0} \end{pmatrix} \in \mathbb{R}^n$. Chiaramente $\mathbf{x} \neq \mathbf{0}$.
Calcoliamo:
$$ 0 < \mathbf{x}^T A \mathbf{x} = \begin{pmatrix} \mathbf{y}^T & \mathbf{0}^T \end{pmatrix} \begin{pmatrix} A_k & C^T \\ C & D \end{pmatrix} \begin{pmatrix} \mathbf{y} \\ \mathbf{0} \end{pmatrix} = \begin{pmatrix} \mathbf{y}^T A_k & \mathbf{y}^T C^T \end{pmatrix} \begin{pmatrix} \mathbf{y} \\ \mathbf{0} \end{pmatrix} = \mathbf{y}^T A_k \mathbf{y} $$
Quindi $\mathbf{y}^T A_k \mathbf{y} > 0$, e $A_k$ è definita positiva.
\end{proof}

\paragraph{Ulteriori Proprietà delle Matrici sdp}

\begin{teorema}
Se $A=(a_{ij})$ è sdp $\implies \forall i=1, \dots, n: a_{ii} > 0$.
\end{teorema}
\begin{proof}
Infatti, $\forall i=1, \dots, n$, sia $\mathbf{e}_i$ l'$i$-esimo versore ($\mathbf{e}_i \neq \mathbf{0}$). Allora:
$$ a_{ii} = \mathbf{e}_i^T A \mathbf{e}_i > 0 \quad (\text{per la definizione di sdp}) $$
\end{proof}

\begin{teorema}[Fattorizzazione $LDL^T$]
$A$ è sdp $\iff \exists L$ triangolare inferiore a diagonale unitaria, e $D = \text{diag}(d_1, \dots, d_n)$ matrice diagonale con $d_i > 0$ per $i=1, \dots, n$, tali che:
$$ A = L D L^T $$
\end{teorema}
\begin{proof}
($\Leftarrow$) Dimostriamo che se $A=LDL^T$ con $L$ unitaria e $D$ positiva, allora $A$ è sdp.
1) Simmetria: $A^T = (LDL^T)^T = (L^T)^T D^T L^T = L D L^T = A$.
2) Def. Positività: $\forall \mathbf{x} \neq \mathbf{0}$, sia $\mathbf{y} = L^T \mathbf{x}$. Poiché $L^T$ è nonsingolare, $\mathbf{x} \neq \mathbf{0} \implies \mathbf{y} \neq \mathbf{0}$.
$$ \mathbf{x}^T A \mathbf{x} = \mathbf{x}^T (L D L^T) \mathbf{x} = (\mathbf{x}^T L) D (L^T \mathbf{x}) = \mathbf{y}^T D \mathbf{y} = \sum_{i=1}^n d_i y_i^2 > 0 $$
(poiché $d_i > 0$ e almeno uno $y_i \neq 0$).

($\Rightarrow$) Dimostriamo $A \text{ sdp} \implies A=LDL^T$, con $L$ e $D$ come nell'enunciato.
Abbiamo visto che se $A$ è sdp, allora $A=LU$, con $L$ triangolare inferiore a diagonale unitaria e $U$ triangolare superiore (e nonsingolare).
Osserviamo che, se $U=(u_{ij}) \in \mathbb{R}^{n \times n}$, allora $U$ può essere fattorizzata come:
$$ U = D \hat{U} $$
con $D = \text{diag}(u_{11}, \dots, u_{nn})$. Ne consegue che $\hat{U}$ sarà triangolare superiore a diagonale unitaria.

\begin{esempio}
\[ U = \begin{pmatrix} 1 & 2 & 3 \\ 0 & 4 & 5 \\ 0 & 0 & 6 \end{pmatrix} = 
\underbrace{\begin{pmatrix} 1 & 0 & 0 \\ 0 & 4 & 0 \\ 0 & 0 & 6 \end{pmatrix}}_{D}
\underbrace{\begin{pmatrix} 1 & 2 & 3 \\ 0 & 1 & 5/4 \\ 0 & 0 & 1 \end{pmatrix}}_{\hat{U}}
\]
\end{esempio}

Pertanto, $A = LU = L D \hat{U}$.
Poiché $A$ è sdp, $A = A^T$. Calcoliamo $A^T$:
$$ A^T = (L D \hat{U})^T = \hat{U}^T D^T L^T = \hat{U}^T (D L^T) \quad (\text{poiché } D \text{ è diagonale, } D^T=D) $$
A questo punto, osserviamo che:
\begin{enumerate}
    \item $\hat{U}^T$ è triangolare inferiore a diagonale unitaria.
    \item $D L^T$ è triangolare superiore.
    \item La fattorizzazione LU (con $L$ unitaria) è unica.
\end{enumerate}
Confrontando le due fattorizzazioni di $A$, $A = L (D \hat{U})$ e $A = \hat{U}^T (D L^T)$, per l'unicità concludiamo che:
$$ \hat{U}^T = L \quad \land \quad D L^T = U $$
Quindi, $A = L D L^T$.

Rimane, quindi, da dimostrare che gli elementi diagonali $d_i$ di $D$ sono positivi.
A questo fine, osserviamo che $\forall i=1, \dots, n$, poiché $L^T$ è nonsingolare (essendo triangolare superiore con diagonale unitaria), esiste un unico $\mathbf{x} \neq \mathbf{0}$ tale che $L^T \mathbf{x} = \mathbf{e}_i$.
Pertanto:
\begin{align*}
    0 &< \mathbf{x}^T A \mathbf{x} \quad (\text{perché } A \text{ è sdp e } \mathbf{x} \neq \mathbf{0}) \\
      &= \mathbf{x}^T L D L^T \mathbf{x} \\
      &= (L^T \mathbf{x})^T D (L^T \mathbf{x}) \\
      &= \mathbf{e}_i^T D \mathbf{e}_i = d_i
\end{align*}
Poiché $i$ è generico, l'asserto segue.
\end{proof}

\begin{osservazione}
Se $A$ è sdp, e quindi $A=LDL^T$, il fattore $U$ non è più da calcolare.
\end{osservazione}

\end{document}