\section{Matlab (Matrix Laboratory)}
Matlab è un linguaggio interpretato orientato alla gestione efficiente delle matrici. Tutte le variabili sono definite in un'area di lavoro (workspace) e non è necessario dichiararne il tipo o la dimensione in anticipo.

\subsection{Gestione delle Variabili e della Workspace}
Il dimensionamento delle variabili è automatico e si adatta alle operazioni. Ad esempio, eseguendo `A(6,7) = 3.14`, Matlab crea una matrice `A` di dimensioni $6 \times 7$. Gli elementi non specificati vengono inizializzati a zero. La rappresentazione interna dei numeri è conforme allo standard IEEE 754 in doppia precisione.

Per ispezionare la workspace si usano i comandi:
\begin{itemize}
    \item \texttt{who}: mostra i nomi delle variabili presenti.
    \item \texttt{whos}: mostra una tabella dettagliata con nome, dimensione, tipo e memoria occupata.
\end{itemize}

È buona norma, per ragioni di efficienza, pre-allocare lo spazio per le matrici se le loro dimensioni finali sono note. Ad esempio:
\begin{lstlisting}
% Inizializza una matrice 200x300 con tutti zeri
A = zeros(200, 300);
\end{lstlisting}
Per sopprimere la visualizzazione del risultato di un'operazione, si termina la riga con un punto e virgola (\texttt{;}).

Per salvare o caricare la workspace si usano i comandi:
\begin{itemize}
    \item \texttt{save <nomefile>}: salva tutte le variabili nel file \texttt{nomefile.mat}.
    \item \texttt{load <nomefile>}: carica le variabili dal file specificato.
\end{itemize}

\paragraph{Formato di Visualizzazione}
Il comando \texttt{format} controlla come i valori numerici vengono visualizzati nella finestra di comando. È importante sottolineare che questo comando modifica solo la visualizzazione e non la precisione con cui i numeri sono memorizzati (che rimane in doppia precisione).

Alcune delle opzioni più comuni sono:
\begin{itemize}
    \item \texttt{format short}: formato a punto fisso con 5 cifre (default).
    \item \texttt{format long}: formato a punto fisso con 15 cifre.
    \item \texttt{format short e}: notazione scientifica con 5 cifre.
    \item \texttt{format long e}: notazione scientifica con 15 cifre.
\end{itemize}

\subsection{Creazione e Manipolazione di Matrici}
Uno dei comandi più utili in Matlab è \lstinline|help|. Questo comando permette di visualizzare la documentazione di una qualsiasi funzione direttamente nella finestra di comando. Ad esempio, per ottenere informazioni sulla funzione \lstinline|zeros|, si digita \lstinline|help zeros|.
\paragraph{Funzioni di base}
\begin{itemize}
    \item \texttt{zeros(m, n)}: crea una matrice $m \times n$ di zeri.
    \item \texttt{eye(n)}: crea una matrice identità $n \times n$.
    \item \texttt{rand(m, n)}: crea una matrice $m \times n$ con elementi casuali da una distribuzione uniformeme.
\end{itemize}

\paragraph{Operatore Colon (:)}
L'operatore \texttt{:} è estremamente versatile per creare vettori e selezionare sottomatrici.
\begin{lstlisting}
v1 = 1:5;       % Crea il vettore [1 2 3 4 5]
v2 = 5:-1:1;    % Crea il vettore [5 4 3 2 1]
v3 = 2:2:20;    % Crea il vettore dei numeri pari da 2 a 20
\end{lstlisting}

\paragraph{Indicizzazione e Concatenazione}
Si possono estrarre righe, colonne o sottomatrici e concatenare matrici esistenti.
\begin{lstlisting}
A = rand(3);
B = rand(3,1);
C = rand(2,4);

% Concatenazione orizzontale e verticale
D = [A B]; % Concatena A e B orizzontalmente (devono avere stesso numero di righe)
E = [A; rand(1,3)]; % Concatena A e una nuova riga verticalmente

% Estrazione
seconda_riga = D(2, :);
terza_colonna = D(:, 3);
sottomatrice = D(1:2, 1:3); % Estrae le prime due righe e le prime tre colonne
\end{lstlisting}

\subsection{Operatori ed Espressioni}
\begin{itemize}
    \item \textbf{Operatori Aritmetici}: \verb|+|, \verb|-|, \verb|*|, \verb|/|, \verb|\|
    \item \textbf{Operazioni con Scalari}: Le operazioni tra una matrice e uno scalare sono applicate a ogni elemento. Es: \texttt{C = A + 1;}, \texttt{C = 2 * A;}.
    \item \textbf{Operatori Element-wise}: Se preceduti da un punto (\texttt{.}), gli operatori agiscono elemento per elemento. Richiedono che le matrici abbiano le stesse dimensioni.
    \begin{itemize}
        \item \texttt{C = A .* B} $\implies C(i,j) = A(i,j) * B(i,j)$
        \item \texttt{C = A ./ B} $\implies C(i,j) = A(i,j) / B(i,j)$
    \end{itemize}
    \item \textbf{Divisione Matriciale}:
    \begin{itemize}
        \item \verb|C = A \ B| (backslash) risolve il sistema lineare $AX = B$, equivalente a $A^{-1}B$.
        \item \texttt{C = A / B} (slash) è equivalente a $AB^{-1}$.
    \end{itemize}
\end{itemize}

\subsection{Funzioni Utili}
Le funzioni in Matlab possono essere raggruppate in:
\begin{itemize}
    \item \textbf{Funzioni orientate a scalari}: operano su ogni elemento di una matrice. Es: \texttt{sin(A)}, \texttt{cos(A)}, \texttt{exp(A)}, \texttt{log(A)}, \texttt{sqrt(A)}. Anche le funzioni di arrotondamento come \texttt{round}, \texttt{floor}, \texttt{ceil}, \texttt{fix} rientrano in questa categoria.
    \item \textbf{Funzioni orientate a vettori}: se applicate a una matrice, operano colonna per colonna restituendo un vettore riga. Es: \texttt{max(A)}, \texttt{min(A)}, \texttt{mean(A)}, \texttt{sum(A)}, \texttt{sort(A)}.
    \item \textbf{Funzioni orientate a matrici}: operano sull'intera matrice. Es: \texttt{size(A)}, \texttt{det(A)}, \texttt{inv(A)}, \texttt{diag(A)}.
\end{itemize}

\subsection{Grafica 2D}
Matlab offre potenti strumenti per la visualizzazione grafica.
\begin{itemize}
    \item \texttt{plot(x, y)}: disegna la spezzata che congiunge i punti definiti dai vettori \texttt{x} e \texttt{y}.
    \item \texttt{semilogx}, \texttt{semilogy}, \texttt{loglog}: disegnano grafici con scale logaritmiche sugli assi.
    \item \texttt{xlabel('testo')}, \texttt{ylabel('testo')}, \texttt{title('titolo')}: aggiungono etichette agli assi e un titolo al grafico.
    \item \texttt{legend('curva1', 'curva2')}: aggiunge una legenda.
    \item \texttt{axis([xmin xmax ymin ymax])}: imposta i limiti degli assi.
    \item \texttt{figure}: apre una nuova finestra grafica.
\end{itemize}

\subsection{Controllo del Flusso e m-files}
Matlab fornisce costrutti per il controllo del flusso di esecuzione (selezioni e iterazioni) e permette di salvare sequenze di comandi in file con estensione \texttt{.m}, detti \textbf{m-files}.

\subsubsection{Selezioni}
Il costrutto di selezione principale è \texttt{if-elseif-else}.
\begin{lstlisting}[frame=none, numbers=none]
if <espressione_booleana_1>
    % blocco di istruzioni 1
elseif <espressione_booleana_2>
    % blocco di istruzioni 2
else
    % blocco di istruzioni 3
end
\end{lstlisting}
Le espressioni booleane si ottengono combinando operatori relazionali (\verb|>|, \verb|<|, \verb|>=|, \verb|<=|, \verb|==|, \verb|~=|) e operatori booleani (\verb|&&| per AND, \verb#||# per OR, \verb|~| per NOT). È buona norma usare le parentesi \verb|()| per definire chiaramente l'ordine di valutazione.

\subsubsection{Iterazioni (Cicli)}
I cicli principali sono \texttt{for} e \texttt{while}.
\paragraph{Il ciclo \texttt{for}} Esegue un blocco di istruzioni per ogni elemento di un vettore.
\begin{lstlisting}
% Esempio: calcola il quadrato dei numeri pari da 10 a 2
a = zeros(10,1);
for i = 10:-2:2
    a(i) = i^2;
end
\end{lstlisting}
Il comando \texttt{break} permette di uscire immediatamente dal ciclo che lo contiene.

\paragraph{Il ciclo \texttt{while}} Esegue un blocco di istruzioni finché una condizione booleana rimane vera.
\begin{lstlisting}[frame=none, numbers=none]
while <condizione_booleana>
    % blocco di istruzioni
    % (eventuale `break` per uscita forzata)
end
\end{lstlisting}

\subsubsection{m-files: Script e Function}
Esistono due tipi di m-files.
\begin{itemize}
    \item \textbf{Script-file}: È una semplice sequenza di comandi. Quando eseguito, opera direttamente sulle variabili presenti nella workspace (le sue variabili sono \textbf{globali}).
    \item \textbf{Function-file}: È un blocco di codice più strutturato che comunica con l'esterno solo tramite parametri di input e output. Tutte le variabili definite al suo interno sono \textbf{locali} e vengono distrutte al termine dell'esecuzione.
\end{itemize}

\paragraph{Anatomia di una Function}
Una function è definita dalla seguente intestazione:
\begin{lstlisting}[frame=none, numbers=none]
function [out1, out2, ...] = nomeFunzione(in1, in2, ...)
\end{lstlisting}
All'interno di una function si possono usare:
\begin{itemize}
    \item \texttt{nargin}: restituisce il numero di argomenti di input forniti alla chiamata.
    \item \texttt{nargout}: restituisce il numero di argomenti di output richiesti alla chiamata.
    \item \texttt{\%}: introduce un commento (il resto della riga viene ignorato).
    \item \texttt{...} (tre punti): permettono di spezzare una singola istruzione su più righe.
    \item \texttt{return}: termina l'esecuzione della function.
\end{itemize}

\paragraph{Help in linea}
Le prime righe di commento consecutive dopo l'intestazione di una function costituiscono il suo \textbf{help in linea}, che viene visualizzato quando si digita \texttt{help nomeFunzione}.

\paragraph{Esempio di Function}
Questo esempio definisce una function che calcola media e varianza di un vettore di dati, gestendo il numero di input e output.

\begin{lstlisting}
function [media, varianza] = esempio(dati)
% [media, varianza] = esempio(dati)
% Calcola la media e la varianza di un insieme di dati.
%
% Input:
%   dati - vettore con i dati di ingresso;
% Output:
%   media    - media dei dati;
%   varianza - varianza dei dati.

% Controllo degli input
if nargin < 1
    error('Dati di ingresso insufficienti');
else
    if isempty(dati), error('Il vettore di dati è vuoto'), end
end

media = mean(dati);

% Calcola la varianza solo se richiesta in output
if nargout > 1
    varianza = var(dati);
end

return
\end{lstlisting}

Digitando \texttt{help esempio} nella console di Matlab, si otterrà il testo scritto nelle prime righe di commento della funzione.
